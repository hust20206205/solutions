Phần mềm được tạo ra để xử lý sự phức tạp trong cuộc sống hiện đại. Việc phát triển phần mềm liên kết chặt chẽ với một số khía cạnh cụ thể trong cuộc sống của chúng ta.

Miền (Domain) đề cập đến phạm vi kiến thức và vấn đề mà hệ thống hoặc dự án cụ thể đang xử lý.

Về góc độ kinh doanh: miền đại diện cho một lĩnh vực hoặc ngành mà doanh nghiệp hoạt động.
Về góc độ phần mềm: miền có thể coi là đại diện cho không gian vấn đề của phần mềm đó.

Phần mềm cần phản ánh đúng miền và hiện thực hóa chính xác miền.

%!<! - - $VD: Ở đồ án này, miền được xác định là bài toán giải pháp hóa đơn điện tử. - - >

% %!<! - - Domain : https:// thiết kế hướng miền - practitioners.com/domain - - >
% %!<! - - Domain : https:// thiết kế hướng miền - practitioners.com/domain - - >
% %!<! - - Domain : https:// thiết kế hướng miền - practitioners.com/domain - - >
% %!<! - - Domain : https:// thiết kế hướng miền - practitioners.com/domain - - >
% %!<! - - Domain : https:// thiết kế hướng miền - practitioners.com/domain - - >
% %!<! - - Domain : https:// thiết kế hướng miền - practitioners.com/domain - - >
% %!<! - - Domain : https:// thiết kế hướng miền - practitioners.com/domain - - >
% %!<! - - Domain : https:// thiết kế hướng miền - practitioners.com/domain - - >

%!<! - - [[Domain]] A sphere of knowledge, influence, or activity. - - >

Chuyển đến nội dung
Đối với người hành nghề bởi người hành nghề
Tìm kiếm
Thiết kế hướng miền: Hướng dẫn dành cho người thực hành
Câu hỏi thường gặp
Bảng chú giải
Về chúng tôi
Cuốn sách của chúng tôi!
Trang chủTrang chủBảng chú giảiLãnh địa
Lãnh địa
Trong thiết kế hướng miền (thiết kế hướng miền), miền là một phạm vi kiến thức, ảnh hưởng hoặc hoạt động đại diện cho một lĩnh vực chuyên môn hoặc mối quan tâm cụ thể. Miền là không gian vấn đề mà phần mềm đang được xây dựng để giải quyết và nó thể hiện vấn đề hoặc cơ hội kinh doanh mà phần mềm có nhiệm vụ giải quyết.

Miền có thể là một ngành cụ thể, chẳng hạn như chăm sóc sức khỏe, tài chính hoặc thương mại điện tử hoặc có thể là một lĩnh vực quan tâm cụ thể trong một ngành, chẳng hạn như quản lý hàng tồn kho, quản lý khách hàng hoặc hậu cần.

Trong thiết kế hướng miền, mục tiêu là tạo ra một mô hình miền thể hiện chính xác các khái niệm và mối quan hệ trong thế giới thực trong miền và sử dụng mô hình này làm cơ sở cho việc thiết kế và phát triển phần mềm. Mô hình miền này phải phản ánh miền kinh doanh, nó phải dễ hiểu đối với các chuyên gia miền, nhà phát triển và các bên liên quan và nó phải là xương sống của hệ thống phần mềm.

Một miền có thể được xác định bởi các quy tắc kinh doanh, quy trình kinh doanh, các thực thể kinh doanh và các mối quan hệ của chúng, các yêu cầu kinh doanh và mục tiêu kinh doanh.

Miền là trái tim của thiết kế hướng miền, là nền tảng của phần mềm và là điểm khởi đầu của quá trình phát triển. Hiểu miền là rất quan trọng để có thể tạo ra một phần mềm phù hợp với nhu cầu kinh doanh, ít phức tạp hơn, dễ bảo trì hơn và dễ thích ứng hơn với thay đổi.

Xem thêm	 Miền phụ, miền vấn đề, miền giải pháp, bối cảnh giới hạn

Thể loại

Phân tích
điều cơ bản
thiết kế hướng miền
thiết kế
câu hỏi thường gặp
Khả năng lãnh đạo
hoa văn
Blog tại WordPress.com.