Bối cảnh ranh giới cung cấp các dịch vụ chung được gọi là dịch vụ nguồn mở

mô tả dịch vụ chung này dưới dạng mẫu được đặt trước bối cảnh giới hạn ngược dòng cung cấp các dịch vụ chung, bối cảnh giới hạn ngược dòng hoặc nhà cung cấp dịch vụ được lưu trữ mở trong mối quan hệ này cung cấp một ngôn ngữ chung để tích hợp.

Đối tác đầu tiên, mẫu dịch vụ được lưu trữ mở, trong đó bối cảnh kết hợp ngược dòng cung cấp một tập hợp các dịch vụ chung hoặc khả năng chung cho bối cảnh giới hạn xuôi dòng.

%! D - OHS

%! @\06BoundedContextRelationships_VVN\000000005.srt

1

00: 00: 00, 150 00: 00: 17, 640

Một đối nhiều, mối quan hệ trong bài giảng này nhấn mạnh hai mẫu khảo sát nội bộ mở và mẫu ngôn ngữ được công bố trong mối quan hệ một đến nhiều liên hệ giới hạn ngược dòng được gọi là Provida cung cấp các dịch vụ chung cho hai hoặc nhiều bối cảnh giới hạn .

2

00: 00: 17, 670 00: 00: 27, 300

Trong ví dụ này, Inbee là bối cảnh giới hạn xuôi dòng và C là bối cảnh ranh giới ngược dòng đang cung cấp các dịch vụ chung.

3

00: 00: 27, 430 00: 00: 42, 780

Và B các mô hình tích hợp chung được xác định bởi bối cảnh giới hạn . C Bất kể nhu cầu của A hoặc B, bối cảnh giới hạn ở hạ lưu có thể quyết định tuân thủ hoặc sử dụng athea.

4

00: 00: 42, 990 00: 00: 55, 340

Vì vậy, trong ví dụ này, A là người tuân thủ, trong khi B đang sử dụng lớp chống đổ vỡ . Những quyết định này được đưa ra bởi các nhóm làm việc độc lập trên các bối cảnh giới hạn này.

5

00: 00: 55, 590 00: 01: 25, 800

Bối cảnh ranh giới ngược dòng cung cấp các dịch vụ chung được gọi là dịch vụ nguồn mở và mẫu này được gọi là mẫu dịch vụ nguồn mở để mô tả dịch vụ chung này dưới dạng mẫu được đặt trước bối cảnh giới hạn ngược dòng cung cấp các dịch vụ chung, bối cảnh giới hạn ngược dòng hoặc nhà cung cấp dịch vụ được lưu trữ mở trong mối quan hệ này cung cấp một ngôn ngữ chung để tích hợp.

6

00: 01: 26, 030 00: 01: 39, 990

Ngôn ngữ chung này được các nhóm làm việc trong bối cảnh giới hạn ở hạ lưu chấp nhận. Ngôn ngữ chung này được gọi là ngôn ngữ được xuất bản và mẫu này được gọi là mẫu ngôn ngữ được xuất bản.

7

00: 01: 40, 650 00: 01: 48, 960

Việc sử dụng ngôn ngữ đã xuất bản được mô tả bằng cách đặt APL ở phía trước thượng nguồn cùng với Oireachtas.

8

00: 01: 49, 240 00: 02: 01, 500

Vì vậy, ánh xạ ngữ cảnh này mô tả một kịch bản trong đó C đang cung cấp các dịch vụ chung. Tức là nó có nhà cung cấp Oireachtas và cũng có ngôn ngữ được xuất bản.

9

00: 02: 02, 070 00: 02: 15, 580

Hãy gọi nó là một ví dụ về việc thực hiện các dịch vụ phổ biến. Chức năng này được hiển thị theo cách có bối cảnh giới hạn và được chấp nhận tốt bởi bối cảnh giới hạn phía dưới.

10

00: 02: 15, 660 00: 02: 30, 840

Vì vậy, hãy nghĩ đến một nhóm tạo ra một dịch vụ giúp bộc lộ những khoảng trống. Và giả sử nhóm này cũng đang xuất bản các lược đồ và mô hình API còn lại thông qua các thông số kỹ thuật API mở hoặc thông số kỹ thuật Swagga.

11

00: 02: 30, 840 00: 02: 44, 220

Các nhóm khác có thể xây dựng dịch vụ của riêng họ bằng cách sử dụng các công thức này và họ sẽ xây dựng các công thức này bằng cách tham khảo các thông số AP mở cho ứng dụng và lược đồ.

12

00: 02: 44, 220 00: 02: 52, 220

Và nhìn vào sơ đồ này, chúng ta có thể nói rằng chúng ta đã tạo API chính xác theo cách này và chúng ta hoàn toàn đúng.

13

00: 02: 52, 230 00: 02: 58, 230

Không có sự khác biệt trong bài học này. Chúng ta đã tìm hiểu về hai mẫu liên quan đến mối quan hệ một - nhiều.

14

00: 02: 58, 240 00: 03: 10, 710

Đối tác đầu tiên, mẫu dịch vụ được lưu trữ mở, trong đó bối cảnh kết hợp ngược dòng cung cấp một tập hợp các dịch vụ chung hoặc khả năng chung cho bối cảnh giới hạn xuôi dòng.

15

00: 03: 10, 800 00: 03: 34, 110

Ngôn ngữ thứ hai là ngôn ngữ được xuất bản, đi đôi với dịch vụ lưu trữ mở. Trở lại ngược dòng, các liên hệ được giới hạn trên nhà cung cấp dịch vụ được lưu trữ mở sẽ hiển thị ngôn ngữ chung cho các dịch vụ chung và ngôn ngữ này được quản lý bởi nhóm chịu trách nhiệm về dịch vụ được lưu trữ mở, các liên hệ được giới hạn ở hạ nguồn ngoại trừ ngôn ngữ được xuất bản này.

% %! @\06BoundedContextRelationships_VVN\000000006.srt

% %! @\06BoundedContextRelationships_VVN\000000006.srt

% %! @\06BoundedContextRelationships_VVN\000000006.srt

% %! @\06BoundedContextRelationships_VVN\000000006.srt

% %! @\06BoundedContextRelationships_VVN\000000006.srt

% %! @\06BoundedContextRelationships_VVN\000000006.srt

% %! @\06BoundedContextRelationships_VVN\000000006.srt

% %! @\06BoundedContextRelationships_VVN\000000006.srt

% %! @\06BoundedContextRelationships_VVN\000000006.srt

% %! Open - Host Service : https:// thiết kế hướng miền - practitioners.com/open - host - dịch vụ

% %! Open - Host Service : https:// thiết kế hướng miền - practitioners.com/open - host - dịch vụ

% %! Open - Host Service : https:// thiết kế hướng miền - practitioners.com/open - host - dịch vụ

% %! Open - Host Service : https:// thiết kế hướng miền - practitioners.com/open - host - dịch vụ

% %! Open - Host Service : https:// thiết kế hướng miền - practitioners.com/open - host - dịch vụ

Trang chủTrang chủBảng chú giảiBối cảnh giới hạn Mối quan hệ bối cảnh giới hạn Dịch vụ máy chủ mở

Dịch vụ máy chủ mở

Mối quan hệ dịch vụ máy chủ mở là một loại mối quan hệ giữa các bối cảnh giới hạn liên quan đến một bối cảnh giới hạn cung cấp dịch vụ mà một bối cảnh giới hạn khác có thể truy cập. Trong mối quan hệ này, một bối cảnh giới hạn được coi là “máy chủ” và bối cảnh giới hạn khác được coi là “khách hàng” hoặc “người tiêu dùng” của dịch vụ. Máy chủ cung cấp giao diện hoặc API để máy khách sử dụng và máy khách chịu trách nhiệm triển khai mã cần thiết để tương tác với máy chủ và sử dụng dịch vụ.

Mối quan hệ này thường được sử dụng khi một bối cảnh giới hạn có chuyên môn hoặc khả năng cụ thể hữu ích cho các bối cảnh giới hạn khác trong hệ thống. Bằng cách cung cấp dịch vụ máy chủ mở, bối cảnh giới hạn có chuyên môn có thể chia sẻ khả năng của nó với các bối cảnh giới hạn khác theo cách được kiểm soát và tiêu chuẩn hóa.

Mối quan hệ dịch vụ máy chủ mở có thể giúp giảm sự trùng lặp chức năng trên các bối cảnh giới hạn và thúc đẩy tính mô đun hóa cũng như khả năng sử dụng lại. Tuy nhiên, nó cũng đưa ra một số sự kết nối giữa các ngữ cảnh giới hạn, vì những thay đổi đối với dịch vụ do máy chủ cung cấp có thể yêu cầu những thay đổi đối với các máy khách phụ thuộc vào nó. Như với tất cả các mối quan hệ giữa các bối cảnh giới hạn, sự cân bằng giữa sự ghép nối và sự gắn kết phải được xem xét cẩn thận khi quyết định có nên sử dụng mối quan hệ dịch vụ máy chủ mở hay không.

Hàm ý

Mối quan hệ dịch vụ máy chủ mở là một loại mối quan hệ giữa các ngữ cảnh giới hạn trong Thiết kế hướng miền. Trong mối quan hệ này, một bối cảnh giới hạn sẽ hiển thị chức năng của nó như một dịch vụ cho một bối cảnh giới hạn khác. Dưới đây là một số ưu và nhược điểm của mối quan hệ này:

Ưu điểm

Cho phép phân tách rõ ràng các mối quan tâm giữa các bối cảnh giới hạn

Khuyến khích kiến trúc hướng dịch vụ và thúc đẩy sự liên kết lỏng lẻo giữa các hệ thống

Cho phép các bối cảnh giới hạn được phát triển và triển khai độc lập

Có thể tạo điều kiện cho khả năng mở rộng bằng cách cho phép các dịch vụ được phân phối trên nhiều hệ thống

Nhược điểm

Có thể tăng thêm độ phức tạp cho hệ thống, đặc biệt là về mặt giao tiếp giữa các dịch vụ

Yêu cầu hợp đồng dịch vụ được xác định rõ ràng để đảm bảo khả năng tương tác giữa các hệ thống

Có thể dẫn đến các vấn đề về hiệu suất nếu có nhiều giao tiếp giữa các dịch vụ

Có thể khiến việc duy trì tính nhất quán giữa các hệ thống trở nên khó khăn hơn, đặc biệt trong trường hợp có lỗi hoặc lỗi một phần

Mối quan hệ dịch vụ máy chủ mở có thể là một công cụ mạnh mẽ để thúc đẩy sự phân tách các mối quan tâm và cho phép phát triển và triển khai độc lập các bối cảnh giới hạn . Tuy nhiên, nó đòi hỏi phải lập kế hoạch và thiết kế cẩn thận để đảm bảo rằng lợi ích lớn hơn những hạn chế.

Ví dụ

Dưới đây là ví dụ về mối quan hệ dịch vụ máy chủ mở giữa các ngữ cảnh giới hạn :

Giả sử chúng ta có hai bối cảnh giới hạn : “Bán hàng” và “Hàng tồn kho”. Bối cảnh giới hạn “Bán hàng” chịu trách nhiệm quản lý các đơn đặt hàng và bối cảnh giới hạn “Hàng tồn kho” chịu trách nhiệm quản lý mức tồn kho của các mặt hàng.

Bối cảnh giới hạn “Bán hàng” cần biết mức tồn kho của các mặt hàng khi xử lý đơn đặt hàng. Thay vì kết hợp chặt chẽ hai bối cảnh giới hạn, chúng có thể thiết lập mối quan hệ dịch vụ máy chủ mở. Ngữ cảnh giới hạn “Hàng tồn kho” hiển thị một API cho ngữ cảnh giới hạn “Bán hàng” để gọi, cung cấp quyền truy cập vào mức tồn kho hiện tại của các mặt hàng.

Ưu điểm của phương pháp này là nó cho phép kết nối lỏng lẻo giữa các bối cảnh giới hạn, cho phép chúng phát triển độc lập. Nhược điểm là nó có thể gây ra các vấn đề phức tạp hơn và tiềm ẩn về hiệu suất nếu dịch vụ máy chủ mở không được thiết kế tốt.

Tóm lại, mối quan hệ dịch vụ máy chủ mở cho phép một ngữ cảnh giới hạn cung cấp dịch vụ có thể được sử dụng bởi một ngữ cảnh giới hạn khác mà không bị liên kết chặt chẽ.

Khi nào nó hoạt động?

Mối quan hệ dịch vụ máy chủ mở hoạt động hiệu quả trong các tình huống sau:

Khi một ngữ cảnh giới hạn cần sử dụng chức năng được cung cấp bởi một ngữ cảnh giới hạn khác.

Khi có nhu cầu giới thiệu khả năng hoặc dịch vụ của miền cho các hệ thống hoặc người dùng bên ngoài.

Khi có yêu cầu về cách tích hợp tiêu chuẩn với các hệ thống hoặc dịch vụ khác.

Trong những tình huống này, mối quan hệ dịch vụ máy chủ mở có thể cung cấp một cách tiêu chuẩn hóa để truy cập chức năng của một bối cảnh hoặc dịch vụ giới hạn khác. Điều này có thể giúp giảm sự ghép nối giữa các hệ thống, trong khi vẫn cho phép chúng tương tác một cách có ý nghĩa.

Khi nào nó không hoạt động?

Mối quan hệ dịch vụ máy chủ mở giữa các bối cảnh giới hạn có thể không hoạt động hiệu quả trong các trường hợp sau:

Các mối lo ngại về bảo mật: Vì mối quan hệ này liên quan đến việc phơi bày các hoạt động nội bộ của một bối cảnh giới hạn này sang một bối cảnh khác, nên các mối lo ngại về bảo mật như vi phạm dữ liệu và truy cập trái phép có thể phát sinh.

Khớp nối chặt chẽ: Mối quan hệ dịch vụ máy chủ mở có thể dẫn đến khớp nối chặt chẽ giữa hai bối cảnh giới hạn, điều này có thể khiến việc thay đổi một bối cảnh giới hạn trở nên khó khăn mà không ảnh hưởng đến bối cảnh kia.

Chi phí liên lạc: Mối quan hệ này yêu cầu liên lạc giữa hai bối cảnh giới hạn, điều này có thể dẫn đến chi phí bổ sung và độ trễ.

Thách thức về bảo trì: Mối quan hệ dịch vụ máy chủ mở yêu cầu duy trì khả năng tương thích ngược cho dịch vụ được cung cấp, điều này có thể trở thành thách thức khi dịch vụ phát triển theo thời gian.

% %! Open - Host Service : https:// thiết kế hướng miền - practitioners.com/open - host - dịch vụ

% %! Open - Host Service : https:// thiết kế hướng miền - practitioners.com/open - host - dịch vụ

% %! Open - Host Service : https:// thiết kế hướng miền - practitioners.com/open - host - dịch vụ

% %! Open - Host Service : https:// thiết kế hướng miền - practitioners.com/open - host - dịch vụ

% %! Open - Host Service : https:// thiết kế hướng miền - practitioners.com/open - host - dịch vụ

