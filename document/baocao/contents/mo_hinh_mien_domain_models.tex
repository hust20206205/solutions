

% %!<! - - Business Model Canvas : https:// thiết kế hướng miền - practitioners.com/business - value - canvas - - >

% %!<! - - có thể nêu thêm thôi - - >

Để tạo một phần mềm tốt, chúng ta cần phải hiểu rõ về phần mềm đó. Trong thiết kế hướng miền để có thể hiểu miền nhanh, chúng ta cần tạo ra các mô hình miền.

Mô hình miền là kiến thức có tổ chức và có cấu trúc về miền phù hợp để giải quyết vấn đề kinh doanh.

Mô hình miền không phải là kiến thức của chuyên gia ngành, mà là sự trừu tượng hóa của cả nhóm.

Trong quá trình phát triển, nhóm trao đổi và thảo luận về mô hình của nhóm.

Mô hình miền giúp nhóm hiểu công việc và đồng thuận khi làm việc.

Ví dụ: Trong đồ án này, mô hình miền của em bao gồm yêu câu nghiệp vụ và các sơ đồ: UML Use Case Diagrams, UML Activity Diagrams, UML Sequence Diagrams, UML Class Diagrams 

% %!<! - - Domain Model: https:// thiết kế hướng miền - practitioners.com/home/glossary/domain - model - - >
 

Trong Thiết kế hướng miền (thiết kế hướng miền), mô hình miền là sự thể hiện của miền có vấn đề dưới dạng mô hình phần mềm. Nó là một tập hợp các đối tượng miền và mối quan hệ giữa chúng, đồng thời được sử dụng để mô hình hóa logic nghiệp vụ và hành vi của miền.

Mô hình miền là trọng tâm của thiết kế hướng miền và đóng vai trò là công cụ giao tiếp giữa các chuyên gia ngành và nhà phát triển phần mềm. Nó cung cấp sự hiểu biết chung về miền và các khái niệm của nó, đồng thời giúp đảm bảo rằng hệ thống phần mềm phản ánh chính xác các yêu cầu và ràng buộc kinh doanh.

Mô hình miền được xây dựng bằng cách sử dụng kết hợp các kỹ thuật hướng đối tượng, chẳng hạn như các lớp và đối tượng, cũng như các kỹ thuật dành riêng cho miền, chẳng hạn như các thực thể, đối tượng giá trị và dịch vụ miền. Mô hình này được cải tiến nhiều lần khi dự án tiến triển và phát triển để phản ánh chính xác nhu cầu thay đổi của doanh nghiệp.

Mục tiêu cuối cùng của mô hình miền là cung cấp sự trình bày rõ ràng, ngắn gọn và chính xác về miền vấn đề và làm cơ sở để triển khai hệ thống phần mềm giải quyết các vấn đề kinh doanh.
 

Nhiều mô hình miền

Có thể có nhiều mô hình miền cùng tồn tại trong một tổ chức hoặc hệ thống. Mỗi mô hình miền tập trung vào một lĩnh vực cụ thể của doanh nghiệp và thể hiện các khái niệm, quy tắc và mối quan hệ của nó theo một cách riêng biệt. Các mô hình miền khác nhau có thể tương tác với nhau nhưng chúng duy trì các ranh giới tự chủ và nhất quán của riêng mình. Điều này cho phép mô hình hóa các hệ thống phức tạp với nhiều lĩnh vực kinh doanh, mỗi lĩnh vực có những yêu cầu và quan điểm riêng. 