\documentclass{article} % loại tài liệu
\usepackage[utf8]{vietnam} % tiếng Việt
\usepackage[left = 3.5cm,right = 2cm,top = 3.5cm,bottom = 3cm]{geometry} % căn lề
\usepackage{tikz} % vẽ hình
\usetikzlibrary{calc} % tính toán vẽ hình
\usepackage{graphicx} % hình ảnh
\usepackage{pdfpages} % thêm file pdf bao_cao_tien_do





% Sử dụng gói color để thay đổi màu
\usepackage{color} %* mau_sac
% Sử dụng gói xcolor để mở rộng tính năng thay đổi màu
\usepackage{xcolor} %* mau_sac
% \definecolor{HustRed}{RGB}{206,22,40}
% \definecolor{HustYellow}{RGB}{242,193,8}
\definecolor{BackgroundCode}{RGB}{242,242,235}
% Sử dụng gói minted để đính kèm và định dạng mã nguồn
\usepackage{minted} %! code_python
\setminted{
fontsize=\footnotesize, % Cỡ chữ cho mã nguồn
frame=lines, % Hiển thị đường viền xung quanh mã nguồn
framesep=2mm, % Khoảng cách giữa đường viền và nội dung mã nguồn
bgcolor=BackgroundCode, % Màu nền cho mã nguồn
linenos, % Hiển thị số dòng ở bên trái
autogobble, % Tự động loại bỏ khoảng trắng ở đầu mỗi dòng
}


\begin{document} % bắt đầu
%%%%%%%%%%%%%%%%%%%%%%%%%%%%%%%%%%%%%
% \begin{titlepage}

\begin{tikzpicture}[remember picture, overlay]\draw [line width = 3pt]($ (current page.north west) + (3.0cm, - 2.5cm)$)rectangle($ (current page.south east) + (- 2.5cm, 2.5cm)$);\draw [line width = 0.5pt]($ (current page.north west) + (3.1cm, - 2.6cm)$)rectangle($ (current page.south east) + (- 2.6cm, 2.6cm)$);\end{tikzpicture}

\begin{center}

\vspace{- 0.4cm}

\textbf{ĐẠI HỌC BÁCH KHOA HÀ NỘI} \\

\textbf{VIỆN TOÁN ỨNG DỤNG VÀ TIN HỌC} \\

\textbf{******}

\vspace{0.8cm}

\begin{figure}[H]

\centering

\includegraphics[scale = .5]{pictures/hust/logoBK.png}

\end{figure}

\vspace{0.7cm}

\textbf{\fontsize{16pt}{30pt}\selectfont {BÁO CÁO ĐỒ ÁN II}} \\

\textbf{\fontsize{10pt}{24pt}\selectfont {CHUYÊN NGÀNH: TOÁN TIN}}

\vspace{1cm}

\textbf{\fontsize{16pt}{30pt}\selectfont {ĐỀ TÀI:}} \\

% \textbf{\fontsize{19pt}{24pt}\selectfont {Xây dựng kiến trúc vi dịch vụ cho \\ bài toán hóa đơn điện tử}} \\

\textbf{\fontsize{20pt}{24pt}\selectfont {Sử dụng thiết kế hướng miền \\ xây dựng kiến trúc vi dịch vụ cho \\ bài toán hóa đơn điện tử}} \\

\end{center}

\vspace{0.7cm}

\hspace{2.6cm}\begin{minipage}{0.8\textwidth}
\textbf{\fontsize{10pt}{24pt}\selectfont {Giảng viên hướng dẫn: TS. Vũ Thành Nam}}
\end{minipage}

\vspace{0.7cm}

\hspace{3cm}\begin{minipage}{0.7\textwidth}

\begin{tabular}{l l l}

\textbf{\fontsize{10pt}{24pt}\selectfont {   Sinh viên thực hiện}} & \textbf{\fontsize{10pt}{24pt}\selectfont {   Vũ Văn Nghĩa }}      \\
\textbf{\fontsize{10pt}{24pt}\selectfont { Mã số sinh viên}}       & \textbf{\fontsize{10pt}{24pt}\selectfont {   20206205 }}          \\
\textbf{\fontsize{10pt}{24pt}\selectfont { Lớp}}                   & \textbf{\fontsize{10pt}{24pt}\selectfont {   Toán Tin 02 - K65 }} \\
\end{tabular}

\end{minipage}

\vspace{0.5cm}

\begin{center}

\textbf{Hà Nội, \the\month~/~\the\year}

\end{center}

\end{titlepage}


% \input{contents/trang_trang}

% \newpage

\begin{center}

{\bfseries NHẬN XÉT CỦA GIẢNG VIÊN HƯỚNG DẪN}

\end{center}

\begin{enumerate}

\item Mục đích và nội dung của đồ án:

\vspace{20ex} % Thêm khoảng cách dọc

\item 	Kết quả đạt được:

\vspace{20ex} % Thêm khoảng cách dọc

\item 	Ý thức làm việc của sinh viên:

\vspace{20ex} % Thêm khoảng cách dọc

\end{enumerate}

\hspace{0.4\textwidth}\begin{minipage}{0.5\textwidth}

\noindent\begin{center}

\textit{Hà Nội, \today} \\

\textbf{Giảng viên hướng dẫn} \\

\textit{(Ký và ghi rõ họ tên)}

\vspace{2cm}

\textbf{TS. Vũ Thành Nam}

\end{center}

\end{minipage}

\pagestyle{empty}

\newpage


% \includepdf[pages = -]{contents/bao_cao_tien_do_1.pdf}
% \includepdf[pages = -]{contents/bao_cao_tien_do_2.pdf}

% \newpage

\renewcommand*\contentsname{\centering MỤC LỤC}

\tableofcontents

\newpage


% \newpage

\section*{\centering LỜI CẢM ƠN}

\addcontentsline{toc}{section}{LỜI CẢM ƠN}

\lipsum[1 - 3] % Tạo ba đoạn văn bản giả mạo

\newpage


% \newpage

\section*{\centering LỜI MỞ ĐẦU}

\addcontentsline{toc}{section}{LỜI MỞ ĐẦU}
\lipsum[1-3] % Tạo ba đoạn văn bản giả mạo

\newpage
% \newpage

\section*{\centering TÓM TẮT NỘI DUNG ĐỒ ÁN}

\addcontentsline{toc}{section}{TÓM TẮT NỘI DUNG ĐỒ ÁN}
\lipsum[1-3] % Tạo ba đoạn văn bản giả mạo

\newpage
% \newpage
\section*{\centering ĐÁNH GIÁ VÀ THẢO LUẬN}
\addcontentsline{toc}{section}{ĐÁNH GIÁ VÀ THẢO LUẬN}
\newpage
 
% \newpage

\section*{\centering DANH SÁCH BẢNG}

\addcontentsline{toc}{section}{DANH SÁCH BẢNG}

\makeatletter

\renewcommand\listoftables{

\@starttoc{lot}

}

\makeatother

\listoftables

\newpage
% \newpage

\section*{\centering DANH SÁCH HÌNH ẢNH}

\addcontentsline{toc}{section}{DANH SÁCH HÌNH ẢNH}

\makeatletter

\renewcommand\listoffigures{

\@starttoc{lof}

}

\makeatother

\listoffigures

\newpage
% \newpage

\section*{\centering DANH SÁCH MÃ NGUỒN}

\addcontentsline{toc}{section}{DANH SÁCH MÃ NGUỒN}

\makeatletter

\renewcommand\listoflistings{

\@starttoc{lol}

}

\makeatother

\listoflistings

\newpage




% \subsubsection{xxxxxxx}
% 



Miền phụ chung cung cấp các giải pháp có sẵn mà doanh nghiệp có thể mua.

Doanh nghiệp không thể đạt được bất kỳ lợi thế cạnh tranh nào bằng cách thực hiện những điều khác biệt trong miền phụ chung.

% $????? VD: Các miền phụ chung như các hoạt động quản lý nhân sự và quản lý cơ sở vật chất không tạo thêm bất kỳ giá trị khác biệt nào cho doanh nghiệp. - - >


% % %!<! - - Generic Subdomain : https:// thiết kế hướng miền - practitioners.com/generic - subdomain - - >

% % %!<! - - Generic Subdomain : https:// thiết kế hướng miền - practitioners.com/generic - subdomain - - >

% Trang chủTrang chủBảng chú giảiLãnh địa Miền phụ chung

% Miền phụ chung

% Trong Thiết kế hướng miền (thiết kế hướng miền), miền phụ chung là loại miền phụ không có bất kỳ đặc điểm cụ thể hoặc duy nhất nào so với các miền khác trong cùng lĩnh vực. Đó là một miền phụ có thể được tìm thấy trên nhiều ngành, thay vì dành riêng cho một ngành hoặc miền.

% Mặc dù các miền phụ chung có thể không phải là duy nhất hoặc dành riêng cho một miền nhưng chúng vẫn cần được xác định rõ ràng và hiểu rõ để triển khai hiệu quả trong hệ thống.

% Ví dụ

% Xác thực và ủy quyền: Miền phụ này xử lý việc quản lý danh tính người dùng và quyền truy cập vào tài nguyên trong hệ thống. Thông thường, cần có một giải pháp chung cho miền phụ này để có thể sử dụng lại trên nhiều hệ thống.

% Thông báo : Miền phụ này xử lý việc gửi thông báo cho người dùng, chẳng hạn như thông báo qua email hoặc SMS. Tương tự như xác thực và ủy quyền, việc có một giải pháp chung cho miền phụ này có thể được sử dụng lại trên nhiều hệ thống thường rất hữu ích.

% Thanh toán : Miền phụ này xử lý các khoản thanh toán, bao gồm thu thập thông tin thanh toán, tính phí thẻ tín dụng và xử lý tiền hoàn lại. Tương tự như các ví dụ trên, giải pháp thanh toán chung có thể được sử dụng lại trên nhiều hệ thống.

% Định vị địa lý : Miền phụ này xử lý việc ánh xạ các vị trí thực tế tới các biểu diễn kỹ thuật số. Một giải pháp định vị địa lý chung có thể được sử dụng trong nhiều hệ thống, chẳng hạn như ánh xạ địa chỉ tới tọa độ GPS hoặc tính toán khoảng cách giữa các vị trí.

% % %!<! - - Generic Subdomain : https:// thiết kế hướng miền - practitioners.com/generic - subdomain - - >

% % %!<! - - Generic Subdomain : https:// thiết kế hướng miền - practitioners.com/generic - subdomain - - >

% % %!<! - - Generic Subdomain : https:// thiết kế hướng miền - practitioners.com/generic - subdomain - - >


 
\begin{listing}[h]  
\centering
\inputminted{javascript}{code/hello.js} 
\caption{MoTaChoDoanMaJavascript} 
\label{code:nghia3} 
\end{listing} 




\begin{figure}[h]
\centering
\includegraphics[height = 3cm]{pictures/logoBK.png}
\caption{ViDuHinhAnhTheoChieuDoc} %!nghia
\label{pictures:nghia2} %!nghia
\end{figure}

\begin{table}[ht]
\centering
\caption{Một ví dụ về bảng}
\begin{tabular}{|c|c|c|}
\hline
\textbf{STT} & \textbf{Tên} & \textbf{Điểm} \\
\hline
1 & Alice & 90 \\
2 & Bob & 85 \\
3 & Charlie & 92 \\
\hline
\end{tabular}
\label{table:example}
\end{table}

\end{document} % kết thúc


% \section{Danh sách các cụm từ viết tắt}
% \input{contents/danh_sach_cac_cum_tu_viet_tat}

% \section{Danh sách các thuật ngữ}
% % STT; Tiếng Anh; Tiếng Việt
% @sau

% kiến trúc nguyên khối, kiến trúc nguyên khối
% kiến trúc nguyên khối, kiến trúc nguyên khối
% kiến trúc vi dịch, kiến trúc vi dịch
% kiến trúc vi dịch, kiến trúc vi dịch
% kiến trúc vi dịch, kiến trúc vi dịch
% kiến trúc vi dịch, kiến trúc vi dịch
% thiết kế hướng miền, thiết kế hướng miền
% thiết kế hướng miền, thiết kế hướng miền

1 thiết kế hướng miền
Thiết kế hướng lĩnh vực
2 Domain (không dịch)
3 Abstraction Trừu tượng
4 chuyên gia ngành

% \subsubsection{xxxxxxx}
% 



Miền phụ chung cung cấp các giải pháp có sẵn mà doanh nghiệp có thể mua.

Doanh nghiệp không thể đạt được bất kỳ lợi thế cạnh tranh nào bằng cách thực hiện những điều khác biệt trong miền phụ chung.

% $????? VD: Các miền phụ chung như các hoạt động quản lý nhân sự và quản lý cơ sở vật chất không tạo thêm bất kỳ giá trị khác biệt nào cho doanh nghiệp. - - >


% % %!<! - - Generic Subdomain : https:// thiết kế hướng miền - practitioners.com/generic - subdomain - - >

% % %!<! - - Generic Subdomain : https:// thiết kế hướng miền - practitioners.com/generic - subdomain - - >

% Trang chủTrang chủBảng chú giảiLãnh địa Miền phụ chung

% Miền phụ chung

% Trong Thiết kế hướng miền (thiết kế hướng miền), miền phụ chung là loại miền phụ không có bất kỳ đặc điểm cụ thể hoặc duy nhất nào so với các miền khác trong cùng lĩnh vực. Đó là một miền phụ có thể được tìm thấy trên nhiều ngành, thay vì dành riêng cho một ngành hoặc miền.

% Mặc dù các miền phụ chung có thể không phải là duy nhất hoặc dành riêng cho một miền nhưng chúng vẫn cần được xác định rõ ràng và hiểu rõ để triển khai hiệu quả trong hệ thống.

% Ví dụ

% Xác thực và ủy quyền: Miền phụ này xử lý việc quản lý danh tính người dùng và quyền truy cập vào tài nguyên trong hệ thống. Thông thường, cần có một giải pháp chung cho miền phụ này để có thể sử dụng lại trên nhiều hệ thống.

% Thông báo : Miền phụ này xử lý việc gửi thông báo cho người dùng, chẳng hạn như thông báo qua email hoặc SMS. Tương tự như xác thực và ủy quyền, việc có một giải pháp chung cho miền phụ này có thể được sử dụng lại trên nhiều hệ thống thường rất hữu ích.

% Thanh toán : Miền phụ này xử lý các khoản thanh toán, bao gồm thu thập thông tin thanh toán, tính phí thẻ tín dụng và xử lý tiền hoàn lại. Tương tự như các ví dụ trên, giải pháp thanh toán chung có thể được sử dụng lại trên nhiều hệ thống.

% Định vị địa lý : Miền phụ này xử lý việc ánh xạ các vị trí thực tế tới các biểu diễn kỹ thuật số. Một giải pháp định vị địa lý chung có thể được sử dụng trong nhiều hệ thống, chẳng hạn như ánh xạ địa chỉ tới tọa độ GPS hoặc tính toán khoảng cách giữa các vị trí.

% % %!<! - - Generic Subdomain : https:// thiết kế hướng miền - practitioners.com/generic - subdomain - - >

% % %!<! - - Generic Subdomain : https:// thiết kế hướng miền - practitioners.com/generic - subdomain - - >

% % %!<! - - Generic Subdomain : https:// thiết kế hướng miền - practitioners.com/generic - subdomain - - >

% \subsubsection{xxxxxxx}
% 



Miền phụ chung cung cấp các giải pháp có sẵn mà doanh nghiệp có thể mua.

Doanh nghiệp không thể đạt được bất kỳ lợi thế cạnh tranh nào bằng cách thực hiện những điều khác biệt trong miền phụ chung.

% $????? VD: Các miền phụ chung như các hoạt động quản lý nhân sự và quản lý cơ sở vật chất không tạo thêm bất kỳ giá trị khác biệt nào cho doanh nghiệp. - - >


% % %!<! - - Generic Subdomain : https:// thiết kế hướng miền - practitioners.com/generic - subdomain - - >

% % %!<! - - Generic Subdomain : https:// thiết kế hướng miền - practitioners.com/generic - subdomain - - >

% Trang chủTrang chủBảng chú giảiLãnh địa Miền phụ chung

% Miền phụ chung

% Trong Thiết kế hướng miền (thiết kế hướng miền), miền phụ chung là loại miền phụ không có bất kỳ đặc điểm cụ thể hoặc duy nhất nào so với các miền khác trong cùng lĩnh vực. Đó là một miền phụ có thể được tìm thấy trên nhiều ngành, thay vì dành riêng cho một ngành hoặc miền.

% Mặc dù các miền phụ chung có thể không phải là duy nhất hoặc dành riêng cho một miền nhưng chúng vẫn cần được xác định rõ ràng và hiểu rõ để triển khai hiệu quả trong hệ thống.

% Ví dụ

% Xác thực và ủy quyền: Miền phụ này xử lý việc quản lý danh tính người dùng và quyền truy cập vào tài nguyên trong hệ thống. Thông thường, cần có một giải pháp chung cho miền phụ này để có thể sử dụng lại trên nhiều hệ thống.

% Thông báo : Miền phụ này xử lý việc gửi thông báo cho người dùng, chẳng hạn như thông báo qua email hoặc SMS. Tương tự như xác thực và ủy quyền, việc có một giải pháp chung cho miền phụ này có thể được sử dụng lại trên nhiều hệ thống thường rất hữu ích.

% Thanh toán : Miền phụ này xử lý các khoản thanh toán, bao gồm thu thập thông tin thanh toán, tính phí thẻ tín dụng và xử lý tiền hoàn lại. Tương tự như các ví dụ trên, giải pháp thanh toán chung có thể được sử dụng lại trên nhiều hệ thống.

% Định vị địa lý : Miền phụ này xử lý việc ánh xạ các vị trí thực tế tới các biểu diễn kỹ thuật số. Một giải pháp định vị địa lý chung có thể được sử dụng trong nhiều hệ thống, chẳng hạn như ánh xạ địa chỉ tới tọa độ GPS hoặc tính toán khoảng cách giữa các vị trí.

% % %!<! - - Generic Subdomain : https:// thiết kế hướng miền - practitioners.com/generic - subdomain - - >

% % %!<! - - Generic Subdomain : https:// thiết kế hướng miền - practitioners.com/generic - subdomain - - >

% % %!<! - - Generic Subdomain : https:// thiết kế hướng miền - practitioners.com/generic - subdomain - - >

%%%%%%%%%%%
% \section{Giới thiệu}
% Trong thời đại ngày nay, nhu cầu phát triển ứng dụng và hệ thống ngày càng tăng, đặt ra thách thức đối với kiến trúc phần mềm. Kiến trúc nguyên khối đã phục vụ hiệu quả trong quá khứ, nhưng kiến trúc này bắt đầu gặp khó khăn đối mặt với sự phức tạp, khả năng mở rộng và khả năng đáp ứng linh hoạt với thay đổi nhanh chóng trong yêu cầu kinh doanh.

Kiến trúc vi dịch vụ là giải pháp cho những thách thức này. Kiến trúc vi dịch vụ chia dự án thành những dịch vụ nhỏ độc lập, mỗi dịch vụ chịu trách nhiệm về một chức năng cụ thể. Từ đó, giảm sự phức tạp của dự án tăng tính linh hoạt và dễ dàng quản lý.

Việc vận dụng kết hợp giữa kiến trúc vi dịch vụ và thiết kế hướng miền là một cách tiếp cận toàn diện, giúp xác định và tổ chức các dịch vụ dựa trên việc hiểu rõ về lĩnh vực kinh doanh. Thiết kế hướng miền giúp xây dựng mô hình dựa trên yêu cầu nghiệp vụ thực tế, giúp dự án phản ánh đúng các quy trình kinh doanh.

% \subsection{Giới thiệu về bài toán hóa đơn điện tử}
% \input{contents/gioi_thieu_ve_bai_toan_hoa_don_dien_tu}

% \subsubsection{Hóa đơn}
% Theo quy định tại khoản 1 Điều 3 Nghị định 123/2020/NĐ - CP:

Hóa đơn là chứng từ kế toán do tổ chức, cá nhân bán hàng hóa, cung cấp dịch vụ lập, ghi nhận thông tin bán hàng hóa, cung cấp dịch vụ. Hóa đơn được thể hiện theo hình thức hóa đơn điện tử hoặc hóa đơn do cơ quan thuế đặt in.

% \subsubsection{Hóa đơn điện tử}
% \input{contents/hoa_don_dien_tu}

% \subsubsection{Bắt buộc sử dụng hóa đơn điện tử từ 01/07/2022}
% Nghị định này có hiệu lực thi hành kể từ ngày 01 tháng 7 năm 2022, khuyến khích cơ quan, tổ chức, cá nhân đáp ứng điều kiện về hạ tầng công nghệ thông tin áp dụng quy định về hóa đơn, chứng từ điện tử của Nghị định này trước ngày 01 tháng 7 năm 2022.

= >

Theo quy định của Chính phủ và Bộ Tài Chính, tất cả các doanh nghiệp, tổ chức và hộ kinh doanh đều bắt buộc phải chuyển từ sử dụng hóa đơn giấy sang hóa đơn điện tử kể từ tháng 07/2022.

Vì vậy, nhu cầu sử dụng và xử lý hóa đơn điện tử trở nên rất lớn. Do đó, em đã chọn chủ đề: \textbf{\textit{"Xây dựng kiến trúc vi dịch vụ cho hệ thống quản lý hóa đơn điện tử"}}



% \subsubsection{Bản thể hiện của hóa đơn điện tử}
% \input{contents/ban_the_hien_cua_hoa_don_dien_tu}

% \subsubsection{Lưu trữ hóa đơn điện tử}
% Thời gian?
% \input{contents/luu_tru_hoa_don_dien_tu}

% \subsubsection{Một số lợi ích của hóa đơn điện tử}
% Giúp tiết kiệm chi phí in ấn, lưu trữ và bảo quản.
Loại bỏ rủi ro cháy, hỏng hoặc mất và dễ dàng sao lưu.
Dễ dàng linh hoạt trong việc tra cứu, phát hành, quản lý và tạo báo cáo.
Tối ưu hóa quá trình kế toán (giảm sai sót và tiết kiệm thời gian) và giảm thủ tục giấy tờ.
Theo dõi tình hình tài chính của công ty (doanh thu, chi phí, lợi nhuận).
Tuân thủ các quy định về thuế và pháp luật.
Thể hiện tính minh bạch trong quá trình kinh doanh (bảo vệ quyền lợi của cả người mua và người bán).

% \subsection{Giới thiệu về kiến trúc vi dịch vụ}
% \subsubsection{Kiến trúc nguyên khối}
% Trước khi kiến trúc vi dịch vụ trở nên phổ biến, kiến trúc nguyên khối đã được áp dụng rộng rãi trong kiến trúc phần mềm truyền thống. Kiến trúc nguyên khối là kiến trúc phần mềm trong đó  tất cả các thành phần      của  dự án    được xây dựng thành một đơn vị triển khai    duy nhất. 

% Trong kiến trúc nguyên khối, bất kỳ thay đổi nào đối với một thành phần     đều yêu cầu toàn bộ ứng dụng phải được     kiểm thử   và triển khai lại. 




Điều này có thể dẫn đến chu kỳ phát triển và triển khai chậm hơn cũng như thiếu khả năng mở rộng vì ứng dụng có thể không đáp ứng được nhu cầu về chức năng hoặc lưu lượng truy cập ngày càng tăng. Tuy nhiên, kiến trúc nguyên khối thường thiết kế, phát triển và bảo trì đơn giản hơn so với các kiến trúc phân tán, khiến chúng trở thành lựa chọn phổ biến cho các ứng dụng quy mô nhỏ hơn hoặc các nhóm có nguồn lực hạn chế.


% 
Ví dụ: Mô hình MVC (Model - View - Controller) là một trong những dạng của kiến trúc nguyên khối.

Trong mô hình này, ứng dụng được chia thành ba thành phần chính:

Mô hình (Model): Đại diện cho dữ liệu và logic xử lý dữ liệu.

Giao diện (View): Đại diện cho giao diện người dùng.

Bộ điều khiển (Controller): Nhận yêu cầu người dùng thông qua View, sau đó tương tác với Model để làm việc với dữ liệu.

%  
 
 
 




% \subsubsection{Kiến trúc vi dịch vụ}
% Kiến trúc vi dịch vụ chia dự án thành các thành phần nhỏ hơn được gọi là các dịch vụ.

Các dịch vụ chịu trách nhiệm cho một chức năng cụ thể nhằm hiện thực hóa khả năng kinh doanh cụ thể.

Các dịch vụ độc lập về ngôn ngữ lập trình, CSDL, triển khai,...

Các dịch vụ tương tác với nhau qua hạ tầng mạng.

% \begin{figure}[h]

% \centering

% \includegraphics[height = 3cm]{pictures/ChuyenTu_KienTrucNguyenKhoi_Sang_KienTrucViDichVu.jpg}

% % \caption{ViDuHinhAnhTheoChieuDoc}

% \end{figure}

% \begin{figure}[h]

% \centering

% \includegraphics[height = 3cm]{pictures/AnhKhacNhau_KienTrucNguyenKhoi_KienTrucViDichVu.png}

% % \caption{ViDuHinhAnhTheoChieuDoc}

% \end{figure}

% Strangler Fig là chuyển mono sang dịch vụ

Kiến trúc kiến trúc vi dịch vụ (hay đơn giản là kiến trúc vi dịch vụ) là một cách xây dựng các ứng dụng phần mềm dưới dạng tập hợp các dịch vụ nhỏ, độc lập giao tiếp với nhau thông qua API. Mỗi dịch vụ tập trung vào một khả năng kinh doanh cụ thể và có thể được triển khai, mở rộng quy mô và duy trì độc lập với các dịch vụ khác trong hệ thống. Cách tiếp cận này nhấn mạnh tính mô - đun, tính linh hoạt và khả năng phục hồi, cho phép các nhóm làm việc đồng thời trên các phần khác nhau của hệ thống và cho phép phát hành nhanh hơn và thường xuyên hơn. Các vi dịch vụ thường dựa vào các giao thức truyền thông nhẹ, chẳng hạn như REST và thường được triển khai bằng các công nghệ chứa trong bộ chứa như Docker và Kubernetes.


% \subsubsection{Một số đặc điểm và ưu điểm của kiến trúc vi dịch vụ}
% Kiến trúc vi dịch vụ có nhiều ưu điểm, đặc biệt với các dự án có quy mô lớn và phức tạp.
% Mô - đun hóa : Kiến trúc nên bao gồm một tập hợp các dịch vụ được kết hợp lỏng lẻo có thể được triển khai, duy trì và mở rộng quy mô một cách độc lập.
Kiến trúc vi dịch vụ phân chia dự án thành các dịch vụ nhỏ.
% Nhìn chung, điều đó có nghĩa là I.T. các nhóm không cần phải đi sâu vào mọi khả năng kinh doanh. Họ có thể tập trung vào năng lực kinh doanh mà họ đang xây dựng trong vi dịch vụ của mình.
% Nhưng giả sử hoạt động kinh doanh cho vay và thế chấp đang trải qua một sự chuyển đổi nghiêm trọng nào đó, trong trường hợp đó, nhóm cho vay và thế chấp có thể quyết định phát hành vi dịch vụ của họ mỗi ngày.

%%%%%%%%%%%%%%%%%%%%%%%%%%%%%%%%%%%%%

% và có thể được triển khai, mở rộng quy mô và duy trì độc lập với các dịch vụ khác trong hệ thống.

% Các dịch vụ độc lập về ngôn ngữ lập trình, CSDL, triển khai,...

% Các dịch vụ tương tác với nhau qua hạ tầng mạng.

% Mô - đun hóa : Kiến trúc nên bao gồm một tập hợp các dịch vụ được kết hợp lỏng lẻo có thể được triển khai, duy trì và mở rộng quy mô một cách độc lập.

% Triển khai độc lập : Mỗi dịch vụ phải có khả năng triển khai độc lập để có thể thực hiện các thay đổi đối với một dịch vụ mà không ảnh hưởng đến các dịch vụ khác.

% Bối cảnh giới hạn : Kiến trúc phải được thiết kế để phù hợp với ranh giới của khả năng kinh doanh, sao cho mỗi dịch vụ chịu trách nhiệm về một tập hợp chức năng cụ thể, gắn kết.

% Tự chủ : Các dịch vụ phải có khả năng hoạt động tự chủ, ít phụ thuộc vào các dịch vụ khác.

% Khả năng phục hồi : Kiến trúc phải được thiết kế để chịu đựng lỗi và các dịch vụ phải có khả năng xử lý lỗi một cách duyên dáng.

% Khả năng mở rộng : Kiến trúc phải được thiết kế để hỗ trợ mở rộng quy mô các dịch vụ riêng lẻ cũng như toàn bộ hệ thống.

% Tính linh hoạt : Kiến trúc phải cho phép phát triển và triển khai nhanh chóng các dịch vụ mới cũng như khả năng thay đổi các dịch vụ hiện có một cách nhanh chóng và dễ dàng.

% Văn hóa DevOps : Kiến trúc phải được hỗ trợ bởi văn hóa nhấn mạnh sự cộng tác và giao tiếp giữa các nhóm phát triển và vận hành, cũng như tự động hóa các quy trình triển khai và thử nghiệm.


Kiến trúc vi dịch vụ phân chia dự án thành các dịch vụ nhỏ.

Giúp việc phát triển và quản lý dễ dàng hơn.

Dễ dàng mở rộng hệ thống.

Tận dụng sử dụng tài nguyên cho từng dịch vụ.

Tập trung yêu cầu nghiệp vụ trong dịch vụ dẫn đến tốc độ định giá doanh nghiệp nhanh hơn.

Vì các dịch vụ được phân chia là độc lập.

% Nhìn chung, điều đó có nghĩa là I.T. các nhóm không cần phải đi sâu vào mọi khả năng kinh doanh. Họ có thể tập trung vào năng lực kinh doanh mà họ đang xây dựng trong vi dịch vụ của mình.

% Nhưng giả sử hoạt động kinh doanh cho vay và thế chấp đang trải qua một sự chuyển đổi nghiêm trọng nào đó, trong trường hợp đó, nhóm cho vay và thế chấp có thể quyết định phát hành vi dịch vụ của họ mỗi ngày.

Các nhóm phát triển riêng dẫn tới tốc độ phát triển thay đổi nhanh.

Giảm thiểu ràng buộc và tăng tính linh hoạt của hệ thống.

Giảm chi phí và thời gian kiểm thử do ít ràng buộc.

Hệ thống có khả năng chịu lỗi cao tăng độ tin cậy.

Kiến trúc vi dịch vụ sử dụng đa ngôn ngữ và công nghệ khác nhau.

Tận dụng hiệu quả thế mạnh của từng ngôn ngữ, công nghệ phù hợp nhất cho yêu cầu nghiệp vụ cụ thể.

% Ví dụ: Mỗi dịch vụ sử dụng ngôn ngữ lập trình nhau khác như: NodeJS, Go, Python, Java, CSharp,...

% \begin{figure}[h]

% \centering

% \includegraphics[height = 3cm]{pictures/DaNgonNgu/_DaNgonNgu.png}

% % \caption{ViDuHinhAnhTheoChieuDoc}

% Thêm vào hình SQL riêng

% \end{figure}



% \subsubsection{Một số nhược điểm và thách thức của kiến trúc vi dịch vụ}
% Tuy nhiên, kiến trúc vi dịch vụ cũng có nhiều thách thức.

Chịu ảnh hưởng của đường truyền mạng.

Khả năng kiểm soát giao dịch (transaction).

Tính nhất quán và toàn vẹn của dữ liệu giữa các dịch vụ.

Giám sát giữa các dịch vụ.

Bảo mật giao tiếp giữa các dịch vụ.

Phát hiện lỗi và sửa lỗi khó khăn.

Ràng buộc về thứ tự sự kiện.

Đồng bộ đồng hồ thời gian.

Chi phí xây dựng, quản lí vận hành lớn.



% \subsubsection{Có thể thêm phần truyền thông trực tiếp, gián tiếp}

% \subsection{Giới thiệu về thiết kế hướng miền}
% % <! - - Lý do tiếp theo là một trong những lý do lớn nhất khiến tổ chức cần chuyển đổi nhu cầu và mong đợi của khách hàng liên tục thay đổi để duy trì và mở rộng cơ sở khách hàng của mình. - - >

% <! - - Các tổ chức cần điều chỉnh hoạt động kinh doanh của mình để đáp ứng nhu cầu và mong đợi của khách hàng. Các doanh nghiệp bỏ qua kỳ vọng của khách hàng có xu hướng thua đối thủ cạnh tranh. - - >

% <! - - Vì vậy, điều xảy ra với những doanh nghiệp không chuyển đổi, câu trả lời ngắn gọn cho câu hỏi này là những doanh nghiệp không chuyển đổi sẽ không thể tồn tại. - - >

% <! - - Một điểm quan trọng cần ghi nhớ là chuyển đổi không phải là sáng kiến hay nhiệm vụ chỉ diễn ra một lần. Các doanh nghiệp cần thay đổi liên tục và điều này đòi hỏi những thay đổi nhanh chóng đối với hệ thống và ứng dụng của họ. - - >

% <! - - Một thách thức chung mà các doanh nghiệp phải đối mặt trong hành trình chuyển đổi là cách xây dựng phần mềm cũ cản trở hoặc gây khó khăn cho các tổ chức trong việc chuyển đổi. - - >

Trong quá trình hoạt động, không phải mọi doanh nghiệp đều sẽ giữ nguyên mô hình kinh doanh được đưa ra ban đầu của mình. Khi quy mô thị trường thay đổi thì việc chuyển đổi mô hình kinh doanh là điều cần thiết. Chuyển đổi kinh doanh như một công cụ linh hoạt giúp các doanh nghiệp có thể phát triển và tồn tại giữa các đối thủ của mình.

Ví dụ:

Amazon từ hiệu sách trực tuyến thành thị trường cho nhà cung cấp khác như: Thương mại điện tử (E - commerce), Dịch vụ đám mây (Cloud Computing),...

<! - - !Thêm google - - >

% ![](pictures/KienTrucViDichVuAmazon.png)

% ![](pictures/KienTrucViDichVuAmazon.png)

\begin{figure}[h]

\centering

\includegraphics[height = 3cm]{pictures/KienTrucViDichVuAmazon.png}

% \caption{ViDuHinhAnhTheoChieuDoc}

\end{figure}

\begin{figure}[h]

\centering

\includegraphics[height = 3cm]{pictures/KienTrucViDichVuAmazon.png}

% \caption{ViDuHinhAnhTheoChieuDoc}

\end{figure}

<! - - Hình kiến trúc vi dịch vụ của Amazon - - >

<! - - !Thêm google - - >

Gần đây, Baemin dịch vụ giao đồ ăn đã rời khỏi thị trường Việt Nam cũng do sức ép từ các đối thủ khác khiến Baemin khó cạnh tranh trong mảng kinh doanh cốt lõi là giao đồ ăn. Các đối thủ này không chỉ cung cấp dịch vụ giao đồ ăn mà còn có đặt xe, giao hàng,...

![](pictures/Baemin.png)

<! - - Hình Baemin đã rời khỏi thị trường Việt Nam - - >

= > Hiện nay, các tổ chức doanh nghiệp có nhu cầu phát triển chuyển đổi kinh doanh để có thể tồn tại và phát triển khi thị trường thay đổi. Từ đó, đáp ứng nhu cầu của khách hàng, giúp mang đến ưu thế cạnh tranh so với các đối thủ. Do đó cần hệ thống chuyển đổi nhanh chóng đáp ứng nhu cầu của dự án và mong đợi của khách hàng.

= > Kiến trúc vi dịch vụ giải quyết những thách thức và hỗ trợ doanh nghiệp chuyển đổi dễ dàng.

Tuy nhiên, để xây dựng được kiến trúc vi dịch vụ tốt cần phải tạo ra các dịch vụ nhỏ phù hợp và duy trì tính độc lập. Nếu không thực hiện đúng các nhóm phụ thuộc lẫn nhau và mất đi lợi thế của kiến trúc vi dịch vụ.

Và từ đó, mẫu thiết kế hướng miền sử dụng để phân tích xây dựng kiến trúc vi dịch vụ.

Thiết kế hướng miền xác định và tổ chức các dịch vụ dựa trên việc hiểu rõ về lĩnh vực kinh doanh, giúp dự án phản ánh đúng các quy trình và quy tắc kinh doanh.


% thiết kế hướng miền có thể trợ giúp như thế nào

% Thiết kế hướng miền (thiết kế hướng miền) có thể hỗ trợ kiến trúc và thiết kế kiến trúc vi dịch vụ theo nhiều cách:

% Bối cảnh giới hạn : thiết kế hướng miền nhấn mạnh việc xác định các bối cảnh giới hạn, là các khu vực riêng biệt của miền có ranh giới được xác định rõ ràng. Điều này có thể giúp xác định ranh giới của kiến trúc vi dịch vụ và đảm bảo rằng mỗi kiến trúc vi dịch vụ đều có trách nhiệm rõ ràng và tập trung.

% ngôn ngữ chung : thiết kế hướng miền khuyến khích sử dụng một ngôn ngữ chung được chia sẻ bởi cả chuyên gia ngành và nhân viên kỹ thuật. Điều này có thể giúp đảm bảo rằng các vi dịch vụ giao tiếp hiệu quả với nhau và phù hợp với yêu cầu kinh doanh.

% Ánh xạ bối cảnh: thiết kế hướng miền cung cấp các kỹ thuật để ánh xạ các mối quan hệ giữa các bối cảnh giới hạn. Điều này có thể giúp đảm bảo rằng các vi dịch vụ được thiết kế với sự hiểu biết rõ ràng về sự phụ thuộc của chúng vào các dịch vụ khác.

% Tập hợp: thiết kế hướng miền định nghĩa tập hợp là các cụm đối tượng liên quan cần được coi là một đơn vị nhất quán duy nhất. Điều này có thể giúp đảm bảo rằng các vi dịch vụ được thiết kế với sự hiểu biết rõ ràng về các yêu cầu về tính nhất quán của dữ liệu.

% Tương quan với bối cảnh giới hạn

% Mặc dù người ta thường khuyên nên căn chỉnh ranh giới của kiến trúc vi dịch vụ với ranh giới của bối cảnh giới hạn, nhưng điều đó không phải lúc nào cũng cần thiết hoặc khả thi. Một vi dịch vụ có thể gói gọn nhiều ngữ cảnh giới hạn hoặc một ngữ cảnh giới hạn có thể được phân chia thành nhiều vi dịch vụ, tùy thuộc vào nhu cầu cụ thể của hệ thống và sự cân bằng liên quan. Cuối cùng, mục tiêu là tạo ra một hệ thống mô - đun và có thể bảo trì, đáp ứng các yêu cầu kinh doanh và mối quan hệ giữa bối cảnh giới hạn và các vi dịch vụ phải được thiết kế phù hợp.

% Tương quan với API thực thể

% Nhìn chung, việc thiết kế các vi dịch vụ chỉ dựa trên các hoạt động CRUD của thực thể không được khuyến khích vì nó có thể dẫn đến một kiến trúc liên kết chặt chẽ và không hiệu quả. Các vi dịch vụ phải được thiết kế dựa trên khả năng kinh doanh, có thể phù hợp hoặc không phù hợp với hoạt động CRUD của thực thể.

% thiết kế hướng miền có thể giúp xác định các khả năng kinh doanh và xác định bối cảnh giới hạn, sau đó có thể được sử dụng để hướng dẫn thiết kế các vi dịch vụ. Bằng cách tập trung vào khả năng kinh doanh thay vì hoạt động CRUD thực thể, các vi dịch vụ có thể được liên kết lỏng lẻo hơn, mang tính mô - đun hơn, dễ dàng duy trì và phát triển hơn theo thời gian.



%%%%%%%%%%%
% \section{Yêu cầu nghiệp vụ}
% Yêu cầu nghiệp vụ xác định nội dung, phạm vi, mục tiêu và chức năng mong muốn của hệ thống.

% thêm là yếu tố quan trọng trong  thiết kế hướng miền vì hướng đến nghiệp vụ kinh doanh
% Nguồn: TCT

% \subsection {Yêu cầu nghiệp vụ của bài toán phụ}
% \input{contents/yeu_cau_nghiep_vu_cua_bai_toan_phu}

% \subsubsection{Các chức năng tổng quan của bài toán phụ}
% \input{contents/cac_chuc_nang_tong_quan_cua_bai_toan_phu}
% \subsection{Yêu cầu nghiệp vụ chưa xong}
% \subsection{Yêu cầu nghiệp vụ chưa xong}
% \subsection{Yêu cầu nghiệp vụ chưa xong}
% \subsection{Yêu cầu nghiệp vụ chưa xong}
% \subsection{Yêu cầu nghiệp vụ chưa xong}
% chưa xong
% \input{contents/yeu_cau_nghiep_vu_chua_xong}

%%%%%%%%%%%
% \section{Chi tiết và áp dụng thiết kế hướng miền}
% https:// thiết kế hướng miền - practitioners.com/home/glossary
% https: //www.infoq.com/minibooks/domain - driven - design - quickly

% \subsection{Đôi nét về thiết kế hướng miền (DomainDrivenDesign)}
% Thiết kế hướng miền được Eric Evans giới thiệu trong cuốn sách "DomainDrivenDesign: Tackling Complexity in the Heart of Software".

Thiết kế hướng miền (DomainDrivenDesign) là một phương pháp thiết kế phần mềm tập trung vào việc hiểu rõ và mô hình hóa lĩnh vực kinh doanh của một tổ chức.

Thiết kế hướng miền nhấn mạnh việc sử dụng lĩnh vực nghiệp vụ kinh doanh để thảo luận và đề xuất giải pháp đáp ứng nhu cầu. Vì để tạo một phần mềm tốt, chúng ta cần phải hiểu rõ về chính phần mềm đó. Chính vì vậy để đạt được kết quả như mong đợi, chúng ta thường bắt đầu từ yêu cầu nghiệp vụ.

Trong nhiều ứng dụng thường có phần xử lý các công việc không liên quan đến vấn đề nghiệp vụ như truy cập file, hạ tầng mạng, CSDL,... trong đối tượng nghiệp vụ kinh doanh. Cách này giúp tốc độ hoàn thiện ứng dụng nhanh. Tuy nhiên, cách này làm cho thiết kế bị mất đi tính hướng đối tượng trong thực tế với mức độ doanh nghiệp lớn. Đây là lý do thiết kế hướng miền trở nên quan trọng.

Trong kiến trúc vi dịch vụ, thiết kế hướng miền giúp đảm bảo rằng mỗi dịch vụ được thiết kế phản ánh một phần cụ thể của lĩnh vực kinh doanh. Mỗi dịch vụ được quản lí bởi một nhóm nhỏ được hỗ trợ bởi các chuyên gia ngành.


% \subsection{Chuyên gia ngành}

An incisive expression of the primary concerns of the chuyên gia ngành s and their most relevant knowledge. A deep model sloughs off superficial aspects of the domain and naive interpretations.

%!<! - - [[Domain Expert]] A member of a software project whose field is the domain of the application, rather than software development. Not just any user of the software, the chuyên gia ngành has deep knowledge of the subject. - - >
 

Chuyên gia ngành

Trong Thiết kế hướng miền (thiết kế hướng miền), chuyên gia ngành là người có kiến thức và hiểu biết sâu sắc về miền kinh doanh hoặc lĩnh vực vấn đề đang được hệ thống phần mềm giải quyết. Chuyên gia ngành có kiến thức chuyên môn về các quy tắc, quy trình và khái niệm kinh doanh liên quan đến hệ thống đang được xây dựng và đóng vai trò là nguồn thông tin chính cho nhóm phát triển. Chuyên gia ngành giúp đảm bảo rằng mô hình miền thể hiện chính xác miền doanh nghiệp và hệ thống đang được xây dựng giải quyết đúng vấn đề cũng như giải quyết đúng nhu cầu kinh doanh.

Sự thể hiện sâu sắc về mối quan tâm hàng đầu của các ngành lớn và kiến thức phù hợp nhất của họ. Một mô hình sâu sắc sẽ loại bỏ các khía cạnh hời hợt của lĩnh vực này và những diễn giải ngây thơ.

%!<! - - [[Chuyên gia miền]] Thành viên của một dự án phần mềm có lĩnh vực là miền ứng dụng chứ không phải phát triển phần mềm. Không chỉ bất kỳ người sử dụng phần mềm nào, các chuyên ngành đều có kiến thức sâu rộng về chủ đề này. - - >

% \subsection{Miền (Domain)}
% Phần mềm được tạo ra để xử lý sự phức tạp trong cuộc sống hiện đại. Việc phát triển phần mềm liên kết chặt chẽ với một số khía cạnh cụ thể trong cuộc sống của chúng ta.

Miền (Domain) đề cập đến phạm vi kiến thức và vấn đề mà hệ thống hoặc dự án cụ thể đang xử lý.

Về góc độ kinh doanh: miền đại diện cho một lĩnh vực hoặc ngành mà doanh nghiệp hoạt động.
Về góc độ phần mềm: miền có thể coi là đại diện cho không gian vấn đề của phần mềm đó.

Phần mềm cần phản ánh đúng miền và hiện thực hóa chính xác miền.

%!<! - - $VD: Ở đồ án này, miền được xác định là bài toán giải pháp hóa đơn điện tử. - - >

% %!<! - - Domain : https:// thiết kế hướng miền - practitioners.com/domain - - >
% %!<! - - Domain : https:// thiết kế hướng miền - practitioners.com/domain - - >
% %!<! - - Domain : https:// thiết kế hướng miền - practitioners.com/domain - - >
% %!<! - - Domain : https:// thiết kế hướng miền - practitioners.com/domain - - >
% %!<! - - Domain : https:// thiết kế hướng miền - practitioners.com/domain - - >
% %!<! - - Domain : https:// thiết kế hướng miền - practitioners.com/domain - - >
% %!<! - - Domain : https:// thiết kế hướng miền - practitioners.com/domain - - >
% %!<! - - Domain : https:// thiết kế hướng miền - practitioners.com/domain - - >

%!<! - - [[Domain]] A sphere of knowledge, influence, or activity. - - >

Trang chủTrang chủBảng chú giảiLãnh địa
Lãnh địa
Trong thiết kế hướng miền (thiết kế hướng miền), miền là một phạm vi kiến thức, ảnh hưởng hoặc hoạt động đại diện cho một lĩnh vực chuyên môn hoặc mối quan tâm cụ thể. Miền là không gian vấn đề mà phần mềm đang được xây dựng để giải quyết và nó thể hiện vấn đề hoặc cơ hội kinh doanh mà phần mềm có nhiệm vụ giải quyết.

Miền có thể là một ngành cụ thể, chẳng hạn như chăm sóc sức khỏe, tài chính hoặc thương mại điện tử hoặc có thể là một lĩnh vực quan tâm cụ thể trong một ngành, chẳng hạn như quản lý hàng tồn kho, quản lý khách hàng hoặc hậu cần.

Trong thiết kế hướng miền, mục tiêu là tạo ra một mô hình miền thể hiện chính xác các khái niệm và mối quan hệ trong thế giới thực trong miền và sử dụng mô hình này làm cơ sở cho việc thiết kế và phát triển phần mềm. Mô hình miền này phải phản ánh miền kinh doanh, nó phải dễ hiểu đối với các chuyên gia ngành, nhà phát triển và các bên liên quan và nó phải là xương sống của hệ thống phần mềm.

Một miền có thể được xác định bởi các quy tắc kinh doanh, quy trình kinh doanh, các thực thể kinh doanh và các mối quan hệ của chúng, các yêu cầu kinh doanh và mục tiêu kinh doanh.

Miền là trái tim của thiết kế hướng miền, là nền tảng của phần mềm và là điểm khởi đầu của quá trình phát triển. Hiểu miền là rất quan trọng để có thể tạo ra một phần mềm phù hợp với nhu cầu kinh doanh, ít phức tạp hơn, dễ bảo trì hơn và dễ thích ứng hơn với thay đổi.

Xem thêm	 Miền phụ, miền vấn đề, miền giải pháp, bối cảnh giới hạn

% %!<! - - Problem Domain :https:// thiết kế hướng miền - practitioners.com/home/glossary/problem - domain - - >
% %!<! - - Problem Domain :https:// thiết kế hướng miền - practitioners.com/home/glossary/problem - domain - - >
% %!<! - - Problem Domain :https:// thiết kế hướng miền - practitioners.com/home/glossary/problem - domain - - >
% %!<! - - Problem Domain :https:// thiết kế hướng miền - practitioners.com/home/glossary/problem - domain - - >
% %!<! - - Problem Domain :https:// thiết kế hướng miền - practitioners.com/home/glossary/problem - domain - - >
% %!<! - - Problem Domain :https:// thiết kế hướng miền - practitioners.com/home/glossary/problem - domain - - >
% %!<! - - Problem Domain :https:// thiết kế hướng miền - practitioners.com/home/glossary/problem - domain - - >
% %!<! - - Problem Domain :https:// thiết kế hướng miền - practitioners.com/home/glossary/problem - domain - - >

Trang chủTrang chủBảng chú giảiMiền vấn đề
Miền vấn đề
Trong Thiết kế hướng miền (thiết kế hướng miền), miền vấn đề đề cập đến lĩnh vực kiến thức hoặc hoạt động kinh doanh cụ thể mà hệ thống phần mềm đang được phát triển để giải quyết. Đó là lĩnh vực chuyên môn mà phần mềm dự định hỗ trợ, chẳng hạn như tài chính, chăm sóc sức khỏe, thương mại điện tử, v.v. Miền vấn đề là bối cảnh mà phần mềm sẽ được sử dụng và các yêu cầu cụ thể mà phần mềm cần đáp ứng.

Miền vấn đề thường được xác định bởi các bên liên quan của dự án, chẳng hạn như người dùng cuối, chuyên gia ngành và nhà phân tích kinh doanh. Mục tiêu của thiết kế hướng miền là mô hình hóa miền vấn đề một cách chính xác nhất có thể bằng cách tạo ra một mô hình miền phản ánh chính xác các khái niệm và quy trình kinh doanh cơ bản. Điều này đạt được bằng cách xác định các khái niệm và khái niệm trừu tượng chính trong miền vấn đề và tạo ra sự hiểu biết chung về các khái niệm này thông qua việc sử dụng ngôn ngữ chung.

Tóm lại, miền vấn đề là lĩnh vực cụ thể của doanh nghiệp mà hệ thống phần mềm đang được phát triển để giải quyết và thiết kế hướng miền là một phương pháp giúp mô hình hóa miền vấn đề một cách chính xác và tạo ra sự hiểu biết chung về các khái niệm cơ bản thông qua việc sử dụng một ngôn ngữ chung.

% %!<! - - Solution Domain :https:// thiết kế hướng miền - practitioners.com/home/glossary/solution - domain - - >
% %!<! - - Solution Domain :https:// thiết kế hướng miền - practitioners.com/home/glossary/solution - domain - - >
% %!<! - - Solution Domain :https:// thiết kế hướng miền - practitioners.com/home/glossary/solution - domain - - >
% %!<! - - Solution Domain :https:// thiết kế hướng miền - practitioners.com/home/glossary/solution - domain - - >
% %!<! - - Solution Domain :https:// thiết kế hướng miền - practitioners.com/home/glossary/solution - domain - - >
% %!<! - - Solution Domain :https:// thiết kế hướng miền - practitioners.com/home/glossary/solution - domain - - >
% %!<! - - Solution Domain :https:// thiết kế hướng miền - practitioners.com/home/glossary/solution - domain - - >
% %!<! - - Solution Domain :https:// thiết kế hướng miền - practitioners.com/home/glossary/solution - domain - - >

Trang chủTrang chủBảng chú giảiMiền giải pháp
Miền giải pháp
Trong Thiết kế hướng miền (thiết kế hướng miền), miền giải pháp đề cập đến hệ thống phần mềm cụ thể đang được phát triển để giải quyết miền vấn đề. Miền giải pháp bao gồm thiết kế, kiến trúc và triển khai hệ thống phần mềm cũng như cách thức ánh xạ tới miền vấn đề.

Miền giải pháp bao gồm mã, thành phần và dịch vụ tạo nên hệ thống phần mềm cũng như các mẫu thiết kế và nguyên tắc kiến trúc được sử dụng để cấu trúc hệ thống. Nó cũng bao gồm các quyết định thiết kế cụ thể được đưa ra trong quá trình phát triển, chẳng hạn như lựa chọn ngôn ngữ và khung lập trình cũng như việc sử dụng các mẫu thiết kế và phong cách kiến trúc cụ thể.

Mục tiêu của thiết kế hướng miền là căn chỉnh miền giải pháp với miền vấn đề càng chặt chẽ càng tốt. Điều này đạt được bằng cách tạo ra một mô hình miền phản ánh chính xác các khái niệm và quy trình kinh doanh cơ bản trong miền có vấn đề và bằng cách sử dụng ngôn ngữ chung để tạo ra sự hiểu biết chung về các khái niệm này trong toàn nhóm phát triển.

Tóm lại, miền giải pháp là hệ thống phần mềm cụ thể đang được phát triển để giải quyết miền vấn đề và thiết kế hướng miền là một phương pháp giúp điều chỉnh miền giải pháp với miền vấn đề càng chặt chẽ càng tốt bằng cách tạo mô hình miền phản ánh chính xác các khái niệm và quy trình kinh doanh cơ bản trong lĩnh vực vấn đề và bằng cách sử dụng ngôn ngữ chung để tạo ra sự hiểu biết chung về các khái niệm này trong nhóm phát triển.



% \subsection{Chuyên gia ngành}
% An incisive expression of the primary concerns of the chuyên gia ngành s and their most relevant knowledge. A deep model sloughs off superficial aspects of the domain and naive interpretations.
%!<! - - [[Domain Expert]] A member of a software project whose field is the domain of the application, rather than software development. Not just any user of the software, the chuyên gia ngành has deep knowledge of the subject. - - >
% https:// thiết kế hướng miền - practitioners.com/home/glossary/domain - expert
% https:// thiết kế hướng miền - practitioners.com/home/glossary/domain - expert
% https:// thiết kế hướng miền - practitioners.com/home/glossary/domain - expert
% https:// thiết kế hướng miền - practitioners.com/home/glossary/domain - expert
% https:// thiết kế hướng miền - practitioners.com/home/glossary/domain - expert
% https:// thiết kế hướng miền - practitioners.com/home/glossary/domain - expert
% https:// thiết kế hướng miền - practitioners.com/home/glossary/domain - expert
% https:// thiết kế hướng miền - practitioners.com/home/glossary/domain - expert
% https:// thiết kế hướng miền - practitioners.com/home/glossary/domain - expert
% https:// thiết kế hướng miền - practitioners.com/home/glossary/domain - expert

Trang chủTrang chủBảng chú giải Chuyên gia ngành
Chuyên gia ngành
Trong Thiết kế hướng miền (thiết kế hướng miền), chuyên gia ngành là người có kiến thức và hiểu biết sâu sắc về miền kinh doanh hoặc lĩnh vực vấn đề đang được hệ thống phần mềm giải quyết. Chuyên gia ngành có kiến thức chuyên môn về các quy tắc, quy trình và khái niệm kinh doanh liên quan đến hệ thống đang được xây dựng và đóng vai trò là nguồn thông tin chính cho nhóm phát triển. Chuyên gia ngành giúp đảm bảo rằng mô hình miền thể hiện chính xác miền doanh nghiệp và hệ thống đang được xây dựng giải quyết đúng vấn đề cũng như giải quyết đúng nhu cầu kinh doanh.



% \subsection{Miền phụ (Sub - Domain)}
% \begin{itemize}

\item Miền được tạo thành từ nhiều miền phụ.

\item Ví dụ: Trong miền thương mại điện tử lớn. Có thể có một số miền phụ:

\begin{itemize}

\item Miền phụ quản lý hàng tồn kho: liên quan đến việc quản lý sản phẩm trong kho hàng.

\item Miền phụ quản lý khách hàng: liên quan đến việc quản lý tài khoản khách hàng.

\item Miền phụ vận chuyển: liên quan đến việc quản lý việc vận chuyển giao hàng.

\end{itemize}

\item Trong một miền phức tạp, không thể có một chuyên gia ngành có kiến thức về tất cả các miền phụ.

\end{itemize}

% \subsection{Mô hình miền (Domain Models)}
% % %!<! - - Business Model Canvas : https:// thiết kế hướng miền - practitioners.com/business - value - canvas - - >

% %!<! - - có thể nêu thêm thôi - - >

Để tạo một phần mềm tốt, chúng ta cần phải hiểu rõ về phần mềm đó. Trong thiết kế hướng miền để có thể hiểu miền nhanh, chúng ta cần tạo ra các mô hình miền.

Mô hình miền là kiến thức có tổ chức và có cấu trúc về miền phù hợp để giải quyết vấn đề kinh doanh.

Mô hình miền không phải là kiến thức của chuyên gia ngành, mà là sự trừu tượng hóa của cả nhóm.

Trong quá trình phát triển, nhóm trao đổi và thảo luận về mô hình của nhóm.

Mô hình miền giúp nhóm hiểu công việc và đồng thuận khi làm việc.

Ví dụ: Trong đồ án này, mô hình miền của em bao gồm yêu câu nghiệp vụ và các sơ đồ: UML Use Case Diagrams, UML Activity Diagrams, UML Sequence Diagrams, UML Class Diagrams

% %!<! - - Domain Model: https:// thiết kế hướng miền - practitioners.com/home/glossary/domain - model - - >

Trong Thiết kế hướng miền (thiết kế hướng miền), mô hình miền là sự thể hiện của miền có vấn đề dưới dạng mô hình phần mềm. Nó là một tập hợp các đối tượng miền và mối quan hệ giữa chúng, đồng thời được sử dụng để mô hình hóa logic nghiệp vụ và hành vi của miền.

Mô hình miền là trọng tâm của thiết kế hướng miền và đóng vai trò là công cụ giao tiếp giữa các chuyên gia ngành và nhà phát triển phần mềm. Nó cung cấp sự hiểu biết chung về miền và các khái niệm của nó, đồng thời giúp đảm bảo rằng hệ thống phần mềm phản ánh chính xác các yêu cầu và ràng buộc kinh doanh.

Mô hình miền được xây dựng bằng cách sử dụng kết hợp các kỹ thuật hướng đối tượng, chẳng hạn như các lớp và đối tượng, cũng như các kỹ thuật dành riêng cho miền, chẳng hạn như các thực thể, đối tượng giá trị và dịch vụ miền. Mô hình này được cải tiến nhiều lần khi dự án tiến triển và phát triển để phản ánh chính xác nhu cầu thay đổi của doanh nghiệp.

Mục tiêu cuối cùng của mô hình miền là cung cấp sự trình bày rõ ràng, ngắn gọn và chính xác về miền vấn đề và làm cơ sở để triển khai hệ thống phần mềm giải quyết các vấn đề kinh doanh.

Nhiều mô hình miền

Có thể có nhiều mô hình miền cùng tồn tại trong một tổ chức hoặc hệ thống. Mỗi mô hình miền tập trung vào một lĩnh vực cụ thể của doanh nghiệp và thể hiện các khái niệm, quy tắc và mối quan hệ của nó theo một cách riêng biệt. Các mô hình miền khác nhau có thể tương tác với nhau nhưng chúng duy trì các ranh giới tự chủ và nhất quán của riêng mình. Điều này cho phép mô hình hóa các hệ thống phức tạp với nhiều lĩnh vực kinh doanh, mỗi lĩnh vực có những yêu cầu và quan điểm riêng.

% \subsection{Các khuôn mẫu trong thiết kế hướng miền}
% Thiết kế hướng miền cung cấp 2 loại mẫu:

\begin{itemize}

\item \textbf{Các mẫu chiến lược (Strategic Patterns):} Thiết kế phân chia một miền lớn và phức tạp thành các phần nhỏ hơn với ranh giới được xác định rõ ràng. Giúp phân chia một miền lớn hợp lý.

\item \textbf{Các mẫu kỹ thuật (Tactical Patterns):} Hiện thực hóa các mô hình khái niệm trong một thành phần nhỏ thành các thiết kế hệ thống phần mềm. Giúp hệ thống phù hợp với kinh doanh.

\end{itemize}


\begin{figure}[H]
    \centering
    \includegraphics[width = 0.5\textwidth]{pictures/StrategicPatternsVATacticalPatterns/main.drawio.png}
    \caption{Khái quát tổng quan về Strategic Patterns và Tactical Patterns}
    \end{figure}



% \subsection{Các mẫu chiến lược (Strategic Patterns)}
% Các mẫu chiến lược phân tích nghiệp vụ kinh doanh sau đó đưa ra việc phân chia các thành phần và hiểu mối quan hệ của các thành phần đó.

Các mẫu chiến lược là giai đoạn xây dựng sự hiểu biết chung về miền giữa chuyên gia ngành và nhóm phân tích hệ thống .





 
\begin{itemize} 
    \item  Các mẫu chiến lược giúp  xác định các thành phần quan trọng của hệ thống.
    \item  Các mẫu chiến lược đảm bảo   kiến trúc phần mềm phản ánh đúng các yêu cầu kinh doanh.
    
\end{itemize}
 






$\Rightarrow$     Các mẫu chiến lược  xác định mục tiêu tạo ra  hệ thống có thể mở rộng,  phát triển linh hoạt theo nhu cầu kinh doanh. 

Các mẫu chiến lược đề cập đến thiết kế tổng thể của hệ thống bao gồm:

\begin{itemize}

\item Muc1

\item Muc2

\item các mục bên dưới \dots

\end{itemize}

%

\begin{figure}[H]

\centering

\includegraphics[width = 1\textwidth]{pictures/CacMoHinhChienLuoc/temp.png}

\caption{Sơ đồ về các thành phần trong mô hình chiến lược}

\end{figure}

%!<! - - $ Vẽ lại sau: - - >

%!<! - - $ Vẽ lại sau: - - >

%!<! - - $ Vẽ lại sau: - - >

%!<! - - $ Vẽ lại sau: - - >

%!<! - - $ Vẽ lại sau: - - >

%!<! - - $ Vẽ lại sau: - - >

%!<! - - $ Vẽ lại sau: - - >

%!<! - - $ Vẽ lại sau: - - >

%!<! - - $ Vẽ lại sau: - - >

%!<! - - $ Vẽ lại sau: - - >

%!<! - - $ Vẽ lại sau: - - >

%!<! - - Bối cảnh giới hạn (Bounded Context) - - >

%!<! - - [Giữ cho mô hình thống nhất] Tích hợp Liên tục (Continuous Integration) - - >

%!<! - - [Tính nhất quán trong trao đổi] Ngôn ngữ chung (Ubiquitous Language) - - >

%!<! - - [Tổng quan mối quan hệ] Bản đồ bối cảnh (Context Maps) - - >

%!<! - - Symmetric Relationship: Separate ways, Shared Kernel - - >

%!<! - - Asymmetric Relationship: Customer - Supplier, Conformist, Anti Corruption Layer - - >

%!<! - - - - >

%!<! - - One - to - Many Relationship: Open Host Service, Published Language - - >

%!<! - - dịch và cách ly đơn phương với - - >

%!<! - - [lớp] lớp (Context Maps) - - >

%

%!<! - - "Bản đồ bối cảnh dịch chuyển và cách ly một cách đơn phương để tạo thành cấu trúc lớp." - - >

%!<! - - Tách biệt - - >



% \subsubsection{Bối cảnh giới hạn (Bounded Context)}
% \begin{itemize}

\item Một miền cần chia đủ nhỏ để phù hợp với một nhóm cụ thể. Để đạt được điều này, chúng ta cần xác định rõ ranh giới giữa các ngữ cảnh.

\item Bối cảnh giới hạn (Bounded Context) giúp định rõ các ranh giới, chia miền thành các phần độc lập để giải quyết sự phức tạp trong mô hình doanh nghiệp.

\item Bối cảnh giới hạn thể hiện phạm vi kinh doanh của dịch vụ.

\item Bối cảnh giới hạn tạo ra các mô hình khác nhau cho các lĩnh vực khác nhau của miền.

\item Bối cảnh giới hạn tương ứng với một nhóm cụ thể trong tổ chức.

\end{itemize}

\begin{figure}[H]

\centering

\includegraphics[width = 0.8\textwidth]{pictures/BoiCanhGioiHan/main.png}

\caption{Ví dụ về bối cảnh giới hạn trong một ngân hàng}

\end{figure}

\subsubsection{Một vài hướng xác định bối cảnh giới hạn}

\begin{itemize}

\item Việc xác định bối cảnh giới hạn được điều chỉnh bởi sự gắn kết giữa các miền phụ.

\item Dựa vào sơ đồ cấu trúc tổ chức các phòng ban của doanh nghiệp.

\item Dựa vào modules của các ứng dụng kiến trúc nguyên khối (nếu việc phân chia tốt).

\item Dựa vào trách nhiệm và hoạt động của chuyên gia ngành.

\end{itemize}

\end{document}

%!<! - - Hướng dẫn 5/10 - - >

\subsubsection{CI/CD}
\input{contents/cicd}

\subsubsection{Ngôn ngữ chung (Ubiquitous Language)}
\begin{itemize}

\item Trong quá trình xây dựng mô hình miền, cần có trao đổi giữa người thiết kế hệ thống và chuyên gia ngành để hiểu đúng về miền. Tuy nhiên, nhóm kinh doanh sử dụng ngôn ngữ kinh doanh và nhóm công nghệ có xu hướng sử dụng các thuật ngữ kỹ thuật trong giao tiếp của họ. Người phát triển phần mềm tập trung vào lớp, phương thức, thuật toán, trong khi chuyên gia ngành thường sử dụng ngôn ngữ chuyên ngành của họ. Sự khác biệt về ngôn ngữ giữa các thành viên có thể dẫn đến những thách thức về giao tiếp.

\item Ngoài ra, trong các lĩnh vực kinh doanh khác nhau, một thuật ngữ có thể được sử dụng trong nhiều miền, cùng với ý nghĩa khác nhau gây ra sự nhầm lẫn và hiểu sai cho các người phát triển phần mềm cũng như các chuyên gia ngành.

$\Rightarrow$ Thiết kế hướng miền đề xuất sử dụng ngôn ngữ chung để giải quyết những thách thức ngôn ngữ.

\item Ngôn ngữ chung (Ubiquitous Language) là một ngôn ngữ được cấu trúc xung quanh mô hình miền và được tất cả các thành viên trong nhóm sử dụng cho mọi hoạt động của nhóm với phần mềm.

\item Ngôn ngữ chung được xác định bởi các từ vựng và có định nghĩa rõ ràng về ngữ cảnh sử dụng từ vựng.

\end{itemize}

\subsubsection{Một số đặc điểm của ngôn ngữ chung}

\begin{itemize}

\item Ngôn ngữ chung được sử dụng bởi cả chuyên gia ngành và chuyên gia công nghệ.

\item Có nhiều ngôn ngữ chung trong một tổ chức được mỗi nhóm tạo và quản lý một cách độc lập.

\item Việc tạo ra ngôn ngữ chung là một quá trình liên tục.

Ngôn ngữ chung phát triển theo thời gian thông qua sự cộng tác giữa doanh nghiệp và các chuyên gia công nghệ.

\item Các thành viên phải sử dụng ngôn ngữ chung cho công việc và trong toàn bộ hệ thống:

\begin{itemize}

\item Các chuyên gia ngành sử dụng.

\item Các chuyên gia công nghệ sử dụng.

\item Sử dụng trong cuộc thảo luận trao đổi.

\item Sử dụng trong các tài liệu phát triển.

\item Sử dụng trong sản phẩm phần mềm.

\item Sử dụng trong kiểm thử phần mềm.

\end{itemize}

\end{itemize}

\begin{figure}[H]

\centering

\includegraphics[width = 0.6\textwidth]{pictures/NgonNguChung/main.png}

\caption{Ngôn ngữ chung được sử dụng trong toàn bộ hệ thống}

\end{figure}

%!<! - - Hướng dẫn 5/7 - - >



\subsubsection{Bản đồ bối cảnh (Context Maps)}




Miền phụ chung cung cấp các giải pháp có sẵn mà doanh nghiệp có thể mua.

Doanh nghiệp không thể đạt được bất kỳ lợi thế cạnh tranh nào bằng cách thực hiện những điều khác biệt trong miền phụ chung.

% $????? VD: Các miền phụ chung như các hoạt động quản lý nhân sự và quản lý cơ sở vật chất không tạo thêm bất kỳ giá trị khác biệt nào cho doanh nghiệp. - - >


% % %!<! - - Generic Subdomain : https:// thiết kế hướng miền - practitioners.com/generic - subdomain - - >

% % %!<! - - Generic Subdomain : https:// thiết kế hướng miền - practitioners.com/generic - subdomain - - >

% Trang chủTrang chủBảng chú giảiLãnh địa Miền phụ chung

% Miền phụ chung

% Trong Thiết kế hướng miền (thiết kế hướng miền), miền phụ chung là loại miền phụ không có bất kỳ đặc điểm cụ thể hoặc duy nhất nào so với các miền khác trong cùng lĩnh vực. Đó là một miền phụ có thể được tìm thấy trên nhiều ngành, thay vì dành riêng cho một ngành hoặc miền.

% Mặc dù các miền phụ chung có thể không phải là duy nhất hoặc dành riêng cho một miền nhưng chúng vẫn cần được xác định rõ ràng và hiểu rõ để triển khai hiệu quả trong hệ thống.

% Ví dụ

% Xác thực và ủy quyền: Miền phụ này xử lý việc quản lý danh tính người dùng và quyền truy cập vào tài nguyên trong hệ thống. Thông thường, cần có một giải pháp chung cho miền phụ này để có thể sử dụng lại trên nhiều hệ thống.

% Thông báo : Miền phụ này xử lý việc gửi thông báo cho người dùng, chẳng hạn như thông báo qua email hoặc SMS. Tương tự như xác thực và ủy quyền, việc có một giải pháp chung cho miền phụ này có thể được sử dụng lại trên nhiều hệ thống thường rất hữu ích.

% Thanh toán : Miền phụ này xử lý các khoản thanh toán, bao gồm thu thập thông tin thanh toán, tính phí thẻ tín dụng và xử lý tiền hoàn lại. Tương tự như các ví dụ trên, giải pháp thanh toán chung có thể được sử dụng lại trên nhiều hệ thống.

% Định vị địa lý : Miền phụ này xử lý việc ánh xạ các vị trí thực tế tới các biểu diễn kỹ thuật số. Một giải pháp định vị địa lý chung có thể được sử dụng trong nhiều hệ thống, chẳng hạn như ánh xạ địa chỉ tới tọa độ GPS hoặc tính toán khoảng cách giữa các vị trí.

% % %!<! - - Generic Subdomain : https:// thiết kế hướng miền - practitioners.com/generic - subdomain - - >

% % %!<! - - Generic Subdomain : https:// thiết kế hướng miền - practitioners.com/generic - subdomain - - >

% % %!<! - - Generic Subdomain : https:// thiết kế hướng miền - practitioners.com/generic - subdomain - - >

% \subsection{Các mẫu kỹ thuật (Tactical Patterns)}
% %!<! - - Tactical Design : https:// thiết kế hướng miền - practitioners.com/?page_id = 453 - - >


Trong thiết kế hướng miền, thiết kế chiến thuật đề cập đến quá trình thiết kế các thành phần riêng lẻ của hệ thống phần mềm để triển khai mô hình miền. Nó liên quan đến việc chia mô hình miền thành các đơn vị chức năng nhỏ hơn, gắn kết và liên kết lỏng lẻo được gọi là bối cảnh giới hạn, tập hợp, thực thể, đối tượng giá trị và dịch vụ miền. Thiết kế chiến thuật liên quan đến việc thiết kế các đơn vị này và mối quan hệ giữa chúng. Nó cũng liên quan đến việc lựa chọn các mẫu, chiến lược và chiến thuật phù hợp để đạt được mục tiêu thiết kế trong khi vẫn tuân thủ các nguyên tắc thiết kế hướng miền.

Các yếu tố của thiết kế chiến thuật trong Thiết kế hướng miền (thiết kế hướng miền) như sau:

Ngữ cảnh giới hạn : Ngữ cảnh giới hạn xác định ranh giới rõ ràng xung quanh một phần cụ thể của mô hình miền và ngôn ngữ chung đi kèm với nó.

Tập hợp : Tập hợp là các cụm đối tượng liên quan được coi là một đơn vị công việc duy nhất.

Thực thể : Thực thể là các đối tượng có danh tính và vòng đời duy nhất và có thể thay đổi theo thời gian.

Đối tượng giá trị : Đối tượng giá trị là các đối tượng không có danh tính duy nhất nhưng được xác định bởi các thuộc tính của chúng.

Dịch vụ : Dịch vụ là các hoạt động hoặc hành vi không phù hợp một cách tự nhiên trong một thực thể hoặc đối tượng giá trị.

Sự kiện miền : Sự kiện miền là những sự kiện quan trọng xảy ra trong miền mà các phần khác của hệ thống có thể cần biết.

Dịch vụ miền : Dịch vụ miền là các hoạt động hoặc hành vi áp dụng cho toàn bộ miền thay vì cho một thực thể hoặc đối tượng giá trị cụ thể.

Nhà máy : Nhà máy được sử dụng để tạo các đối tượng hoặc tập hợp phức tạp có thể yêu cầu nhiều bước hoặc logic phức tạp.

Kho lưu trữ : Kho lưu trữ được sử dụng để trừu tượng hóa việc lưu trữ và truy xuất các tập hợp và thực thể.

Thiết kế chiến thuật được gọi như vậy vì nó tập trung vào các quyết định chiến thuật liên quan đến mô hình hóa miền và triển khai logic nghiệp vụ trong mã. Thiết kế chiến thuật là quá trình triển khai các mẫu, nguyên tắc và thực tiễn thiết kế hướng miền để tạo ra một hệ thống phần mềm linh hoạt, có thể bảo trì và có thể mở rộng. Đây là một phần thiết yếu của thiết kế hướng miền giúp các nhà phát triển sắp xếp mã của họ và cải thiện chất lượng của hệ thống phần mềm. Thiết kế chiến thuật cung cấp một bộ nguyên tắc và phương pháp hay nhất mà nhà phát triển có thể làm theo để tạo mô hình miền phù hợp với yêu cầu kinh doanh và có thể phát triển theo thời gian.

%!<! - - $ Vẽ lại sau: - - >

%!<! - - $ Vẽ lại sau: - - >

%!<! - - $ Vẽ lại sau: - - >

%!<! - - $ Vẽ lại sau: - - >

%!<! - - $ Vẽ lại sau: - - >

%!<! - - $ Vẽ lại sau: - - >

%!<! - - $ Vẽ lại sau: - - >

%!<! - - $ Vẽ lại sau: - - >

%!<! - - $ Vẽ lại sau: - - >

%!<! - - $ Vẽ lại sau: - - >

%!<! - - $ Vẽ lại sau: - - >

%!<! - - $ Vẽ lại sau: - - >

%!<! - - $ Vẽ lại sau: - - >

% \subsubsection{xxxxxxx}
% 



Miền phụ chung cung cấp các giải pháp có sẵn mà doanh nghiệp có thể mua.

Doanh nghiệp không thể đạt được bất kỳ lợi thế cạnh tranh nào bằng cách thực hiện những điều khác biệt trong miền phụ chung.

% $????? VD: Các miền phụ chung như các hoạt động quản lý nhân sự và quản lý cơ sở vật chất không tạo thêm bất kỳ giá trị khác biệt nào cho doanh nghiệp. - - >


% % %!<! - - Generic Subdomain : https:// thiết kế hướng miền - practitioners.com/generic - subdomain - - >

% % %!<! - - Generic Subdomain : https:// thiết kế hướng miền - practitioners.com/generic - subdomain - - >

% Trang chủTrang chủBảng chú giảiLãnh địa Miền phụ chung

% Miền phụ chung

% Trong Thiết kế hướng miền (thiết kế hướng miền), miền phụ chung là loại miền phụ không có bất kỳ đặc điểm cụ thể hoặc duy nhất nào so với các miền khác trong cùng lĩnh vực. Đó là một miền phụ có thể được tìm thấy trên nhiều ngành, thay vì dành riêng cho một ngành hoặc miền.

% Mặc dù các miền phụ chung có thể không phải là duy nhất hoặc dành riêng cho một miền nhưng chúng vẫn cần được xác định rõ ràng và hiểu rõ để triển khai hiệu quả trong hệ thống.

% Ví dụ

% Xác thực và ủy quyền: Miền phụ này xử lý việc quản lý danh tính người dùng và quyền truy cập vào tài nguyên trong hệ thống. Thông thường, cần có một giải pháp chung cho miền phụ này để có thể sử dụng lại trên nhiều hệ thống.

% Thông báo : Miền phụ này xử lý việc gửi thông báo cho người dùng, chẳng hạn như thông báo qua email hoặc SMS. Tương tự như xác thực và ủy quyền, việc có một giải pháp chung cho miền phụ này có thể được sử dụng lại trên nhiều hệ thống thường rất hữu ích.

% Thanh toán : Miền phụ này xử lý các khoản thanh toán, bao gồm thu thập thông tin thanh toán, tính phí thẻ tín dụng và xử lý tiền hoàn lại. Tương tự như các ví dụ trên, giải pháp thanh toán chung có thể được sử dụng lại trên nhiều hệ thống.

% Định vị địa lý : Miền phụ này xử lý việc ánh xạ các vị trí thực tế tới các biểu diễn kỹ thuật số. Một giải pháp định vị địa lý chung có thể được sử dụng trong nhiều hệ thống, chẳng hạn như ánh xạ địa chỉ tới tọa độ GPS hoặc tính toán khoảng cách giữa các vị trí.

% % %!<! - - Generic Subdomain : https:// thiết kế hướng miền - practitioners.com/generic - subdomain - - >

% % %!<! - - Generic Subdomain : https:// thiết kế hướng miền - practitioners.com/generic - subdomain - - >

% % %!<! - - Generic Subdomain : https:// thiết kế hướng miền - practitioners.com/generic - subdomain - - >

% %
% Yêu cầu nghiệp vụ của từng sub
% %
% Sơ đồ if else Đ S
% %
% sub trước model
% %
%%%%%%%%%%%%%%%%%%%%%%%%%%%%%%%%%%%%%
\end{document}

\section{xxxxxxx}
\subsection{xxxxxxx}
\subsubsection{xxxxxxx}




Miền phụ chung cung cấp các giải pháp có sẵn mà doanh nghiệp có thể mua.

Doanh nghiệp không thể đạt được bất kỳ lợi thế cạnh tranh nào bằng cách thực hiện những điều khác biệt trong miền phụ chung.

% $????? VD: Các miền phụ chung như các hoạt động quản lý nhân sự và quản lý cơ sở vật chất không tạo thêm bất kỳ giá trị khác biệt nào cho doanh nghiệp. - - >


% % %!<! - - Generic Subdomain : https:// thiết kế hướng miền - practitioners.com/generic - subdomain - - >

% % %!<! - - Generic Subdomain : https:// thiết kế hướng miền - practitioners.com/generic - subdomain - - >

% Trang chủTrang chủBảng chú giảiLãnh địa Miền phụ chung

% Miền phụ chung

% Trong Thiết kế hướng miền (thiết kế hướng miền), miền phụ chung là loại miền phụ không có bất kỳ đặc điểm cụ thể hoặc duy nhất nào so với các miền khác trong cùng lĩnh vực. Đó là một miền phụ có thể được tìm thấy trên nhiều ngành, thay vì dành riêng cho một ngành hoặc miền.

% Mặc dù các miền phụ chung có thể không phải là duy nhất hoặc dành riêng cho một miền nhưng chúng vẫn cần được xác định rõ ràng và hiểu rõ để triển khai hiệu quả trong hệ thống.

% Ví dụ

% Xác thực và ủy quyền: Miền phụ này xử lý việc quản lý danh tính người dùng và quyền truy cập vào tài nguyên trong hệ thống. Thông thường, cần có một giải pháp chung cho miền phụ này để có thể sử dụng lại trên nhiều hệ thống.

% Thông báo : Miền phụ này xử lý việc gửi thông báo cho người dùng, chẳng hạn như thông báo qua email hoặc SMS. Tương tự như xác thực và ủy quyền, việc có một giải pháp chung cho miền phụ này có thể được sử dụng lại trên nhiều hệ thống thường rất hữu ích.

% Thanh toán : Miền phụ này xử lý các khoản thanh toán, bao gồm thu thập thông tin thanh toán, tính phí thẻ tín dụng và xử lý tiền hoàn lại. Tương tự như các ví dụ trên, giải pháp thanh toán chung có thể được sử dụng lại trên nhiều hệ thống.

% Định vị địa lý : Miền phụ này xử lý việc ánh xạ các vị trí thực tế tới các biểu diễn kỹ thuật số. Một giải pháp định vị địa lý chung có thể được sử dụng trong nhiều hệ thống, chẳng hạn như ánh xạ địa chỉ tới tọa độ GPS hoặc tính toán khoảng cách giữa các vị trí.

% % %!<! - - Generic Subdomain : https:// thiết kế hướng miền - practitioners.com/generic - subdomain - - >

% % %!<! - - Generic Subdomain : https:// thiết kế hướng miền - practitioners.com/generic - subdomain - - >

% % %!<! - - Generic Subdomain : https:// thiết kế hướng miền - practitioners.com/generic - subdomain - - >

% phải có CQRS (Phân chia trách nhiệm truy vấn lệnh)

CQRS là một mẫu kiến trúc riêng biệt có thể được sử dụng kết hợp với thiết kế hướng miền để đạt được những lợi ích nhất định, chẳng hạn như cải thiện hiệu suất và khả năng mở rộng. Tuy nhiên, nó không phải là một yêu cầu để triển khai thiết kế hướng miền.

% phải có event
Ngôn ngữ chung (Ubiquitous Language)

%%%%%%%%%%%%%%%%%%%%%%%%%%%%%%%%%%%%%
\end{document} % kết thúc