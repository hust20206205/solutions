%!<! - - @ - - >

Miền được tạo thành từ nhiều miền phụ.

%!<! - - $VD: Người dùng Sub - Domain, Thông báo Sub - Domain, Hóa đơn Sub - Domain - - >

Trong một miền phức tạp, không thể có một chuyên gia ngành có kiến thức về tất cả các miền phụ.
Việc xác định các miền phụ liên quan đến việc chia nhỏ các khả năng kinh doanh thành các đơn vị kinh doanh gắn kết.

%!<! - - @Phân loại các miền phụ - - >

Có 3 loại miền phụ:

%!<! - - @Miền phụ chung (Generic Subdomain) - - >

Miền phụ chung cung cấp các giải pháp có sẵn mà doanh nghiệp có thể mua.
Doanh nghiệp không thể đạt được bất kỳ lợi thế cạnh tranh nào bằng cách thực hiện những điều khác biệt trong miền phụ chung.

%!<! - - $????? VD: Các miền phụ chung như các hoạt động quản lý nhân sự và quản lý cơ sở vật chất không tạo thêm bất kỳ giá trị khác biệt nào cho doanh nghiệp. - - >

%!<! - - @Miền phụ cốt lõi (Core Subdomain) - - >

Miền phụ cốt lõi là điểm khác biệt quan trọng cho doanh nghiệp.

Thành công của một doanh nghiệp nằm ở miền phụ cốt lõi. Vì mỗi doanh nghiệp trong một ngành cụ thể hoạt động khác nhau trong các miền phụ cốt lõi để đạt được một số lợi thế so với đối thủ cạnh tranh.

= > Doanh nghiệp luôn tìm cách thực hiện những điều khác biệt trong các miền phụ cốt lõi này để có được một số lợi thế cạnh tranh.

%!<! - - $????? VD: - - >

%!<! - - @Miền phụ hỗ trợ (Supporting Subdomain) - - >

Các miền phụ cốt lõi phụ thuộc vào các miền phụ hỗ trợ.

Miền phụ hỗ trợ cung cấp các dịch vụ để miền phụ cốt lõi hoạt động hiệu quả.

Miền phụ hỗ trợ không có mức độ phức tạp cao về logic nghiệp vụ.

%!<! - - $????? VD: miền phụ hỗ trợ chăm sóc khách hàng - - >

%!<! - - @Cách xác định các miền phụ - - >
%!<! - - Sơ đồ: - - >

![](pictures/XacDinhMienPhu/_XacDinhMienPhu.png)

%!<! - - Mô tả: - - >

Bắt đầu bằng cách xem xét nghiệp vụ kinh doanh.

Nếu có sẵn giải pháp đã biết thì có khả năng là miền phụ chung. Ngược lại, chúng ta kiểm tra xem miền phụ đó có thêm giá trị kinh doanh nào không?

Nếu không có giá trị kinh doanh thì chúng ta kiểm tra xem các miền phụ cốt lõi có phụ thuộc vào miền phụ này hay không? Nếu có thì có khả năng là miền phụ hỗ trợ. Nếu không thì đó là miền phụ chung.

Nếu miền phụ có tiềm năng bổ sung một số giá trị kinh doanh thì bước kiểm tra tiếp theo là xem liệu miền doanh nghiệp có độ phức tạp cao hay không?

Nếu miền doanh nghiệp không có độ phức tạp cao thì có khả năng là miền phụ hỗ trợ. Ngược lại thì nó có khả năng là miền phụ cốt lõi.

%!<! - - @Tại sao cần phân loại các miền phụ? - - >

Việc phân loại miền phụ giúp doanh nghiệp đưa ra quyết định với từng loại miền phụ khác nhau.

Doanh nghiệp có nguồn lực hạn chế như nguồn nhân lực và kinh phí dành cho các sáng kiến. Việc phân loại các miền phụ giúp ưu tiên các sáng kiến khác nhau.

Các doanh nghiệp mong muốn tối đa hóa lợi nhuận đầu tư. Do đó, các sáng kiến liên quan đến miền phụ cốt lõi sẽ được ưu tiên.

%!<! - - Hướng dẫn: 5/3 - - >
%