Để đơn giản hóa bài toán, các chức năng trong đồ án này đã thay đổi so với bài toán thực tế trong tài liệu hướng dẫn sử dụng cổng thông tin điện tử của Tổng Cục Thuế cho hóa đơn điện tử:

\textbf{Em đã bỏ qua hình thức hóa đơn:}

Hóa đơn có mã của cơ quan thuế

Hóa đơn không có mã của cơ quan thuế

\textbf{Bỏ qua các loại hóa đơn khác nhau:}

Hóa đơn điện tử giá trị gia tăng

Hóa đơn bán hàng

Hóa đơn bán tài sản công

Hóa đơn bán hàng dự trữ quốc gia

Hóa đơn khác

Phiếu xuất kho kiêm vận chuyển nội bộ

Phiếu xuất kho gửi bán hàng đại lý

\textbf{Bỏ qua phần ký số:}
USB Token
chữ ký số theo đúng quy định của Bộ Thông tin và Truyền thông mang  thông tin chứng thư số.

% \textbf{ChuInDam}


<! - - Ký hiệu hóa đơn - - >

<! - - Chức năng: "Lập hóa đơn điều chỉnh" - - >

<! - - Không có phê duyệt hóa đơn - - >

<! - - Đọc hóa đơn XML - - >

<! - - Có mã của cơ quan thuế - - >

<! - - Không có mã của cơ quan thuế - - >

<! - - Phát hành hóa đơn điện tử - - >

<! - - CHƯƠNG V. THÔNG BÁO HÓA ĐƠN CÓ SAI SÓT.............................. 130 - - >

<! - - CHƯƠNG VI. ĐỀ NGHỊ CẤP HÓA ĐƠN THEO LẦN PHÁT SINH..... 143 - - >

<! - - CHƯƠNG VII. QUẢN LÝ HÓA ĐƠN PHÁT SINH........................ - - >

<! - - - - >

\textbf{Các chức năng tổng quan của Tổng Cục Thuế Demo bao gồm:}

% QUẢN LÝ TÀI KHOẢN

Đăng ký sử dụng hóa đơn điện tử

Thay đổi đăng ký sử dụng hóa đơn điện tử

Đăng nhập tài khoản

Đăng xuất tài khoản

Đổi mật khẩu

Quên mật khẩu

% QUẢN LÝ HỆ THỐNG

Quản lý vai trò

Quản lý người dùng

% QUẢN LÝ DANH MỤC

Danh mục khách hàng

Danh mục hàng hóa

% QUẢN LÝ HÓA ĐƠN ĐIỆN TỬ

Lập hóa đơn mới

Lập hóa đơn thay thế

Hủy hóa đơn

% TRA CỨU HÓA ĐƠN

Tra cứu hóa đơn khi NNT chưa đăng nhập

Tra cứu hóa đơn khi NNT đã đăng nhập

% GỬI PHẢN HỒI QUA THƯ ĐIỆN TỬ

Gửi thông tin của TCT đến NNT