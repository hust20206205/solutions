Phần mềm được tạo ra để xử lý sự phức tạp trong cuộc sống hiện đại. Việc phát triển phần mềm liên kết chặt chẽ với một số khía cạnh cụ thể trong cuộc sống của chúng ta.

Miền (Domain) đề cập đến phạm vi kiến thức và vấn đề mà hệ thống hoặc dự án cụ thể đang xử lý.

Về góc độ kinh doanh: miền đại diện cho một lĩnh vực hoặc ngành mà doanh nghiệp hoạt động.

Về góc độ phần mềm: miền có thể coi là đại diện cho không gian vấn đề của phần mềm đó.

Phần mềm cần phản ánh đúng miền và hiện thực hóa chính xác miền.

%!<! - - $VD: Trong đồ án này, miền được xác định là bài toán giải pháp hóa đơn điện tử. - - >

% %!<! - - Domain : https:// thiết kế hướng miền - practitioners.com/domain - - >

% %!<! - - Domain : https:// thiết kế hướng miền - practitioners.com/domain - - >

% %!<! - - Domain : https:// thiết kế hướng miền - practitioners.com/domain - - >

% %!<! - - Domain : https:// thiết kế hướng miền - practitioners.com/domain - - >

% %!<! - - Domain : https:// thiết kế hướng miền - practitioners.com/domain - - >

% %!<! - - Domain : https:// thiết kế hướng miền - practitioners.com/domain - - >

% %!<! - - Domain : https:// thiết kế hướng miền - practitioners.com/domain - - >

% %!<! - - Domain : https:// thiết kế hướng miền - practitioners.com/domain - - >

%!<! - - [[Domain]] A sphere of knowledge, influence, or activity. - - >

Trang chủTrang chủBảng chú giảiLãnh địa

Lãnh địa

Trong thiết kế hướng miền (thiết kế hướng miền), miền là một phạm vi kiến thức, ảnh hưởng hoặc hoạt động đại diện cho một lĩnh vực chuyên môn hoặc mối quan tâm cụ thể. Miền là không gian vấn đề mà phần mềm đang được xây dựng để giải quyết và nó thể hiện vấn đề hoặc cơ hội kinh doanh mà phần mềm có nhiệm vụ giải quyết.

Miền có thể là một ngành cụ thể, chẳng hạn như chăm sóc sức khỏe, tài chính hoặc thương mại điện tử hoặc có thể là một lĩnh vực quan tâm cụ thể trong một ngành, chẳng hạn như quản lý hàng tồn kho, quản lý khách hàng hoặc hậu cần.

Trong thiết kế hướng miền, mục tiêu là tạo ra một mô hình miền thể hiện chính xác các khái niệm và mối quan hệ trong thế giới thực trong miền và sử dụng mô hình này làm cơ sở cho việc thiết kế và phát triển phần mềm. Mô hình miền này phải phản ánh miền kinh doanh, nó phải dễ hiểu đối với các chuyên gia ngành, nhà phát triển và các bên liên quan và nó phải là xương sống của hệ thống phần mềm.

Một miền có thể được xác định bởi các quy tắc kinh doanh, quy trình kinh doanh, các thực thể kinh doanh và các mối quan hệ của chúng, các yêu cầu kinh doanh và mục tiêu kinh doanh.

Miền là trái tim của thiết kế hướng miền, là nền tảng của phần mềm và là điểm khởi đầu của quá trình phát triển. Hiểu miền là rất quan trọng để có thể tạo ra một phần mềm phù hợp với nhu cầu kinh doanh, ít phức tạp hơn, dễ bảo trì hơn và dễ thích ứng hơn với thay đổi.

Xem thêm	 Miền phụ, miền vấn đề, miền giải pháp, bối cảnh giới hạn

% %!<! - - Problem Domain :https:// thiết kế hướng miền - practitioners.com/home/glossary/problem - domain - - >

% %!<! - - Problem Domain :https:// thiết kế hướng miền - practitioners.com/home/glossary/problem - domain - - >

% %!<! - - Problem Domain :https:// thiết kế hướng miền - practitioners.com/home/glossary/problem - domain - - >

% %!<! - - Problem Domain :https:// thiết kế hướng miền - practitioners.com/home/glossary/problem - domain - - >

% %!<! - - Problem Domain :https:// thiết kế hướng miền - practitioners.com/home/glossary/problem - domain - - >

% %!<! - - Problem Domain :https:// thiết kế hướng miền - practitioners.com/home/glossary/problem - domain - - >

% %!<! - - Problem Domain :https:// thiết kế hướng miền - practitioners.com/home/glossary/problem - domain - - >

% %!<! - - Problem Domain :https:// thiết kế hướng miền - practitioners.com/home/glossary/problem - domain - - >

Trang chủTrang chủBảng chú giảiMiền vấn đề

Miền vấn đề

Trong Thiết kế hướng miền (thiết kế hướng miền), miền vấn đề đề cập đến lĩnh vực kiến thức hoặc hoạt động kinh doanh cụ thể mà hệ thống phần mềm đang được phát triển để giải quyết. Đó là lĩnh vực chuyên môn mà phần mềm dự định hỗ trợ, chẳng hạn như tài chính, chăm sóc sức khỏe, thương mại điện tử, v.v. Miền vấn đề là bối cảnh mà phần mềm sẽ được sử dụng và các yêu cầu cụ thể mà phần mềm cần đáp ứng.

Miền vấn đề thường được xác định bởi các bên liên quan của dự án, chẳng hạn như người dùng cuối, chuyên gia ngành và nhà phân tích kinh doanh. Mục tiêu của thiết kế hướng miền là mô hình hóa miền vấn đề một cách chính xác nhất có thể bằng cách tạo ra một mô hình miền phản ánh chính xác các khái niệm và quy trình kinh doanh cơ bản. Điều này đạt được bằng cách xác định các khái niệm và khái niệm trừu tượng chính trong miền vấn đề và tạo ra sự hiểu biết chung về các khái niệm này thông qua việc sử dụng ngôn ngữ chung.

Tóm lại, miền vấn đề là lĩnh vực cụ thể của doanh nghiệp mà hệ thống phần mềm đang được phát triển để giải quyết và thiết kế hướng miền là một phương pháp giúp mô hình hóa miền vấn đề một cách chính xác và tạo ra sự hiểu biết chung về các khái niệm cơ bản thông qua việc sử dụng một ngôn ngữ chung.

% %!<! - - Solution Domain :https:// thiết kế hướng miền - practitioners.com/home/glossary/solution - domain - - >

% %!<! - - Solution Domain :https:// thiết kế hướng miền - practitioners.com/home/glossary/solution - domain - - >

% %!<! - - Solution Domain :https:// thiết kế hướng miền - practitioners.com/home/glossary/solution - domain - - >

% %!<! - - Solution Domain :https:// thiết kế hướng miền - practitioners.com/home/glossary/solution - domain - - >

% %!<! - - Solution Domain :https:// thiết kế hướng miền - practitioners.com/home/glossary/solution - domain - - >

% %!<! - - Solution Domain :https:// thiết kế hướng miền - practitioners.com/home/glossary/solution - domain - - >

% %!<! - - Solution Domain :https:// thiết kế hướng miền - practitioners.com/home/glossary/solution - domain - - >

% %!<! - - Solution Domain :https:// thiết kế hướng miền - practitioners.com/home/glossary/solution - domain - - >

Trang chủTrang chủBảng chú giảiMiền giải pháp

Miền giải pháp

Trong Thiết kế hướng miền (thiết kế hướng miền), miền giải pháp đề cập đến hệ thống phần mềm cụ thể đang được phát triển để giải quyết miền vấn đề. Miền giải pháp bao gồm thiết kế, kiến trúc và triển khai hệ thống phần mềm cũng như cách thức ánh xạ tới miền vấn đề.

Miền giải pháp bao gồm mã, thành phần và dịch vụ tạo nên hệ thống phần mềm cũng như các mẫu thiết kế và nguyên tắc kiến trúc được sử dụng để cấu trúc hệ thống. Nó cũng bao gồm các quyết định thiết kế cụ thể được đưa ra trong quá trình phát triển, chẳng hạn như lựa chọn ngôn ngữ và khung lập trình cũng như việc sử dụng các mẫu thiết kế và phong cách kiến trúc cụ thể.

Mục tiêu của thiết kế hướng miền là căn chỉnh miền giải pháp với miền vấn đề càng chặt chẽ càng tốt. Điều này đạt được bằng cách tạo ra một mô hình miền phản ánh chính xác các khái niệm và quy trình kinh doanh cơ bản trong miền có vấn đề và bằng cách sử dụng ngôn ngữ chung để tạo ra sự hiểu biết chung về các khái niệm này trong toàn nhóm phát triển.

Tóm lại, miền giải pháp là hệ thống phần mềm cụ thể đang được phát triển để giải quyết miền vấn đề và thiết kế hướng miền là một phương pháp giúp điều chỉnh miền giải pháp với miền vấn đề càng chặt chẽ càng tốt bằng cách tạo mô hình miền phản ánh chính xác các khái niệm và quy trình kinh doanh cơ bản trong lĩnh vực vấn đề và bằng cách sử dụng ngôn ngữ chung để tạo ra sự hiểu biết chung về các khái niệm này trong nhóm phát triển.

