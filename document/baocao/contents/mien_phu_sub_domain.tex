Miền được tạo thành từ nhiều miền phụ.

Ví dụ: Trong miền thương mại điện tử lớn. Có thể có một số miền phụ:

\begin{itemize}

\item Miền phụ quản lý hàng tồn kho: liên quan đến việc quản lý sản phẩm trong kho hàng.

\item Miền phụ quản lý khách hàng: liên quan đến việc quản lý tài khoản khách hàng.

\item Miền phụ vận chuyển: liên quan đến việc quản lý việc vận chuyển giao hàng.

\end{itemize}

Trong một miền phức tạp, không thể có một chuyên gia ngành có kiến thức về tất cả các miền phụ.

Có ba loại miền phụ:

\begin{itemize}

\item Miền phụ chung (Generic Subdomain)

\item Miền phụ cốt lõi (Core Subdomain)

\item Miền phụ hỗ trợ (Supporting Subdomain)

\end{itemize}

\end{document}

\subsubsection{xxxxxxxxxxxxxxx}

%!<! - - @Phân loại các miền phụ - - >

Có 3 loại miền phụ:

%!<! - - @Miền phụ chung (Generic Subdomain) - - >

Miền phụ chung cung cấp các giải pháp có sẵn mà doanh nghiệp có thể mua.

Doanh nghiệp không thể đạt được bất kỳ lợi thế cạnh tranh nào bằng cách thực hiện những điều khác biệt trong miền phụ chung.

%!<! - - $????? VD: Các miền phụ chung như các hoạt động quản lý nhân sự và quản lý cơ sở vật chất không tạo thêm bất kỳ giá trị khác biệt nào cho doanh nghiệp. - - >

%!<! - - @Miền phụ cốt lõi (Core Subdomain) - - >

%!<! - - [[Core Domain]] The distinctive part of the model, central to the user’s goals, that differentiates the application and makes it valuable. - - >

Miền phụ cốt lõi là điểm khác biệt quan trọng cho doanh nghiệp.

Thành công của một doanh nghiệp nằm ở miền phụ cốt lõi. Vì mỗi doanh nghiệp trong một ngành cụ thể hoạt động khác nhau trong các miền phụ cốt lõi để đạt được một số lợi thế so với đối thủ cạnh tranh.

= > Doanh nghiệp luôn tìm cách thực hiện những điều khác biệt trong các miền phụ cốt lõi này để có được một số lợi thế cạnh tranh.

%!<! - - $????? VD: - - >

%!<! - - @Miền phụ hỗ trợ (Supporting Subdomain) - - >

Các miền phụ cốt lõi phụ thuộc vào các miền phụ hỗ trợ.

Miền phụ hỗ trợ cung cấp các dịch vụ để miền phụ cốt lõi hoạt động hiệu quả.

Miền phụ hỗ trợ không có mức độ phức tạp cao về logic nghiệp vụ.

%!<! - - $????? VD: miền phụ hỗ trợ chăm sóc khách hàng - - >

%!<! - - @Cách xác định các miền phụ - - >

%!<! - - Sơ đồ: - - >

% ![](pictures/XacDinhMienPhu/_XacDinhMienPhu.png)

%!<! - - Mô tả: - - >

Bắt đầu bằng cách xem xét nghiệp vụ kinh doanh.

Nếu có sẵn giải pháp đã biết thì có khả năng là miền phụ chung. Ngược lại, chúng ta kiểm tra xem miền phụ đó có thêm giá trị kinh doanh nào không?

Nếu không có giá trị kinh doanh thì chúng ta kiểm tra xem các miền phụ cốt lõi có phụ thuộc vào miền phụ này hay không? Nếu có thì có khả năng là miền phụ hỗ trợ. Nếu không thì đó là miền phụ chung.

Nếu miền phụ có tiềm năng bổ sung một số giá trị kinh doanh thì bước kiểm tra tiếp theo là xem liệu miền doanh nghiệp có độ phức tạp cao hay không?

Nếu miền doanh nghiệp không có độ phức tạp cao thì có khả năng là miền phụ hỗ trợ. Ngược lại thì nó có khả năng là miền phụ cốt lõi.

%!<! - - @Tại sao cần phân loại các miền phụ? - - >

Việc phân loại miền phụ giúp doanh nghiệp đưa ra quyết định với từng loại miền phụ khác nhau.

Doanh nghiệp có nguồn lực hạn chế như nguồn nhân lực và kinh phí dành cho các sáng kiến. Việc phân loại các miền phụ giúp ưu tiên các sáng kiến khác nhau.

Các doanh nghiệp mong muốn tối đa hóa lợi nhuận đầu tư. Do đó, các sáng kiến liên quan đến miền phụ cốt lõi sẽ được ưu tiên.

%!<! - - Hướng dẫn: 5/3 - - >

%

%!<! - - - - >

%!<! - - - - >

%!<! - - - - >

% %!<! - - Core Domain https:// thiết kế hướng miền - practitioners.com/home/glossary/domain/core - domain/ - - >

% %!<! - - Core Domain https:// thiết kế hướng miền - practitioners.com/home/glossary/domain/core - domain/ - - >

Trang chủTrang chủBảng chú giảiLãnh địa Miền cốt lõi

Miền cốt lõi

Miền lõi hoặc miền phụ trong thiết kế hướng miền (Thiết kế theo hướng miền) là một phần của hệ thống phần mềm chứa logic và quy trình kinh doanh chính, thể hiện trung tâm chức năng của ứng dụng. Đây là phần quan trọng và có giá trị nhất của hệ thống và việc triển khai nó có ý nghĩa quyết định đối với sự thành công của phần mềm. Miền cốt lõi được xác định thông qua phân tích cẩn thận về miền vấn đề cũng như các quy tắc và quy trình kinh doanh tương ứng của nó và nó phải được xác định rõ ràng, theo mô - đun và có thể bảo trì được. Trong thiết kế hướng miền, miền lõi thường được gói gọn trong một tập hợp các đối tượng gắn kết và liên kết lỏng lẻo, được gọi là mô hình miền, mô hình hóa các khái niệm, quy tắc và quy trình kinh doanh của miền vấn đề.

Ví dụ

Dưới đây là danh sách các miền phụ cốt lõi có thể có cho một doanh nghiệp hoạt động trong miền thẻ tín dụng :

Phát hành thẻ : Subdomain này chịu trách nhiệm về quá trình phát hành thẻ tín dụng mới cho khách hàng. Nó bao gồm các nhiệm vụ như thu thập thông tin khách hàng, thực hiện kiểm tra tín dụng, in và giao thẻ vật lý cũng như kích hoạt thẻ.

Thanh toán bằng thẻ : Miền phụ này xử lý việc xử lý các giao dịch thẻ tín dụng. Nó bao gồm các nhiệm vụ như ủy quyền thanh toán, thu hồi thanh toán, thanh toán tiền cho người bán và quản lý khoản bồi hoàn.

Phát hiện gian lận : Miền phụ này tập trung vào việc phát hiện và ngăn chặn hoạt động gian lận trên tài khoản thẻ tín dụng. Nó bao gồm các nhiệm vụ như phân tích dữ liệu giao dịch, giám sát các mẫu đáng ngờ và bắt đầu điều tra gian lận.

Chương trình phần thưởng : Miền phụ này quản lý các chương trình khách hàng thân thiết cung cấp phần thưởng cho khách hàng khi sử dụng thẻ tín dụng của họ. Nó bao gồm các nhiệm vụ như xác định cấu trúc phần thưởng, theo dõi điểm và phần thưởng của khách hàng cũng như quản lý việc quy đổi phần thưởng.

Dịch vụ khách hàng : Miền phụ này xử lý các yêu cầu và hỗ trợ của khách hàng liên quan đến tài khoản thẻ tín dụng. Nó bao gồm các nhiệm vụ như trả lời các câu hỏi về số dư tài khoản, hỗ trợ các tranh chấp về thanh toán và giải quyết các vấn đề kỹ thuật khi sử dụng thẻ.

Tuân thủ : Miền phụ này đảm bảo rằng hoạt động kinh doanh thẻ tín dụng hoạt động phù hợp với luật pháp và quy định có liên quan. Nó bao gồm các nhiệm vụ như giám sát các vi phạm tuân thủ, duy trì tài liệu quy định và thực hiện các thay đổi cần thiết để luôn tuân thủ.

Cần lưu ý rằng ý tưởng về miền phụ cốt lõi, hỗ trợ và chung có thể khác nhau ngay cả đối với các doanh nghiệp hoạt động trong cùng một miền. Điều này là do các miền phụ và vai trò của chúng được xác định theo nhu cầu kinh doanh và bối cảnh cụ thể của mỗi tổ chức. Ví dụ: trong miền thẻ tín dụng, một công ty tập trung vào các chương trình phần thưởng thẻ tín dụng có thể coi miền phụ của chương trình phần thưởng là cốt lõi, trong khi một công ty khác có trọng tâm khác có thể coi nó là hỗ trợ hoặc chung chung. Tương tự, một công ty tập trung mạnh vào việc ngăn chặn gian lận có thể coi miền phụ phát hiện gian lận là cốt lõi, trong khi một công ty khác có thể coi nó là miền hỗ trợ hoặc chung chung. Do đó, điều quan trọng là mỗi tổ chức phải xác định và ưu tiên các miền phụ cụ thể dựa trên nhu cầu và mục tiêu kinh doanh riêng của họ.

% %!<! - - Core Domain https:// thiết kế hướng miền - practitioners.com/home/glossary/domain/core - domain/ - - >

% %!<! - - Core Domain https:// thiết kế hướng miền - practitioners.com/home/glossary/domain/core - domain/ - - >

% %!<! - - Core Domain https:// thiết kế hướng miền - practitioners.com/home/glossary/domain/core - domain/ - - >

Cần lưu ý rằng ý tưởng về miền phụ cốt lõi, hỗ trợ và chung có thể khác nhau ngay cả đối với các doanh nghiệp hoạt động trong cùng một miền. Điều này là do các miền phụ và vai trò của chúng được xác định theo nhu cầu kinh doanh và bối cảnh cụ thể của mỗi tổ chức. Ví dụ:

% %!<! - - Highlighted Core : https:// thiết kế hướng miền - practitioners.com/highlighted - core - - >

% %!<! - - Highlighted Core : https:// thiết kế hướng miền - practitioners.com/highlighted - core - - >

Trang chủTrang chủBảng chú giảiLãnh địa Miền cốt lõiCốt lõi nổi bật

Cốt lõi nổi bật

Trong ngữ cảnh của thiết kế hướng miền, phần cốt lõi được đánh dấu đề cập đến phần quan trọng và phức tạp nhất của hệ thống phần mềm, thể hiện logic miền cốt lõi và mang lại giá trị cao nhất cho doanh nghiệp. Phần lõi này phải được tách biệt khỏi các miền phụ hỗ trợ và chung, đồng thời phải được phát triển và duy trì bởi các chuyên gia và nhà phát triển miền có hiểu biết sâu sắc về doanh nghiệp cũng như các yêu cầu của nó. Lõi được đánh dấu phải được bảo vệ và cách ly khỏi những thay đổi và sửa đổi không liên quan trực tiếp đến miền lõi, điều này đạt được thông qua việc sử dụng các bối cảnh giới hạn, các sự kiện miền và các khái niệm thiết kế hướng miền khác. Bằng cách giữ phần lõi được đánh dấu tách biệt khỏi các miền phụ hỗ trợ và chung, hệ thống có thể duy trì mức độ gắn kết và mô đun hóa cao, giúp dễ hiểu, duy trì và phát triển hơn theo thời gian.

% %!<! - - Highlighted Core : https:// thiết kế hướng miền - practitioners.com/highlighted - core - - >

% %!<! - - Highlighted Core : https:// thiết kế hướng miền - practitioners.com/highlighted - core - - >

% %!<! - - Highlighted Core : https:// thiết kế hướng miền - practitioners.com/highlighted - core - - >

% %!<! - - Segregated Core : https:// thiết kế hướng miền - practitioners.com/?page_id = 378 - - >

% %!<! - - Segregated Core : https:// thiết kế hướng miền - practitioners.com/?page_id = 378 - - >

Trang chủTrang chủBảng chú giảiLãnh địa Miền cốt lõiLõi tách biệt

Lõi tách biệt

Lõi tách biệt là một mẫu thiết kế hướng miền bao gồm việc tách miền lõi thành các phần hoặc mô - đun nhỏ hơn, độc lập, mỗi phần có bối cảnh giới hạn riêng. Điều này cho phép tính linh hoạt và khả năng mở rộng cao hơn trong việc thiết kế các hệ thống phức tạp, cũng như cải thiện tính mô - đun và khả năng bảo trì. Ý tưởng là để tránh việc có một miền cốt lõi nguyên khối, được liên kết chặt chẽ, có thể trở nên khó thay đổi hoặc duy trì theo thời gian. Bằng cách tách phần cốt lõi thành các phần nhỏ hơn, tập trung, mỗi phần có bối cảnh và trách nhiệm riêng, các nhóm có thể làm việc hiệu quả hơn và thực hiện các thay đổi dễ dàng hơn. Mẫu lõi tách biệt thường được sử dụng kết hợp với các mẫu thiết kế hướng miền khác, chẳng hạn như bối cảnh giới hạn và bản đồ bối cảnh, để tạo ra kiến trúc có cấu trúc tốt và linh hoạt cho các hệ thống quy mô lớn.

% %!<! - - Segregated Core : https:// thiết kế hướng miền - practitioners.com/?page_id = 378 - - >

% %!<! - - Segregated Core : https:// thiết kế hướng miền - practitioners.com/?page_id = 378 - - >

% %!<! - - Segregated Core : https:// thiết kế hướng miền - practitioners.com/?page_id = 378 - - >

% %!<! - - Generic Subdomain : https:// thiết kế hướng miền - practitioners.com/generic - subdomain - - >

% %!<! - - Generic Subdomain : https:// thiết kế hướng miền - practitioners.com/generic - subdomain - - >

Trang chủTrang chủBảng chú giảiLãnh địa Miền phụ chung

Miền phụ chung

Trong Thiết kế hướng miền (thiết kế hướng miền), miền phụ chung là loại miền phụ không có bất kỳ đặc điểm cụ thể hoặc duy nhất nào so với các miền khác trong cùng lĩnh vực. Đó là một miền phụ có thể được tìm thấy trên nhiều ngành, thay vì dành riêng cho một ngành hoặc miền.

Mặc dù các miền phụ chung có thể không phải là duy nhất hoặc dành riêng cho một miền nhưng chúng vẫn cần được xác định rõ ràng và hiểu rõ để triển khai hiệu quả trong hệ thống.

Ví dụ

Xác thực và ủy quyền: Miền phụ này xử lý việc quản lý danh tính người dùng và quyền truy cập vào tài nguyên trong hệ thống. Thông thường, cần có một giải pháp chung cho miền phụ này để có thể sử dụng lại trên nhiều hệ thống.

Thông báo : Miền phụ này xử lý việc gửi thông báo cho người dùng, chẳng hạn như thông báo qua email hoặc SMS. Tương tự như xác thực và ủy quyền, việc có một giải pháp chung cho miền phụ này có thể được sử dụng lại trên nhiều hệ thống thường rất hữu ích.

Thanh toán : Miền phụ này xử lý các khoản thanh toán, bao gồm thu thập thông tin thanh toán, tính phí thẻ tín dụng và xử lý tiền hoàn lại. Tương tự như các ví dụ trên, giải pháp thanh toán chung có thể được sử dụng lại trên nhiều hệ thống.

Định vị địa lý : Miền phụ này xử lý việc ánh xạ các vị trí thực tế tới các biểu diễn kỹ thuật số. Một giải pháp định vị địa lý chung có thể được sử dụng trong nhiều hệ thống, chẳng hạn như ánh xạ địa chỉ tới tọa độ GPS hoặc tính toán khoảng cách giữa các vị trí.

% %!<! - - Generic Subdomain : https:// thiết kế hướng miền - practitioners.com/generic - subdomain - - >

% %!<! - - Generic Subdomain : https:// thiết kế hướng miền - practitioners.com/generic - subdomain - - >

% %!<! - - Generic Subdomain : https:// thiết kế hướng miền - practitioners.com/generic - subdomain - - >

% %!<! - - Supporting Subdomain : https:// thiết kế hướng miền - practitioners.com/supporting - subdomain - - >

% %!<! - - Supporting Subdomain : https:// thiết kế hướng miền - practitioners.com/supporting - subdomain - - >

Trang chủTrang chủBảng chú giảiHỗ trợ miền phụ

Hỗ trợ miền phụ

Trong thiết kế hướng miền, miền phụ hỗ trợ đề cập đến miền phụ hỗ trợ miền phụ cốt lõi để đạt được mục tiêu của nó. Nó bao gồm các chức năng như quản trị, bảo mật, giám sát và báo cáo, cùng nhiều chức năng khác. Miền phụ hỗ trợ rất quan trọng trong việc cung cấp các dịch vụ thiết yếu cho miền phụ cốt lõi, có thể cho phép miền phụ đạt được mục tiêu hiệu quả hơn. Tuy nhiên, nó không được thống trị hoặc làm lu mờ miền phụ cốt lõi, vì nó chỉ nhằm mục đích tạo điều kiện thuận lợi cho hoạt động của nó.

% %!<! - - Supporting Subdomain : https:// thiết kế hướng miền - practitioners.com/supporting - subdomain - - >

% %!<! - - Supporting Subdomain : https:// thiết kế hướng miền - practitioners.com/supporting - subdomain - - >

% %!<! - - Supporting Subdomain : https:// thiết kế hướng miền - practitioners.com/supporting - subdomain - - >

% %!<! - - - - >

% %!<! - - - - >

% %!<! - - - - >

% %!<! - - - - >