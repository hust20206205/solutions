Thiết kế hướng miền được Eric Evans giới thiệu trong cuốn sách "DomainDrivenDesign: Tackling Complexity in the Heart of Software".

Thiết kế hướng miền (DomainDrivenDesign) là một phương pháp thiết kế phần mềm tập trung vào việc hiểu rõ và mô hình hóa lĩnh vực kinh doanh của một tổ chức.

Thiết kế hướng miền nhấn mạnh việc sử dụng lĩnh vực nghiệp vụ kinh doanh để thảo luận và đề xuất giải pháp đáp ứng nhu cầu. Vì để tạo một phần mềm tốt, chúng ta cần phải hiểu rõ về chính phần mềm đó. Chính vì vậy để đạt được kết quả như mong đợi, chúng ta thường bắt đầu từ yêu cầu nghiệp vụ.

Trong nhiều ứng dụng thường có phần xử lý các công việc không liên quan đến vấn đề nghiệp vụ như truy cập file, hạ tầng mạng, CSDL,... trong đối tượng nghiệp vụ kinh doanh. Cách này giúp tốc độ hoàn thiện ứng dụng nhanh. Tuy nhiên, cách này làm cho thiết kế bị mất đi tính hướng đối tượng trong thực tế với mức độ doanh nghiệp lớn. Đây là lý do thiết kế hướng miền trở nên quan trọng.

Trong kiến trúc vi dịch vụ, thiết kế hướng miền giúp đảm bảo rằng mỗi dịch vụ được thiết kế phản ánh một phần cụ thể của lĩnh vực kinh doanh. Mỗi dịch vụ được quản lí bởi một nhóm nhỏ được hỗ trợ bởi các chuyên gia ngành.


% \subsection{Chuyên gia ngành}

An incisive expression of the primary concerns of the chuyên gia ngành s and their most relevant knowledge. A deep model sloughs off superficial aspects of the domain and naive interpretations.

%!<! - - [[Domain Expert]] A member of a software project whose field is the domain of the application, rather than software development. Not just any user of the software, the chuyên gia ngành has deep knowledge of the subject. - - >
 

Chuyên gia ngành

Trong Thiết kế hướng miền (thiết kế hướng miền), chuyên gia ngành là người có kiến thức và hiểu biết sâu sắc về miền kinh doanh hoặc lĩnh vực vấn đề đang được hệ thống phần mềm giải quyết. Chuyên gia ngành có kiến thức chuyên môn về các quy tắc, quy trình và khái niệm kinh doanh liên quan đến hệ thống đang được xây dựng và đóng vai trò là nguồn thông tin chính cho nhóm phát triển. Chuyên gia ngành giúp đảm bảo rằng mô hình miền thể hiện chính xác miền doanh nghiệp và hệ thống đang được xây dựng giải quyết đúng vấn đề cũng như giải quyết đúng nhu cầu kinh doanh.

Sự thể hiện sâu sắc về mối quan tâm hàng đầu của các ngành lớn và kiến thức phù hợp nhất của họ. Một mô hình sâu sắc sẽ loại bỏ các khía cạnh hời hợt của lĩnh vực này và những diễn giải ngây thơ.

%!<! - - [[Chuyên gia miền]] Thành viên của một dự án phần mềm có lĩnh vực là miền ứng dụng chứ không phải phát triển phần mềm. Không chỉ bất kỳ người sử dụng phần mềm nào, các chuyên ngành đều có kiến thức sâu rộng về chủ đề này. - - >