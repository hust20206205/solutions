Thiết kế hướng miền được Eric Evans giới thiệu trong cuốn sách "DomainDrivenDesign: Tackling Complexity in the Heart of Software".

Thiết kế hướng miền (DomainDrivenDesign) là một phương pháp thiết kế phần mềm tập trung vào việc hiểu rõ và mô hình hóa lĩnh vực kinh doanh của một tổ chức.

Thiết kế hướng miền nhấn mạnh việc sử dụng lĩnh vực nghiệp vụ kinh doanh để thảo luận và đề xuất giải pháp đáp ứng nhu cầu. Vì để tạo một phần mềm tốt, chúng ta cần phải hiểu rõ về chính phần mềm đó. Chính vì vậy để đạt được kết quả như mong đợi, chúng ta thường bắt đầu từ yêu cầu nghiệp vụ.

Trong nhiều ứng dụng thường có phần xử lý các công việc không liên quan đến vấn đề nghiệp vụ như truy cập file, hạ tầng mạng, CSDL,... trong đối tượng nghiệp vụ kinh doanh. Cách này giúp tốc độ hoàn thiện ứng dụng nhanh. Tuy nhiên, cách này làm cho thiết kế bị mất đi tính hướng đối tượng trong thực tế với mức độ doanh nghiệp lớn. Đây là lý do thiết kế hướng miền trở nên quan trọng.

Trong kiến trúc vi dịch vụ, thiết kế hướng miền giúp đảm bảo rằng mỗi dịch vụ được thiết kế phản ánh một phần cụ thể của lĩnh vực kinh doanh. Mỗi dịch vụ được quản lí bởi một nhóm nhỏ được hỗ trợ bởi các chuyên gia ngành.  

Chuyên gia ngành  (Domain Expert)    là người có kiến thức và hiểu biết sâu sắc về miền kinh doanh hoặc lĩnh vực vấn đề đang được hệ thống phần mềm giải quyết.
Chuyên gia ngành    thể hiện chính xác     vấn đề   kinh doanh,    đóng vai trò là nguồn thông tin   cho nhóm phát triển.
  
  
  
  
  

 