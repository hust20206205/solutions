Ngôn ngữ chung này được các nhóm làm việc trong bối cảnh giới hạn ở hạ lưu chấp nhận. Ngôn ngữ chung này được gọi là ngôn ngữ được xuất bản và mẫu này được gọi là mẫu ngôn ngữ được xuất bản.

%! D - OHS|PL

Ngôn ngữ thứ hai là ngôn ngữ được xuất bản, đi đôi với dịch vụ lưu trữ mở. Trở lại ngược dòng, các liên hệ được giới hạn trên nhà cung cấp dịch vụ được lưu trữ mở sẽ hiển thị ngôn ngữ chung cho các dịch vụ chung và ngôn ngữ này được quản lý bởi nhóm chịu trách nhiệm về dịch vụ được lưu trữ mở, các liên hệ được giới hạn ở hạ nguồn ngoại trừ ngôn ngữ được xuất bản này.

%! Hướng dẫn 6/6

, bối cảnh giới hạn ngược dòng hoặc nhà cung cấp dịch vụ được lưu trữ mở trong mối quan hệ này cung cấp một ngôn ngữ chung để tích hợp.

Ngôn ngữ chung này được các nhóm làm việc trong bối cảnh giới hạn ở hạ lưu chấp nhận.

Ngôn ngữ chung này được gọi là ngôn ngữ được xuất bản và mẫu này được gọi là mẫu ngôn ngữ được xuất bản.

% Việc sử dụng ngôn ngữ đã xuất bản được mô tả bằng cách đặt PL ở phía trước thượng nguồn cùng với OHS.

Vì vậy, ánh xạ ngữ cảnh này mô tả một kịch bản trong đó C đang cung cấp các dịch vụ chung. Tức là nó có nhà cung cấp Oireachtas và cũng có ngôn ngữ được xuất bản.

9

00: 02: 02, 070 00: 02: 15, 580

Hãy gọi nó là một ví dụ về việc thực hiện các dịch vụ phổ biến. Chức năng này được hiển thị theo cách có bối cảnh giới hạn và được chấp nhận tốt bởi bối cảnh giới hạn phía dưới.

10

00: 02: 15, 660 00: 02: 30, 840

Vì vậy, hãy nghĩ đến một nhóm tạo ra một dịch vụ giúp bộc lộ những khoảng trống. Và giả sử nhóm này cũng đang xuất bản các lược đồ và mô hình API còn lại thông qua các thông số kỹ thuật API mở hoặc thông số kỹ thuật Swagga.

11

00: 02: 30, 840 00: 02: 44, 220

Các nhóm khác có thể xây dựng dịch vụ của riêng họ bằng cách sử dụng các công thức này và họ sẽ xây dựng các công thức này bằng cách tham khảo các thông số AP mở cho ứng dụng và lược đồ.

12

00: 02: 44, 220 00: 02: 52, 220

Và nhìn vào sơ đồ này, chúng ta có thể nói rằng chúng ta đã tạo API chính xác theo cách này và chúng ta hoàn toàn đúng.

% %! Published Language : https:// thiết kế hướng miền - practitioners.com/published - language

% %! Published Language : https:// thiết kế hướng miền - practitioners.com/published - language

% %! Published Language : https:// thiết kế hướng miền - practitioners.com/published - language

% %! Published Language : https:// thiết kế hướng miền - practitioners.com/published - language

Trang chủTrang chủBảng chú giảiBối cảnh giới hạn Mối quan hệ bối cảnh giới hạn Ngôn ngữ xuất bản

Ngôn ngữ xuất bản

Ngôn ngữ được xuất bản đề cập đến một ngôn ngữ được chia sẻ và phổ biến được tất cả các bên liên quan trong một lĩnh vực cụ thể đồng ý. Đó là một cách giao tiếp về miền một cách nhất quán và chính xác, sử dụng từ vựng được mọi người hiểu rõ và đồng ý. Ngôn ngữ được xuất bản thường được ghi lại trong bảng chú giải thuật ngữ hoặc tài liệu tham khảo khác và được sử dụng để mô tả các khái niệm miền, quy tắc nghiệp vụ và các khía cạnh khác của mô hình miền. Mục tiêu của việc sử dụng ngôn ngữ được xuất bản là để đảm bảo rằng mọi người tham gia vào dự án đều có sự hiểu biết rõ ràng và nhất quán về lĩnh vực đó, điều này có thể giúp giảm thiểu hiểu lầm và tăng khả năng thành công của dự án.

Hàm ý

Khái niệm về ngôn ngữ được xuất bản trong thiết kế hướng miền có thể có cả ưu điểm và nhược điểm, tùy thuộc vào bối cảnh cụ thể. Dưới đây là một số ưu và nhược điểm:

Ưu điểm

Hiểu biết được chia sẻ: Ngôn ngữ được xuất bản giúp tạo ra sự hiểu biết chung về các khái niệm miền và mối quan hệ của chúng giữa các thành viên trong nhóm, các bên liên quan và các bối cảnh giới hạn khác.

Tính nhất quán: Bằng cách sử dụng một ngôn ngữ đã được xuất bản, việc duy trì tính nhất quán giữa các phần khác nhau của hệ thống tương tác với cùng một khái niệm miền sẽ trở nên dễ dàng hơn.

Tính mô - đun: Việc sử dụng ngôn ngữ được xuất bản sẽ thúc đẩy tính mô - đun, vì nó cho phép các nhóm phát triển các thành phần độc lập của hệ thống bằng cách sử dụng ngôn ngữ chung phù hợp với miền.

Phát triển theo hướng miền : Ngôn ngữ được xuất bản là thành phần chính của phát triển theo hướng miền, có thể dẫn đến thiết kế hệ thống và giải quyết vấn đề hiệu quả hơn.

Nhược điểm

Độ phức tạp: Một ngôn ngữ được xuất bản có thể tăng thêm độ phức tạp cho hệ thống bằng cách đưa ra những khái niệm trừu tượng và miền mới mà các thành viên trong nhóm cần hiểu.

Chi phí chung: Việc tạo và duy trì ngôn ngữ đã xuất bản có thể tốn thời gian và đòi hỏi nỗ lực bổ sung từ các thành viên trong nhóm.

Áp dụng: Có thể mất thời gian để các thành viên trong nhóm tiếp thu và thành thạo ngôn ngữ đã xuất bản, đặc biệt nếu nó phức tạp hoặc xa lạ.

Giải thích sai: Việc sử dụng ngôn ngữ đã được xuất bản đôi khi có thể dẫn đến hiểu sai hoặc truyền đạt sai nếu các thành viên trong nhóm có cách hiểu khác nhau về các khái niệm lĩnh vực hoặc mối quan hệ của chúng.

Ví dụ

Dưới đây là ví dụ về cách sử dụng ngôn ngữ đã xuất bản trong hệ thống thương mại điện tử đơn giản hóa:

Giả sử chúng ta có hai bối cảnh giới hạn : “Quản lý đơn hàng” và “Quản lý hàng tồn kho”. Trong bối cảnh Quản lý đơn hàng, nhóm xử lý đơn hàng nói về đơn hàng, sản phẩm và khách hàng, trong khi ở bối cảnh Quản lý hàng tồn kho, nhóm kho hàng nói về hàng tồn kho, sản phẩm và nhà cung cấp.

Để đảm bảo rằng cả hai nhóm giao tiếp hiệu quả, họ đồng ý sử dụng một ngôn ngữ chung đã được xuất bản để xác định ý nghĩa của các thuật ngữ chính, chẳng hạn như “sản phẩm” và “nhà cung cấp”. Họ thiết lập một bảng thuật ngữ chung mô tả các thuật ngữ và cách sử dụng chúng trong ngữ cảnh của mỗi nhóm.

Ví dụ: bối cảnh Quản lý đơn hàng có thể định nghĩa “sản phẩm” là “một mặt hàng mà khách hàng có thể mua”, trong khi bối cảnh Quản lý hàng tồn kho có thể định nghĩa nó là “một mặt hàng được lưu trữ trong kho”. Các nhóm đồng ý với các định nghĩa này và sử dụng chúng một cách nhất quán khi giao tiếp giữa các bối cảnh.

Bằng cách sử dụng ngôn ngữ đã được xuất bản, các nhóm có thể tránh nhầm lẫn và hiểu lầm khi thảo luận về các khái niệm có thể có ý nghĩa khác nhau trong các bối cảnh khác nhau. Điều này có thể giúp cải thiện khả năng giao tiếp và cộng tác giữa các nhóm và đảm bảo rằng họ đang làm việc hướng tới cùng một mục tiêu.

Khi nào nó hoạt động?

Việc sử dụng ngôn ngữ đã xuất bản sẽ có hiệu quả khi có sự hiểu biết và thống nhất chung về ý nghĩa của các thuật ngữ và khái niệm được sử dụng trong lĩnh vực đó. Ngôn ngữ dùng chung này có thể giúp thúc đẩy giao tiếp và cộng tác tốt hơn giữa các thành viên trong nhóm và giữa các nhóm, dẫn đến việc phát triển và phân phối hệ thống phần mềm hiệu quả và hiệu quả hơn.

Hơn nữa, việc có một ngôn ngữ được xuất bản có thể giúp tránh những hiểu lầm, giảm sai sót và làm lại, đồng thời cung cấp nền tảng cho tài liệu rõ ràng và nhất quán. Nó cũng có thể giúp điều chỉnh các giải pháp kỹ thuật phù hợp với nhu cầu và yêu cầu kinh doanh, đồng thời tạo điều kiện chia sẻ và chuyển giao kiến thức giữa các thành viên trong nhóm.

Sử dụng ngôn ngữ đã xuất bản sẽ hiệu quả nhất khi có mức độ phức tạp và biến đổi cao trong miền cũng như khi có nhiều nhóm hoặc các bên liên quan tham gia vào quá trình phát triển hệ thống.

Khi nào nó không hoạt động?

Việc sử dụng ngôn ngữ đã xuất bản có thể không hiệu quả khi:

Các bên liên quan có quan điểm và cách giải thích ngôn ngữ khác nhau, điều này có thể dẫn đến hiểu lầm và nhầm lẫn.

Ngôn ngữ quá cứng nhắc và thiếu linh hoạt, gây khó khăn cho việc thích ứng với những nhu cầu và yêu cầu kinh doanh đang thay đổi.

Ngôn ngữ quá trừu tượng hoặc mang tính kỹ thuật, khiến các bên liên quan không chuyên về kỹ thuật khó hiểu và tham gia vào quá trình lập mô hình miền.

Có sự phản đối hoặc miễn cưỡng giữa các bên liên quan trong việc chấp nhận và sử dụng ngôn ngữ, điều này có thể cản trở sự thành công của nỗ lực lập mô hình miền.

% %! Published Language : https:// thiết kế hướng miền - practitioners.com/published - language

% %! Published Language : https:// thiết kế hướng miền - practitioners.com/published - language