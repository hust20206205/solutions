%!<! - - Strategic Design : https:// thiết kế hướng miền - practitioners.com/strategic - design - - >
%!<! - - [[Strategic Design]] Modeling and design decisions that apply to large parts of the system. Such decisions affect the entire project and have to be decided at team level. - - >

Trong thiết kế hướng miền, thiết kế chiến lược đề cập đến quá trình phân tích và xác định thiết kế tổng thể của hệ thống phần mềm hoặc ứng dụng, bao gồm kiến trúc, miền phụ, mối quan hệ và bản đồ ngữ cảnh của nó. Thiết kế chiến lược liên quan đến việc xác định các miền phụ quan trọng và bối cảnh giới hạn của hệ thống, hiểu mối quan hệ của chúng và xác định cách chúng nên được tổ chức và tích hợp để đạt được các mục tiêu và mục đích kinh doanh của hệ thống. Mục tiêu của thiết kế chiến lược là tạo ra một kiến trúc phần mềm gắn kết và có thể mở rộng, phù hợp với các mục tiêu kinh doanh và cung cấp nền tảng vững chắc cho sự phát triển và tăng trưởng trong tương lai.

%
kiến trúc phân lớp
miền phụ
mối quan hệ và bản đồ
modun

các mục bên dưới \dots

\begin{itemize}
\item Muc1
\item Muc2
\end{itemize}

% xem ảnh 2 cái

%!<! - - @Các mô hình chiến lược (Strategic Patterns) - - >

%!<! - - @Các mô hình chiến lược (Strategic Patterns) - - >

![](pictures/CacMoHinhChienLuoc/0_CacMoHinhChienLuoc.png)

%!<! - - Sơ đồ về các mô hình chiến lược - - >
%!<! - - $ Vẽ lại sau: - - >
%!<! - - Bối cảnh giới hạn (Bounded Context) - - >
%!<! - - [Giữ cho mô hình thống nhất] Tích hợp Liên tục (Continuous Integration) - - >
%!<! - - [Tính nhất quán trong trao đổi] Ngôn ngữ chung (Ubiquitous Language) - - >

%!<! - - [Tổng quan mối quan hệ] Bản đồ bối cảnh (Context Maps) - - >

%!<! - - Symmetric Relationship: Separate ways, Shared Kernel - - >
%!<! - - Asymmetric Relationship: Customer - Supplier, Conformist, Anti Corruption Layer - - >
%!<! - - - - >

%!<! - - One - to - Many Relationship: Open Host Service, Published Language - - >

%!<! - - dịch và cách ly đơn phương với - - >
%!<! - - [lớp] lớp (Context Maps) - - >
%!<! - - "Bản đồ bối cảnh dịch chuyển và cách ly một cách đơn phương để tạo thành cấu trúc lớp." - - >
%!<! - - Tách biệt - - >

%!<! - - $ Vẽ lại sau: - - >
%!<! - - $ Vẽ lại sau: - - >
%!<! - - $ Vẽ lại sau: - - >
%!<! - - $ Vẽ lại sau: - - >
%!<! - - $ Vẽ lại sau: - - >
%!<! - - $ Vẽ lại sau: - - >
%!<! - - $ Vẽ lại sau: - - >
%!<! - - $ Vẽ lại sau: - - >
%!<! - - $ Vẽ lại sau: - - >
%!<! - - $ Vẽ lại sau: - - >
%!<! - - $ Vẽ lại sau: - - >
%!<! - - $ Vẽ lại sau: - - >
%!<! - - $ Vẽ lại sau: - - >