% tên gọi chung thượng nguồn, hạ lưu

Trong mối quan hệ bất đối xứng, một bối cảnh giới hạn có sự phụ thuộc vào một bối cảnh giới hạn khác. Mối quan hệ này được mô tả bằng cách gán vai trò cho bối cảnh giới hạn:

Bối cảnh giới hạn thượng nguồn (Upstream): bối cảnh giới hạn cung cấp cho bối cảnh giới hạn khác.
Bối cảnh giới hạn hạ lưu (Downstream): bối cảnh giới hạn phụ thuộc vào bối cảnh giới hạn khác.

%! !ký hiệu: D - U - - >
%! $VD: - - >
%! $VD: A Downstream (D) - B Upstream (U) - - >
%! $VD: Bối cảnh A ràng buộc với bối cảnh B thì: - - >
%! $VD: Bối cảnh A đóng vai trò là bối cảnh giới hạn hạ lưu (Downstream) - - >
%! $VD: Bối cảnh B đóng vai trò là bối cảnh giới hạn thượng nguồn (Upstream) - - >
%! $VD: Bối cảnh giới hạn A có kiến thức về các mô hình trong bối cảnh giới hạn B - - >
%! $VD: Bối cảnh B không có bất kỳ kiến thức nào về mô hình trong bối cảnh giới hạn A - - >

