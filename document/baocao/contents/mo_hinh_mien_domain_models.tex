%!<! - - @ - - >

Để tạo một phần mềm tốt, chúng ta cần phải hiểu rõ về phần mềm đó. Trong thiết kế hướng miền để có thể hiểu miền nhanh, chúng ta tạo ra các mô hình miền.

Mô hình miền là kiến thức có tổ chức và có cấu trúc về miền phù hợp để giải quyết vấn đề kinh doanh.

Mô hình miền không phải là kiến thức của chuyên gia ngành, mà là sự trừu tượng hóa của cả nhóm.

Trong quá trình phát triển, nhóm trao đổi và thảo luận về mô hình của nhóm.

Mô hình miền giúp nhóm hiểu công việc và đồng thuận khi làm việc.

%!<! - - $VD: Ở đồ án này, mô hình miền của em bao gồm yêu câu nghiệp vụ và các sơ đồ: UML Use Case Diagrams, UML Activity Diagrams, UML Sequence Diagrams, UML Class Diagrams - - >

% %!<! - - Business Model Canvas : https:// thiết kế hướng miền - practitioners.com/business - value - canvas - - >

% %!<! - - có thể nêu thêm thôi - - >

% %!<! - - Domain Model: https:// thiết kế hướng miền - practitioners.com/home/glossary/domain - model - - >

% %!<! - - Domain Model: https:// thiết kế hướng miền - practitioners.com/home/glossary/domain - model - - >

% %!<! - - Domain Model: https:// thiết kế hướng miền - practitioners.com/home/glossary/domain - model - - >

% %!<! - - Domain Model: https:// thiết kế hướng miền - practitioners.com/home/glossary/domain - model - - >

% %!<! - - Domain Model: https:// thiết kế hướng miền - practitioners.com/home/glossary/domain - model - - >

% %!<! - - Domain Model: https:// thiết kế hướng miền - practitioners.com/home/glossary/domain - model - - >

% %!<! - - Domain Model: https:// thiết kế hướng miền - practitioners.com/home/glossary/domain - model - - >

%!<! - - [[Model]] A system of abstractions that describes selected aspects of a domain and can be used to solve problems related to that domain. - - >

Trang chủTrang chủBảng chú giảiMô hình miền

Mô hình miền

Trong Thiết kế hướng miền (thiết kế hướng miền), mô hình miền là sự thể hiện của miền có vấn đề dưới dạng mô hình phần mềm. Nó là một tập hợp các đối tượng miền và mối quan hệ giữa chúng, đồng thời được sử dụng để mô hình hóa logic nghiệp vụ và hành vi của miền.

Mô hình miền là trọng tâm của thiết kế hướng miền và đóng vai trò là công cụ giao tiếp giữa các chuyên gia ngành và nhà phát triển phần mềm. Nó cung cấp sự hiểu biết chung về miền và các khái niệm của nó, đồng thời giúp đảm bảo rằng hệ thống phần mềm phản ánh chính xác các yêu cầu và ràng buộc kinh doanh.

Mô hình miền được xây dựng bằng cách sử dụng kết hợp các kỹ thuật hướng đối tượng, chẳng hạn như các lớp và đối tượng, cũng như các kỹ thuật dành riêng cho miền, chẳng hạn như các thực thể, đối tượng giá trị và dịch vụ miền. Mô hình này được cải tiến nhiều lần khi dự án tiến triển và phát triển để phản ánh chính xác nhu cầu thay đổi của doanh nghiệp.

Mục tiêu cuối cùng của mô hình miền là cung cấp sự trình bày rõ ràng, ngắn gọn và chính xác về miền vấn đề và làm cơ sở để triển khai hệ thống phần mềm giải quyết các vấn đề kinh doanh.

thiết kế hướng miền có giới hạn với Lập trình hướng đối tượng không?

Không, thiết kế hướng miền không gắn liền với lập trình hướng đối tượng. thiết kế hướng miền cung cấp một tập hợp các nguyên tắc và mẫu để thiết kế hệ thống phần mềm có thể áp dụng cho các mô hình lập trình khác nhau, bao gồm hướng đối tượng, chức năng và thủ tục. Mặc dù OOP là cách phổ biến để triển khai thiết kế hướng miền, nhưng các khái niệm này cũng có thể được áp dụng trong các mô hình lập trình khác như lập trình chức năng hoặc lập trình thủ tục. Điều quan trọng là nắm bắt chính xác các khái niệm miền chứ không chỉ sử dụng một mô hình lập trình cụ thể.

Một ví dụ

Hãy tưởng tượng một doanh nghiệp bán lẻ bán sản phẩm cho khách hàng. Mô hình miền cho doanh nghiệp này có thể bao gồm các khái niệm sau:

Sản phẩm : Đại diện cho sản phẩm mà doanh nghiệp bán, chẳng hạn như áo phông hoặc một đôi giày. Các thuộc tính có thể bao gồm tên, mô tả, giá và tình trạng còn hàng của sản phẩm.

Khách hàng : Đại diện cho khách hàng thực hiện mua hàng từ doanh nghiệp. Các thuộc tính có thể bao gồm tên, địa chỉ, email và lịch sử mua hàng của khách hàng.

Đặt hàng : Thể hiện đơn đặt hàng mà khách hàng đặt cho một hoặc nhiều sản phẩm. Các thuộc tính có thể bao gồm ngày đặt hàng, các sản phẩm có trong đơn đặt hàng và tổng giá của đơn đặt hàng.

Thanh toán : Thể hiện khoản thanh toán được thực hiện bởi khách hàng cho một đơn đặt hàng. Các thuộc tính có thể bao gồm ngày thanh toán, phương thức thanh toán (chẳng hạn như thẻ tín dụng hoặc PayPal) và số tiền đã thanh toán.

Hàng tồn kho : Thể hiện lượng sản phẩm tồn kho của doanh nghiệp. Các thuộc tính có thể bao gồm số lượng từng sản phẩm trong kho và số lượng tối thiểu mỗi sản phẩm cần được giữ trong kho.

Những khái niệm này tạo thành một mô hình miền đơn giản nắm bắt các khái niệm cốt lõi của một doanh nghiệp bán lẻ. Mô hình miền có thể được sử dụng để triển khai logic nghiệp vụ và lưu trữ dữ liệu cho doanh nghiệp bán lẻ.

Nhiều mô hình miền

Có thể có nhiều mô hình miền cùng tồn tại trong một tổ chức hoặc hệ thống. Mỗi mô hình miền tập trung vào một lĩnh vực cụ thể của doanh nghiệp và thể hiện các khái niệm, quy tắc và mối quan hệ của nó theo một cách riêng biệt. Các mô hình miền khác nhau có thể tương tác với nhau nhưng chúng duy trì các ranh giới tự chủ và nhất quán của riêng mình. Điều này cho phép mô hình hóa các hệ thống phức tạp với nhiều lĩnh vực kinh doanh, mỗi lĩnh vực có những yêu cầu và quan điểm riêng.

% %!<! - - Domain Model: https:// thiết kế hướng miền - practitioners.com/home/glossary/domain - model - - >

% %!<! - - Domain Model: https:// thiết kế hướng miền - practitioners.com/home/glossary/domain - model - - >

% %!<! - - Domain Model: https:// thiết kế hướng miền - practitioners.com/home/glossary/domain - model - - >

% %!<! - - Domain Model: https:// thiết kế hướng miền - practitioners.com/home/glossary/domain - model - - >

% %!<! - - Domain Model: https:// thiết kế hướng miền - practitioners.com/home/glossary/domain - model - - >

% %!<! - - Domain Model: https:// thiết kế hướng miền - practitioners.com/home/glossary/domain - model - - >

% %!<! - - Domain Model: https:// thiết kế hướng miền - practitioners.com/home/glossary/domain - model - - >

% %!<! - - Domain Model: https:// thiết kế hướng miền - practitioners.com/home/glossary/domain - model - - >

