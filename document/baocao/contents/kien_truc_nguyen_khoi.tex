Trước khi kiến trúc vi dịch vụ trở nên phổ biến, kiến trúc nguyên khối đã được áp dụng rộng rãi trong kiến trúc phần mềm truyền thống. Kiến trúc nguyên khối là kiến trúc phần mềm trong đó  tất cả các thành phần      của  dự án    được xây dựng thành một đơn vị triển khai    duy nhất. 

% Trong kiến trúc nguyên khối, bất kỳ thay đổi nào đối với một thành phần     đều yêu cầu toàn bộ ứng dụng phải được     kiểm thử   và triển khai lại. 




Điều này có thể dẫn đến chu kỳ phát triển và triển khai chậm hơn cũng như thiếu khả năng mở rộng vì ứng dụng có thể không đáp ứng được nhu cầu về chức năng hoặc lưu lượng truy cập ngày càng tăng. Tuy nhiên, kiến trúc nguyên khối thường thiết kế, phát triển và bảo trì đơn giản hơn so với các kiến trúc phân tán, khiến chúng trở thành lựa chọn phổ biến cho các ứng dụng quy mô nhỏ hơn hoặc các nhóm có nguồn lực hạn chế.


% 
Ví dụ: Mô hình MVC (Model - View - Controller) là một trong những dạng của kiến trúc nguyên khối.

Trong mô hình này, ứng dụng được chia thành ba thành phần chính:

Mô hình (Model): Đại diện cho dữ liệu và logic xử lý dữ liệu.

Giao diện (View): Đại diện cho giao diện người dùng.

Bộ điều khiển (Controller): Nhận yêu cầu người dùng thông qua View, sau đó tương tác với Model để làm việc với dữ liệu.

%  
 
 
 


