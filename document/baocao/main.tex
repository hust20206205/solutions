\documentclass{article}
\usepackage[utf8]{vietnam}
\usepackage{graphicx}
\begin{document}
%%%%%%%%%%%%%%%%%%%%%%%%%%%%%%%%%%%%%
% \section{Trang bìa}
% \begin{titlepage}

\begin{tikzpicture}[remember picture, overlay]\draw [line width = 3pt]($ (current page.north west) + (3.0cm, - 2.5cm)$)rectangle($ (current page.south east) + (- 2.5cm, 2.5cm)$);\draw [line width = 0.5pt]($ (current page.north west) + (3.1cm, - 2.6cm)$)rectangle($ (current page.south east) + (- 2.6cm, 2.6cm)$);\end{tikzpicture}

\begin{center}

\vspace{- 0.4cm}

\textbf{ĐẠI HỌC BÁCH KHOA HÀ NỘI} \\

\textbf{VIỆN TOÁN ỨNG DỤNG VÀ TIN HỌC} \\

\textbf{******}

\vspace{0.8cm}

\begin{figure}[H]

\centering

\includegraphics[scale = .5]{pictures/hust/logoBK.png}

\end{figure}

\vspace{0.7cm}

\textbf{\fontsize{16pt}{30pt}\selectfont {BÁO CÁO ĐỒ ÁN II}} \\

\textbf{\fontsize{10pt}{24pt}\selectfont {CHUYÊN NGÀNH: TOÁN TIN}}

\vspace{1cm}

\textbf{\fontsize{16pt}{30pt}\selectfont {ĐỀ TÀI:}} \\

% \textbf{\fontsize{19pt}{24pt}\selectfont {Xây dựng kiến trúc vi dịch vụ cho \\ bài toán hóa đơn điện tử}} \\

\textbf{\fontsize{20pt}{24pt}\selectfont {Sử dụng thiết kế hướng miền \\ xây dựng kiến trúc vi dịch vụ cho \\ bài toán hóa đơn điện tử}} \\

\end{center}

\vspace{0.7cm}

\hspace{2.6cm}\begin{minipage}{0.8\textwidth}
\textbf{\fontsize{10pt}{24pt}\selectfont {Giảng viên hướng dẫn: TS. Vũ Thành Nam}}
\end{minipage}

\vspace{0.7cm}

\hspace{3cm}\begin{minipage}{0.7\textwidth}

\begin{tabular}{l l l}

\textbf{\fontsize{10pt}{24pt}\selectfont {   Sinh viên thực hiện}} & \textbf{\fontsize{10pt}{24pt}\selectfont {   Vũ Văn Nghĩa }}      \\
\textbf{\fontsize{10pt}{24pt}\selectfont { Mã số sinh viên}}       & \textbf{\fontsize{10pt}{24pt}\selectfont {   20206205 }}          \\
\textbf{\fontsize{10pt}{24pt}\selectfont { Lớp}}                   & \textbf{\fontsize{10pt}{24pt}\selectfont {   Toán Tin 02 - K65 }} \\
\end{tabular}

\end{minipage}

\vspace{0.5cm}

\begin{center}

\textbf{Hà Nội, \the\month~/~\the\year}

\end{center}

\end{titlepage}


% \newpage
% \thispagestyle{empty}

% \section{Nhận xét của giảng viên}
% \newpage

\begin{center}

{\bfseries NHẬN XÉT CỦA GIẢNG VIÊN HƯỚNG DẪN}

\end{center}

\begin{enumerate}

\item Mục đích và nội dung của đồ án:

\vspace{20ex} % Thêm khoảng cách dọc

\item 	Kết quả đạt được:

\vspace{20ex} % Thêm khoảng cách dọc

\item 	Ý thức làm việc của sinh viên:

\vspace{20ex} % Thêm khoảng cách dọc

\end{enumerate}

\hspace{0.4\textwidth}\begin{minipage}{0.5\textwidth}

\noindent\begin{center}

\textit{Hà Nội, \today} \\

\textbf{Giảng viên hướng dẫn} \\

\textit{(Ký và ghi rõ họ tên)}

\vspace{2cm}

\textbf{TS. Vũ Thành Nam}

\end{center}

\end{minipage}

\pagestyle{empty}

\newpage



% \section{Mục lục}
% \tableofcontents

% \section{Lời cảm ơn}
% \newpage

\section*{\centering LỜI CẢM ƠN}

\addcontentsline{toc}{section}{LỜI CẢM ƠN}

\lipsum[1 - 3] % Tạo ba đoạn văn bản giả mạo

\newpage



% \section{Lời mở đầu}
% \newpage

\section*{\centering LỜI MỞ ĐẦU}

\addcontentsline{toc}{section}{LỜI MỞ ĐẦU}
\lipsum[1-3] % Tạo ba đoạn văn bản giả mạo

\newpage

% \section{Tóm tắt nội dung đồ án}
% \newpage

\section*{\centering TÓM TẮT NỘI DUNG ĐỒ ÁN}

\addcontentsline{toc}{section}{TÓM TẮT NỘI DUNG ĐỒ ÁN}
\lipsum[1-3] % Tạo ba đoạn văn bản giả mạo

\newpage

% \section{Đánh giá và thảo luận}
% \newpage
\section*{\centering ĐÁNH GIÁ VÀ THẢO LUẬN}
\addcontentsline{toc}{section}{ĐÁNH GIÁ VÀ THẢO LUẬN}
\newpage

% \section{Danh sách bảng}
% \section{Danh sách hình ảnh}
% \section{Danh sách mã nguồn}

% \section{Danh sách các cụm từ viết tắt}
% \input{contents/danh_sach_cac_cum_tu_viet_tat}

% \section{Danh sách các thuật ngữ}
% % STT; Tiếng Anh; Tiếng Việt
% @sau

% kiến trúc nguyên khối, kiến trúc nguyên khối
% kiến trúc nguyên khối, kiến trúc nguyên khối
% kiến trúc vi dịch, kiến trúc vi dịch
% kiến trúc vi dịch, kiến trúc vi dịch
% kiến trúc vi dịch, kiến trúc vi dịch
% kiến trúc vi dịch, kiến trúc vi dịch
% thiết kế hướng miền, thiết kế hướng miền
% thiết kế hướng miền, thiết kế hướng miền

1 thiết kế hướng miền
Thiết kế hướng lĩnh vực
2 Domain (không dịch)
3 Abstraction Trừu tượng
4 chuyên gia ngành

%%%%%%%%%%%
% \section{Giới thiệu}
% Trong thời đại ngày nay, nhu cầu phát triển ứng dụng và hệ thống ngày càng tăng, đặt ra thách thức đối với kiến trúc phần mềm. Kiến trúc nguyên khối đã phục vụ hiệu quả trong quá khứ, nhưng kiến trúc này bắt đầu gặp khó khăn đối mặt với sự phức tạp, khả năng mở rộng và khả năng đáp ứng linh hoạt với thay đổi nhanh chóng trong yêu cầu kinh doanh.

Kiến trúc vi dịch vụ là giải pháp cho những thách thức này. Kiến trúc vi dịch vụ chia dự án thành những dịch vụ nhỏ độc lập, mỗi dịch vụ chịu trách nhiệm về một chức năng cụ thể. Từ đó, giảm sự phức tạp của dự án tăng tính linh hoạt và dễ dàng quản lý.

Việc vận dụng kết hợp giữa kiến trúc vi dịch vụ và thiết kế hướng miền là một cách tiếp cận toàn diện, giúp xác định và tổ chức các dịch vụ dựa trên việc hiểu rõ về lĩnh vực kinh doanh. Thiết kế hướng miền giúp xây dựng mô hình dựa trên yêu cầu nghiệp vụ thực tế, giúp dự án phản ánh đúng các quy trình kinh doanh.

% \subsection{Giới thiệu về bài toán hóa đơn điện tử}
% \input{contents/gioi_thieu_ve_bai_toan_hoa_don_dien_tu}

% \subsubsection{Hóa đơn}
% Theo quy định tại khoản 1 Điều 3 Nghị định 123/2020/NĐ - CP:

Hóa đơn là chứng từ kế toán do tổ chức, cá nhân bán hàng hóa, cung cấp dịch vụ lập, ghi nhận thông tin bán hàng hóa, cung cấp dịch vụ. Hóa đơn được thể hiện theo hình thức hóa đơn điện tử hoặc hóa đơn do cơ quan thuế đặt in.

% \subsubsection{Hóa đơn điện tử}
% \input{contents/hoa_don_dien_tu}

% \subsubsection{Bắt buộc sử dụng hóa đơn điện tử từ 01/07/2022}
% Nghị định này có hiệu lực thi hành kể từ ngày 01 tháng 7 năm 2022, khuyến khích cơ quan, tổ chức, cá nhân đáp ứng điều kiện về hạ tầng công nghệ thông tin áp dụng quy định về hóa đơn, chứng từ điện tử của Nghị định này trước ngày 01 tháng 7 năm 2022.

= >

Theo quy định của Chính phủ và Bộ Tài Chính, tất cả các doanh nghiệp, tổ chức và hộ kinh doanh đều bắt buộc phải chuyển từ sử dụng hóa đơn giấy sang hóa đơn điện tử kể từ tháng 07/2022.

Vì vậy, nhu cầu sử dụng và xử lý hóa đơn điện tử trở nên rất lớn. Do đó, em đã chọn chủ đề: \textbf{\textit{"Xây dựng kiến trúc vi dịch vụ cho hệ thống quản lý hóa đơn điện tử"}}



% \subsubsection{Bản thể hiện của hóa đơn điện tử}
% \input{contents/ban_the_hien_cua_hoa_don_dien_tu}

% \subsubsection{Lưu trữ hóa đơn điện tử}
% Thời gian?
% \input{contents/luu_tru_hoa_don_dien_tu}

% \subsubsection{Một số lợi ích của hóa đơn điện tử}
% Giúp tiết kiệm chi phí in ấn, lưu trữ và bảo quản.
Loại bỏ rủi ro cháy, hỏng hoặc mất và dễ dàng sao lưu.
Dễ dàng linh hoạt trong việc tra cứu, phát hành, quản lý và tạo báo cáo.
Tối ưu hóa quá trình kế toán (giảm sai sót và tiết kiệm thời gian) và giảm thủ tục giấy tờ.
Theo dõi tình hình tài chính của công ty (doanh thu, chi phí, lợi nhuận).
Tuân thủ các quy định về thuế và pháp luật.
Thể hiện tính minh bạch trong quá trình kinh doanh (bảo vệ quyền lợi của cả người mua và người bán).

% \subsection{Giới thiệu về kiến trúc vi dịch vụ}
% \subsubsection{Kiến trúc nguyên khối}
% Trước khi kiến trúc vi dịch vụ trở nên phổ biến, kiến trúc nguyên khối đã được áp dụng rộng rãi trong kiến trúc phần mềm truyền thống. Kiến trúc nguyên khối là kiến trúc phần mềm trong đó  tất cả các thành phần      của  dự án    được xây dựng thành một đơn vị triển khai    duy nhất. 

% Trong kiến trúc nguyên khối, bất kỳ thay đổi nào đối với một thành phần     đều yêu cầu toàn bộ ứng dụng phải được     kiểm thử   và triển khai lại. 




Điều này có thể dẫn đến chu kỳ phát triển và triển khai chậm hơn cũng như thiếu khả năng mở rộng vì ứng dụng có thể không đáp ứng được nhu cầu về chức năng hoặc lưu lượng truy cập ngày càng tăng. Tuy nhiên, kiến trúc nguyên khối thường thiết kế, phát triển và bảo trì đơn giản hơn so với các kiến trúc phân tán, khiến chúng trở thành lựa chọn phổ biến cho các ứng dụng quy mô nhỏ hơn hoặc các nhóm có nguồn lực hạn chế.


% 
Ví dụ: Mô hình MVC (Model - View - Controller) là một trong những dạng của kiến trúc nguyên khối.

Trong mô hình này, ứng dụng được chia thành ba thành phần chính:

Mô hình (Model): Đại diện cho dữ liệu và logic xử lý dữ liệu.

Giao diện (View): Đại diện cho giao diện người dùng.

Bộ điều khiển (Controller): Nhận yêu cầu người dùng thông qua View, sau đó tương tác với Model để làm việc với dữ liệu.

%  
 
 
 




% \subsubsection{Kiến trúc vi dịch vụ}
% Kiến trúc vi dịch vụ chia dự án thành các thành phần nhỏ hơn được gọi là các dịch vụ.

Các dịch vụ chịu trách nhiệm cho một chức năng cụ thể nhằm hiện thực hóa khả năng kinh doanh cụ thể.

Các dịch vụ độc lập về ngôn ngữ lập trình, CSDL, triển khai,...

Các dịch vụ tương tác với nhau qua hạ tầng mạng.

% \begin{figure}[h]

% \centering

% \includegraphics[height = 3cm]{pictures/ChuyenTu_KienTrucNguyenKhoi_Sang_KienTrucViDichVu.jpg}

% % \caption{ViDuHinhAnhTheoChieuDoc}

% \end{figure}

% \begin{figure}[h]

% \centering

% \includegraphics[height = 3cm]{pictures/AnhKhacNhau_KienTrucNguyenKhoi_KienTrucViDichVu.png}

% % \caption{ViDuHinhAnhTheoChieuDoc}

% \end{figure}

% Strangler Fig là chuyển mono sang dịch vụ

Kiến trúc kiến trúc vi dịch vụ (hay đơn giản là kiến trúc vi dịch vụ) là một cách xây dựng các ứng dụng phần mềm dưới dạng tập hợp các dịch vụ nhỏ, độc lập giao tiếp với nhau thông qua API. Mỗi dịch vụ tập trung vào một khả năng kinh doanh cụ thể và có thể được triển khai, mở rộng quy mô và duy trì độc lập với các dịch vụ khác trong hệ thống. Cách tiếp cận này nhấn mạnh tính mô - đun, tính linh hoạt và khả năng phục hồi, cho phép các nhóm làm việc đồng thời trên các phần khác nhau của hệ thống và cho phép phát hành nhanh hơn và thường xuyên hơn. Các vi dịch vụ thường dựa vào các giao thức truyền thông nhẹ, chẳng hạn như REST và thường được triển khai bằng các công nghệ chứa trong bộ chứa như Docker và Kubernetes.


% \subsubsection{Một số đặc điểm và ưu điểm của kiến trúc vi dịch vụ}
% Kiến trúc vi dịch vụ có nhiều ưu điểm, đặc biệt với các dự án có quy mô lớn và phức tạp.
% Mô - đun hóa : Kiến trúc nên bao gồm một tập hợp các dịch vụ được kết hợp lỏng lẻo có thể được triển khai, duy trì và mở rộng quy mô một cách độc lập.
Kiến trúc vi dịch vụ phân chia dự án thành các dịch vụ nhỏ.
% Nhìn chung, điều đó có nghĩa là I.T. các nhóm không cần phải đi sâu vào mọi khả năng kinh doanh. Họ có thể tập trung vào năng lực kinh doanh mà họ đang xây dựng trong vi dịch vụ của mình.
% Nhưng giả sử hoạt động kinh doanh cho vay và thế chấp đang trải qua một sự chuyển đổi nghiêm trọng nào đó, trong trường hợp đó, nhóm cho vay và thế chấp có thể quyết định phát hành vi dịch vụ của họ mỗi ngày.

%%%%%%%%%%%%%%%%%%%%%%%%%%%%%%%%%%%%%

% và có thể được triển khai, mở rộng quy mô và duy trì độc lập với các dịch vụ khác trong hệ thống.

% Các dịch vụ độc lập về ngôn ngữ lập trình, CSDL, triển khai,...

% Các dịch vụ tương tác với nhau qua hạ tầng mạng.

% Mô - đun hóa : Kiến trúc nên bao gồm một tập hợp các dịch vụ được kết hợp lỏng lẻo có thể được triển khai, duy trì và mở rộng quy mô một cách độc lập.

% Triển khai độc lập : Mỗi dịch vụ phải có khả năng triển khai độc lập để có thể thực hiện các thay đổi đối với một dịch vụ mà không ảnh hưởng đến các dịch vụ khác.

% Bối cảnh giới hạn : Kiến trúc phải được thiết kế để phù hợp với ranh giới của khả năng kinh doanh, sao cho mỗi dịch vụ chịu trách nhiệm về một tập hợp chức năng cụ thể, gắn kết.

% Tự chủ : Các dịch vụ phải có khả năng hoạt động tự chủ, ít phụ thuộc vào các dịch vụ khác.

% Khả năng phục hồi : Kiến trúc phải được thiết kế để chịu đựng lỗi và các dịch vụ phải có khả năng xử lý lỗi một cách duyên dáng.

% Khả năng mở rộng : Kiến trúc phải được thiết kế để hỗ trợ mở rộng quy mô các dịch vụ riêng lẻ cũng như toàn bộ hệ thống.

% Tính linh hoạt : Kiến trúc phải cho phép phát triển và triển khai nhanh chóng các dịch vụ mới cũng như khả năng thay đổi các dịch vụ hiện có một cách nhanh chóng và dễ dàng.

% Văn hóa DevOps : Kiến trúc phải được hỗ trợ bởi văn hóa nhấn mạnh sự cộng tác và giao tiếp giữa các nhóm phát triển và vận hành, cũng như tự động hóa các quy trình triển khai và thử nghiệm.


Kiến trúc vi dịch vụ phân chia dự án thành các dịch vụ nhỏ.

Giúp việc phát triển và quản lý dễ dàng hơn.

Dễ dàng mở rộng hệ thống.

Tận dụng sử dụng tài nguyên cho từng dịch vụ.

Tập trung yêu cầu nghiệp vụ trong dịch vụ dẫn đến tốc độ định giá doanh nghiệp nhanh hơn.

Vì các dịch vụ được phân chia là độc lập.

% Nhìn chung, điều đó có nghĩa là I.T. các nhóm không cần phải đi sâu vào mọi khả năng kinh doanh. Họ có thể tập trung vào năng lực kinh doanh mà họ đang xây dựng trong vi dịch vụ của mình.

% Nhưng giả sử hoạt động kinh doanh cho vay và thế chấp đang trải qua một sự chuyển đổi nghiêm trọng nào đó, trong trường hợp đó, nhóm cho vay và thế chấp có thể quyết định phát hành vi dịch vụ của họ mỗi ngày.

Các nhóm phát triển riêng dẫn tới tốc độ phát triển thay đổi nhanh.

Giảm thiểu ràng buộc và tăng tính linh hoạt của hệ thống.

Giảm chi phí và thời gian kiểm thử do ít ràng buộc.

Hệ thống có khả năng chịu lỗi cao tăng độ tin cậy.

Kiến trúc vi dịch vụ sử dụng đa ngôn ngữ và công nghệ khác nhau.

Tận dụng hiệu quả thế mạnh của từng ngôn ngữ, công nghệ phù hợp nhất cho yêu cầu nghiệp vụ cụ thể.

% Ví dụ: Mỗi dịch vụ sử dụng ngôn ngữ lập trình nhau khác như: NodeJS, Go, Python, Java, CSharp,...

% \begin{figure}[h]

% \centering

% \includegraphics[height = 3cm]{pictures/DaNgonNgu/_DaNgonNgu.png}

% % \caption{ViDuHinhAnhTheoChieuDoc}

% Thêm vào hình SQL riêng

% \end{figure}



% \subsubsection{Một số nhược điểm và thách thức của kiến trúc vi dịch vụ}
% Tuy nhiên, kiến trúc vi dịch vụ cũng có nhiều thách thức.

Chịu ảnh hưởng của đường truyền mạng.

Khả năng kiểm soát giao dịch (transaction).

Tính nhất quán và toàn vẹn của dữ liệu giữa các dịch vụ.

Giám sát giữa các dịch vụ.

Bảo mật giao tiếp giữa các dịch vụ.

Phát hiện lỗi và sửa lỗi khó khăn.

Ràng buộc về thứ tự sự kiện.

Đồng bộ đồng hồ thời gian.

Chi phí xây dựng, quản lí vận hành lớn.



% \subsubsection{Có thể thêm phần truyền thông trực tiếp, gián tiếp}

% \subsection{Giới thiệu về thiết kế hướng miền}
% % <! - - Lý do tiếp theo là một trong những lý do lớn nhất khiến tổ chức cần chuyển đổi nhu cầu và mong đợi của khách hàng liên tục thay đổi để duy trì và mở rộng cơ sở khách hàng của mình. - - >

% <! - - Các tổ chức cần điều chỉnh hoạt động kinh doanh của mình để đáp ứng nhu cầu và mong đợi của khách hàng. Các doanh nghiệp bỏ qua kỳ vọng của khách hàng có xu hướng thua đối thủ cạnh tranh. - - >

% <! - - Vì vậy, điều xảy ra với những doanh nghiệp không chuyển đổi, câu trả lời ngắn gọn cho câu hỏi này là những doanh nghiệp không chuyển đổi sẽ không thể tồn tại. - - >

% <! - - Một điểm quan trọng cần ghi nhớ là chuyển đổi không phải là sáng kiến hay nhiệm vụ chỉ diễn ra một lần. Các doanh nghiệp cần thay đổi liên tục và điều này đòi hỏi những thay đổi nhanh chóng đối với hệ thống và ứng dụng của họ. - - >

% <! - - Một thách thức chung mà các doanh nghiệp phải đối mặt trong hành trình chuyển đổi là cách xây dựng phần mềm cũ cản trở hoặc gây khó khăn cho các tổ chức trong việc chuyển đổi. - - >

Trong quá trình hoạt động, không phải mọi doanh nghiệp đều sẽ giữ nguyên mô hình kinh doanh được đưa ra ban đầu của mình. Khi quy mô thị trường thay đổi thì việc chuyển đổi mô hình kinh doanh là điều cần thiết. Chuyển đổi kinh doanh như một công cụ linh hoạt giúp các doanh nghiệp có thể phát triển và tồn tại giữa các đối thủ của mình.

Ví dụ:

Amazon từ hiệu sách trực tuyến thành thị trường cho nhà cung cấp khác như: Thương mại điện tử (E - commerce), Dịch vụ đám mây (Cloud Computing),...

<! - - !Thêm google - - >

% ![](pictures/KienTrucViDichVuAmazon.png)

% ![](pictures/KienTrucViDichVuAmazon.png)

\begin{figure}[h]

\centering

\includegraphics[height = 3cm]{pictures/KienTrucViDichVuAmazon.png}

% \caption{ViDuHinhAnhTheoChieuDoc}

\end{figure}

\begin{figure}[h]

\centering

\includegraphics[height = 3cm]{pictures/KienTrucViDichVuAmazon.png}

% \caption{ViDuHinhAnhTheoChieuDoc}

\end{figure}

<! - - Hình kiến trúc vi dịch vụ của Amazon - - >

<! - - !Thêm google - - >

Gần đây, Baemin dịch vụ giao đồ ăn đã rời khỏi thị trường Việt Nam cũng do sức ép từ các đối thủ khác khiến Baemin khó cạnh tranh trong mảng kinh doanh cốt lõi là giao đồ ăn. Các đối thủ này không chỉ cung cấp dịch vụ giao đồ ăn mà còn có đặt xe, giao hàng,...

![](pictures/Baemin.png)

<! - - Hình Baemin đã rời khỏi thị trường Việt Nam - - >

= > Hiện nay, các tổ chức doanh nghiệp có nhu cầu phát triển chuyển đổi kinh doanh để có thể tồn tại và phát triển khi thị trường thay đổi. Từ đó, đáp ứng nhu cầu của khách hàng, giúp mang đến ưu thế cạnh tranh so với các đối thủ. Do đó cần hệ thống chuyển đổi nhanh chóng đáp ứng nhu cầu của dự án và mong đợi của khách hàng.

= > Kiến trúc vi dịch vụ giải quyết những thách thức và hỗ trợ doanh nghiệp chuyển đổi dễ dàng.

Tuy nhiên, để xây dựng được kiến trúc vi dịch vụ tốt cần phải tạo ra các dịch vụ nhỏ phù hợp và duy trì tính độc lập. Nếu không thực hiện đúng các nhóm phụ thuộc lẫn nhau và mất đi lợi thế của kiến trúc vi dịch vụ.

Và từ đó, mẫu thiết kế hướng miền sử dụng để phân tích xây dựng kiến trúc vi dịch vụ.

Thiết kế hướng miền xác định và tổ chức các dịch vụ dựa trên việc hiểu rõ về lĩnh vực kinh doanh, giúp dự án phản ánh đúng các quy trình và quy tắc kinh doanh.


% thiết kế hướng miền có thể trợ giúp như thế nào

% Thiết kế hướng miền (thiết kế hướng miền) có thể hỗ trợ kiến trúc và thiết kế kiến trúc vi dịch vụ theo nhiều cách:

% Bối cảnh giới hạn : thiết kế hướng miền nhấn mạnh việc xác định các bối cảnh giới hạn, là các khu vực riêng biệt của miền có ranh giới được xác định rõ ràng. Điều này có thể giúp xác định ranh giới của kiến trúc vi dịch vụ và đảm bảo rằng mỗi kiến trúc vi dịch vụ đều có trách nhiệm rõ ràng và tập trung.

% ngôn ngữ chung : thiết kế hướng miền khuyến khích sử dụng một ngôn ngữ chung được chia sẻ bởi cả chuyên gia ngành và nhân viên kỹ thuật. Điều này có thể giúp đảm bảo rằng các vi dịch vụ giao tiếp hiệu quả với nhau và phù hợp với yêu cầu kinh doanh.

% Ánh xạ bối cảnh: thiết kế hướng miền cung cấp các kỹ thuật để ánh xạ các mối quan hệ giữa các bối cảnh giới hạn. Điều này có thể giúp đảm bảo rằng các vi dịch vụ được thiết kế với sự hiểu biết rõ ràng về sự phụ thuộc của chúng vào các dịch vụ khác.

% Tập hợp: thiết kế hướng miền định nghĩa tập hợp là các cụm đối tượng liên quan cần được coi là một đơn vị nhất quán duy nhất. Điều này có thể giúp đảm bảo rằng các vi dịch vụ được thiết kế với sự hiểu biết rõ ràng về các yêu cầu về tính nhất quán của dữ liệu.

% Tương quan với bối cảnh giới hạn

% Mặc dù người ta thường khuyên nên căn chỉnh ranh giới của kiến trúc vi dịch vụ với ranh giới của bối cảnh giới hạn, nhưng điều đó không phải lúc nào cũng cần thiết hoặc khả thi. Một vi dịch vụ có thể gói gọn nhiều ngữ cảnh giới hạn hoặc một ngữ cảnh giới hạn có thể được phân chia thành nhiều vi dịch vụ, tùy thuộc vào nhu cầu cụ thể của hệ thống và sự cân bằng liên quan. Cuối cùng, mục tiêu là tạo ra một hệ thống mô - đun và có thể bảo trì, đáp ứng các yêu cầu kinh doanh và mối quan hệ giữa bối cảnh giới hạn và các vi dịch vụ phải được thiết kế phù hợp.

% Tương quan với API thực thể

% Nhìn chung, việc thiết kế các vi dịch vụ chỉ dựa trên các hoạt động CRUD của thực thể không được khuyến khích vì nó có thể dẫn đến một kiến trúc liên kết chặt chẽ và không hiệu quả. Các vi dịch vụ phải được thiết kế dựa trên khả năng kinh doanh, có thể phù hợp hoặc không phù hợp với hoạt động CRUD của thực thể.

% thiết kế hướng miền có thể giúp xác định các khả năng kinh doanh và xác định bối cảnh giới hạn, sau đó có thể được sử dụng để hướng dẫn thiết kế các vi dịch vụ. Bằng cách tập trung vào khả năng kinh doanh thay vì hoạt động CRUD thực thể, các vi dịch vụ có thể được liên kết lỏng lẻo hơn, mang tính mô - đun hơn, dễ dàng duy trì và phát triển hơn theo thời gian.



%%%%%%%%%%%
% \section{Yêu cầu nghiệp vụ}
% Yêu cầu nghiệp vụ xác định nội dung, phạm vi, mục tiêu và chức năng mong muốn của hệ thống.

% thêm là yếu tố quan trọng trong  thiết kế hướng miền vì hướng đến nghiệp vụ kinh doanh
% Nguồn: TCT

% \subsection {Yêu cầu nghiệp vụ của bài toán phụ}
% \input{contents/yeu_cau_nghiep_vu_cua_bai_toan_phu}

\subsubsection{Các chức năng tổng quan của bài toán phụ}




Miền phụ chung cung cấp các giải pháp có sẵn mà doanh nghiệp có thể mua.

Doanh nghiệp không thể đạt được bất kỳ lợi thế cạnh tranh nào bằng cách thực hiện những điều khác biệt trong miền phụ chung.

% $????? VD: Các miền phụ chung như các hoạt động quản lý nhân sự và quản lý cơ sở vật chất không tạo thêm bất kỳ giá trị khác biệt nào cho doanh nghiệp. - - >


% % %!<! - - Generic Subdomain : https:// thiết kế hướng miền - practitioners.com/generic - subdomain - - >

% % %!<! - - Generic Subdomain : https:// thiết kế hướng miền - practitioners.com/generic - subdomain - - >

% Trang chủTrang chủBảng chú giảiLãnh địa Miền phụ chung

% Miền phụ chung

% Trong Thiết kế hướng miền (thiết kế hướng miền), miền phụ chung là loại miền phụ không có bất kỳ đặc điểm cụ thể hoặc duy nhất nào so với các miền khác trong cùng lĩnh vực. Đó là một miền phụ có thể được tìm thấy trên nhiều ngành, thay vì dành riêng cho một ngành hoặc miền.

% Mặc dù các miền phụ chung có thể không phải là duy nhất hoặc dành riêng cho một miền nhưng chúng vẫn cần được xác định rõ ràng và hiểu rõ để triển khai hiệu quả trong hệ thống.

% Ví dụ

% Xác thực và ủy quyền: Miền phụ này xử lý việc quản lý danh tính người dùng và quyền truy cập vào tài nguyên trong hệ thống. Thông thường, cần có một giải pháp chung cho miền phụ này để có thể sử dụng lại trên nhiều hệ thống.

% Thông báo : Miền phụ này xử lý việc gửi thông báo cho người dùng, chẳng hạn như thông báo qua email hoặc SMS. Tương tự như xác thực và ủy quyền, việc có một giải pháp chung cho miền phụ này có thể được sử dụng lại trên nhiều hệ thống thường rất hữu ích.

% Thanh toán : Miền phụ này xử lý các khoản thanh toán, bao gồm thu thập thông tin thanh toán, tính phí thẻ tín dụng và xử lý tiền hoàn lại. Tương tự như các ví dụ trên, giải pháp thanh toán chung có thể được sử dụng lại trên nhiều hệ thống.

% Định vị địa lý : Miền phụ này xử lý việc ánh xạ các vị trí thực tế tới các biểu diễn kỹ thuật số. Một giải pháp định vị địa lý chung có thể được sử dụng trong nhiều hệ thống, chẳng hạn như ánh xạ địa chỉ tới tọa độ GPS hoặc tính toán khoảng cách giữa các vị trí.

% % %!<! - - Generic Subdomain : https:// thiết kế hướng miền - practitioners.com/generic - subdomain - - >

% % %!<! - - Generic Subdomain : https:// thiết kế hướng miền - practitioners.com/generic - subdomain - - >

% % %!<! - - Generic Subdomain : https:// thiết kế hướng miền - practitioners.com/generic - subdomain - - >

% \subsection{Yêu cầu nghiệp vụ của bài toán chính}
% Hai vấn đề cốt lõi được giải quyết bằng thiết kế hướng miền:
% 1. Thiết kế, phân chia cơ cấu doanh nghiệp như thế nào cho hợp lý?
% 2. Kiến trúc kỹ thuật có phù hợp với kiến trúc kinh doanh không?
% %
% Yêu cầu nghiệp vụ của từng sub
% %
% Sơ đồ if else Đ S
% %
% sub trước model
% %
%%%%%%%%%%%%%%%%%%%%%%%%%%%%%%%%%%%%%
\end{document}

\section{xxxxxxx}
\subsection{xxxxxxx}
\subsubsection{xxxxxxx}




Miền phụ chung cung cấp các giải pháp có sẵn mà doanh nghiệp có thể mua.

Doanh nghiệp không thể đạt được bất kỳ lợi thế cạnh tranh nào bằng cách thực hiện những điều khác biệt trong miền phụ chung.

% $????? VD: Các miền phụ chung như các hoạt động quản lý nhân sự và quản lý cơ sở vật chất không tạo thêm bất kỳ giá trị khác biệt nào cho doanh nghiệp. - - >


% % %!<! - - Generic Subdomain : https:// thiết kế hướng miền - practitioners.com/generic - subdomain - - >

% % %!<! - - Generic Subdomain : https:// thiết kế hướng miền - practitioners.com/generic - subdomain - - >

% Trang chủTrang chủBảng chú giảiLãnh địa Miền phụ chung

% Miền phụ chung

% Trong Thiết kế hướng miền (thiết kế hướng miền), miền phụ chung là loại miền phụ không có bất kỳ đặc điểm cụ thể hoặc duy nhất nào so với các miền khác trong cùng lĩnh vực. Đó là một miền phụ có thể được tìm thấy trên nhiều ngành, thay vì dành riêng cho một ngành hoặc miền.

% Mặc dù các miền phụ chung có thể không phải là duy nhất hoặc dành riêng cho một miền nhưng chúng vẫn cần được xác định rõ ràng và hiểu rõ để triển khai hiệu quả trong hệ thống.

% Ví dụ

% Xác thực và ủy quyền: Miền phụ này xử lý việc quản lý danh tính người dùng và quyền truy cập vào tài nguyên trong hệ thống. Thông thường, cần có một giải pháp chung cho miền phụ này để có thể sử dụng lại trên nhiều hệ thống.

% Thông báo : Miền phụ này xử lý việc gửi thông báo cho người dùng, chẳng hạn như thông báo qua email hoặc SMS. Tương tự như xác thực và ủy quyền, việc có một giải pháp chung cho miền phụ này có thể được sử dụng lại trên nhiều hệ thống thường rất hữu ích.

% Thanh toán : Miền phụ này xử lý các khoản thanh toán, bao gồm thu thập thông tin thanh toán, tính phí thẻ tín dụng và xử lý tiền hoàn lại. Tương tự như các ví dụ trên, giải pháp thanh toán chung có thể được sử dụng lại trên nhiều hệ thống.

% Định vị địa lý : Miền phụ này xử lý việc ánh xạ các vị trí thực tế tới các biểu diễn kỹ thuật số. Một giải pháp định vị địa lý chung có thể được sử dụng trong nhiều hệ thống, chẳng hạn như ánh xạ địa chỉ tới tọa độ GPS hoặc tính toán khoảng cách giữa các vị trí.

% % %!<! - - Generic Subdomain : https:// thiết kế hướng miền - practitioners.com/generic - subdomain - - >

% % %!<! - - Generic Subdomain : https:// thiết kế hướng miền - practitioners.com/generic - subdomain - - >

% % %!<! - - Generic Subdomain : https:// thiết kế hướng miền - practitioners.com/generic - subdomain - - >

% phải có CQRS (Phân chia trách nhiệm truy vấn lệnh)

CQRS là một mẫu kiến trúc riêng biệt có thể được sử dụng kết hợp với thiết kế hướng miền để đạt được những lợi ích nhất định, chẳng hạn như cải thiện hiệu suất và khả năng mở rộng. Tuy nhiên, nó không phải là một yêu cầu để triển khai thiết kế hướng miền.

% phải có event
Ngôn ngữ chung (Ubiquitous Language)