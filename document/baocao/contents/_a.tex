Miền phụ cốt lõi là điểm khác biệt quan trọng cho doanh nghiệp.
Miền phụ cốt lõi tập trung vào mục tiêu và yêu cầu  của   khách hàng của doanh nghiệp.
Miền phụ cốt lõi  giúp phân biệt các doanh nghiệp và làm cho các doanh nghiệp có giá trị.
Miền phụ cốt lõi    quyết định đối với sự thành công của   doanh nghiệp   vì vậy  mỗi doanh nghiệp trong một ngành cụ thể hoạt động khác nhau trong các miền phụ cốt lõi để đạt được một số lợi thế so với đối thủ cạnh tranh.



$\Rightarrow$ Doanh nghiệp luôn tìm cách thực hiện những điều khác biệt trong các miền phụ cốt lõi này để có được một số lợi thế cạnh tranh.

Ví dụ: Trong miền thẻ tín dụng, miền phụ cốt lõi có thể là \textit{"phát hành thẻ"} chịu trách nhiệm về quá trình phát hành thẻ tín dụng cho khách hàng. Miền phụ cốt lõi này bao gồm các nhiệm vụ như: thu thập thông tin khách hàng, thực hiện kiểm tra tín dụng, kích hoạt thẻ, \dots

% Lưu ý về các miền phụ
Lưu ý về các miền phụ: Các miền phụ   cốt lõi, hỗ trợ và chung  có thể khác nhau     đối với các doanh nghiệp       hoạt động trong cùng một miền. 
Vì các miền phụ  được  xác định tùy  theo nhu cầu kinh doanh và bối cảnh cụ thể của mỗi tổ chức. 



