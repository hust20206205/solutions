\begin{itemize}

\item Trong quá trình xây dựng mô hình miền, cần có trao đổi giữa người thiết kế hệ thống và chuyên gia ngành để hiểu đúng về miền. Tuy nhiên, nhóm kinh doanh sử dụng ngôn ngữ kinh doanh và nhóm công nghệ có xu hướng sử dụng các thuật ngữ kỹ thuật trong giao tiếp của họ. Người phát triển phần mềm tập trung vào lớp, phương thức, thuật toán, trong khi chuyên gia ngành thường sử dụng ngôn ngữ chuyên ngành của họ. Sự khác biệt về ngôn ngữ giữa các thành viên có thể dẫn đến những thách thức về giao tiếp.

\item Ngoài ra, trong các lĩnh vực kinh doanh khác nhau, một thuật ngữ có thể được sử dụng trong nhiều miền, cùng với ý nghĩa khác nhau gây ra sự nhầm lẫn và hiểu sai cho các người phát triển phần mềm cũng như các chuyên gia ngành.

$\Rightarrow$ Thiết kế hướng miền đề xuất sử dụng ngôn ngữ chung để giải quyết những thách thức ngôn ngữ.

\item Ngôn ngữ chung (Ubiquitous Language)    là một ngôn ngữ được cấu trúc xung quanh mô hình miền và được tất cả các thành viên trong nhóm sử dụng cho  mọi hoạt động của nhóm với phần mềm. 


\item Ngôn ngữ chung được xác định bởi các  từ vựng và có định nghĩa rõ ràng về ngữ cảnh sử dụng từ vựng.







\end{itemize}
\subsubsection{ Một số đặc điểm   của ngôn ngữ chung }


\begin{itemize} 



    \item Ngôn ngữ chung được sử dụng bởi cả chuyên gia ngành và chuyên gia công nghệ.
    
    \item Có nhiều ngôn ngữ chung trong một tổ chức được mỗi nhóm tạo và quản lý một cách độc lập.
    
    \item Việc tạo ra ngôn ngữ chung là một quá trình liên tục. Ngôn ngữ chung phát triển theo thời gian thông qua sự cộng tác giữa doanh nghiệp và các chuyên gia công nghệ.
    
    \item Các thành viên phải sử dụng ngôn ngữ chung cho công việc và trong toàn bộ hệ thống như:
    
    \item Sử dụng trong cuộc thảo luận trao đổi giữa các chuyên gia ngành và các chuyên gia công nghệ
    
    \item Sử dụng trong các tài liệu phát triển của nhóm
    
    \item Sử dụng trong sản phẩm và kiểm thử phần mềm
    
     \end{itemize} 
    
    
\end{document}

 

Ngôn ngữ chung được sử dụng bởi cả chuyên gia ngành và chuyên gia công nghệ.

Có nhiều ngôn ngữ chung trong một tổ chức được mỗi nhóm tạo và quản lý một cách độc lập.

Việc tạo ra ngôn ngữ chung là một quá trình liên tục. Ngôn ngữ chung phát triển theo thời gian thông qua sự cộng tác giữa doanh nghiệp và các chuyên gia công nghệ.

Các thành viên phải sử dụng ngôn ngữ chung cho công việc và trong toàn bộ hệ thống như:

Sử dụng trong cuộc thảo luận trao đổi giữa các chuyên gia ngành và các chuyên gia công nghệ

Sử dụng trong các tài liệu phát triển của nhóm

Sử dụng trong sản phẩm và kiểm thử phần mềm

%!<! - - $VD: Ngôn ngữ chung được sử dụng, áp dụng trong toàn bộ hệ thống. - - >

![](pictures/NgonNguChung/___NgonNguPhoBien.png)

%!<! - - Hướng dẫn 5/7 - - >

% %!<! - - Ubiquitous Language : https:// thiết kế hướng miền - practitioners.com/home/glossary/ubiquitous - language - - >

Trong thiết kế hướng miền (thiết kế hướng miền), ngôn ngữ chung là ngôn ngữ dùng chung được sử dụng để mô tả miền và mô hình miền. Đó là ngôn ngữ chung được các nhà phát triển, các bên liên quan và chuyên gia ngành sử dụng để giao tiếp và hiểu miền kinh doanh.

Ý tưởng đằng sau ngôn ngữ chung là tạo ra sự hiểu biết chung về miền bằng cách sử dụng một ngôn ngữ chung để mô tả các khái niệm và mối quan hệ trong miền. Ngôn ngữ chung này nên được sử dụng nhất quán trong suốt quá trình phát triển phần mềm, từ giai đoạn phân tích và thiết kế đến giai đoạn triển khai và bảo trì.

ngôn ngữ chung phải dựa trên ngôn ngữ đã được sử dụng trong miền và nên được sử dụng để đặt tên cho các lớp, giao diện, phương thức và biến trong mã. Điều này giúp đảm bảo rằng mã có tính biểu cảm, dễ hiểu và phù hợp với nhu cầu kinh doanh.

Mục tiêu của ngôn ngữ chung là đảm bảo rằng phần mềm phù hợp với nhu cầu kinh doanh và miền kinh doanh được thể hiện chính xác trong phần mềm. Nó cũng giúp cải thiện khả năng giao tiếp và cộng tác giữa các nhóm khác nhau và đảm bảo rằng phần mềm có thể bảo trì và thích ứng với sự thay đổi.

Điều quan trọng cần lưu ý là ngôn ngữ chung không chỉ giới hạn ở cách đặt tên mã mà còn là một cách để đảm bảo rằng các khái niệm và mối quan hệ kinh doanh được thể hiện chính xác trong phần mềm và phần mềm phù hợp với nhu cầu kinh doanh.

Dưới đây là một số ví dụ về cách áp dụng ngôn ngữ chung trong thiết kế hướng miền:

Quy ước đặt tên : Tên của các lớp, giao diện, phương thức và biến trong mã phải dựa trên ngôn ngữ đã được sử dụng trong miền. Ví dụ: trong miền thương mại điện tử, khái niệm “đặt hàng” phải được gọi một cách nhất quán là “đặt hàng” trong toàn bộ mã, thay vì sử dụng các tên khác nhau như “mua hàng” hoặc “giao dịch”.

Mô hình hóa quy trình nghiệp vụ : Các quy trình nghiệp vụ trong miền phải được mô hình hóa bằng cách sử dụng cùng các thuật ngữ và khái niệm đã được sử dụng trong miền. Ví dụ: trong lĩnh vực chăm sóc sức khỏe, quy trình “tiếp nhận bệnh nhân” phải được đề cập đến việc sử dụng các thuật ngữ và khái niệm tương tự được các chuyên gia chăm sóc sức khỏe sử dụng, chẳng hạn như “quy trình tiếp nhận” hoặc “quy trình đăng ký bệnh nhân”

Mô hình hóa các thực thể kinh doanh : Các thực thể kinh doanh trong miền phải được mô hình hóa bằng cách sử dụng cùng các thuật ngữ và khái niệm đã được sử dụng trong miền. Ví dụ: trong lĩnh vực tài chính, khái niệm “tài khoản” phải được gọi một cách nhất quán là “tài khoản” trong toàn bộ mã, thay vì sử dụng các tên khác nhau như “sổ cái” hoặc “hồ sơ tài chính”

Giao tiếp : ngôn ngữ chung nên được sử dụng trong mọi giao tiếp, bao gồm các cuộc họp, tài liệu và nhận xét mã, để đảm bảo rằng mọi người đều sử dụng các thuật ngữ và khái niệm giống nhau.

