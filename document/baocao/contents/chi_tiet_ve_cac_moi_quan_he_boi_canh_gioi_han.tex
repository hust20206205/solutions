Có 3 loại mối quan hệ bối cảnh giới hạn là:
- xxxxxxxxxxxxxxx
- xxxxxxxxxxxxxxx
- xxxxxxxxxxxxxxx


%  Để phát triển tốt 
% cần tạo một bộ kiểm thử tích hợp tự động
CI/CD đã trình bày bên trên
%  nhằm kiểm tra tính đúng đắn  

% <!-- Test-Driven Development : https://ddd-practitioners.com/test-driven-development -->
% <!-- Test-Driven Development : https://ddd-practitioners.com/test-driven-development -->
% <!-- Test-Driven Development : https://ddd-practitioners.com/test-driven-development -->
% <!-- Test-Driven Development : https://ddd-practitioners.com/test-driven-development -->
% <!-- Test-Driven Development : https://ddd-practitioners.com/test-driven-development -->
[[Test-Driven Development]] TDD is a lightweight programming methodology that emphasizes fast, incremental development and especially writing tests before writing code. Ideally these follow one another in cycles measured in minutes. (see full definition under [[Test-Driven Development]] topic)


Chuyển đến nội dung
Đối với người hành nghề bởi người hành nghề
Tìm kiếm
Thiết kế hướng miền: Hướng dẫn dành cho người thực hành
Câu hỏi thường gặp
Bảng chú giải
Về chúng tôi
Cuốn sách của chúng tôi!
Trang chủTrang chủBảng chú giảiHướng phát triển thử nghiệm
Hướng phát triển thử nghiệm
Phát triển dựa trên thử nghiệm (TDD) là một phương pháp phát triển phần mềm trong đó các thử nghiệm được viết trước khi mã thực tế được phát triển. Mục đích của TDD là đảm bảo rằng mỗi đoạn mã đều được kiểm tra đầy đủ và đáp ứng các yêu cầu của doanh nghiệp. TDD liên quan đến việc viết một bài kiểm tra thất bại trước tiên, viết mã vừa đủ để vượt qua, sau đó tái cấu trúc mã để cải thiện thiết kế của nó trong khi vẫn đảm bảo rằng tất cả các bài kiểm tra đều vượt qua.

TDD là một phương pháp quan trọng trong Thiết kế hướng miền (DDD) vì nó giúp đảm bảo rằng mã được phát triển phù hợp với mô hình miền và các quy tắc miền. Bằng cách viết bài kiểm tra trước, nhà phát triển buộc phải suy nghĩ về mô hình miền và các quy tắc miền trước khi viết bất kỳ mã nào. Các thử nghiệm trở thành một cách để xác định hành vi của hệ thống và giúp tập trung vào các yêu cầu kinh doanh. Khi mã được phát triển và tái cấu trúc, các thử nghiệm sẽ đảm bảo rằng hệ thống vẫn phù hợp với mô hình miền và các yêu cầu kinh doanh.


Thể loại

Phân tích
điều cơ bản
ddd
thiết kế
câu hỏi thường gặp
Khả năng lãnh đạo
hoa văn
Blog tại WordPress.com.

% <!-- Test-Driven Development : https://ddd-practitioners.com/test-driven-development -->
% <!-- Test-Driven Development : https://ddd-practitioners.com/test-driven-development -->
% <!-- Test-Driven Development : https://ddd-practitioners.com/test-driven-development -->
% <!-- Test-Driven Development : https://ddd-practitioners.com/test-driven-development -->
