% khi quan hệ riêng biệt, không có sự phụ thuộc lẫn nhau.

Các liên hệ trong bối cảnh giới hạn thực sự độc lập.

Các liên hệ không có mối quan hệ nào với các liên hệ khác.

Các liên hệ có mô hình độc lập và thực thi riêng biệt.

Các nhóm phát triển không phải cộng tác hay phối hợp cho bất kỳ nhiệm vụ nào.

% %! $VD: trong trường hợp ngân hàng, thẻ tín dụng và khoản vay mua nhà không có mối quan hệ nào. - - >

% %! Separate Ways : https:// thiết kế hướng miền - practitioners.com/separate - ways - - >

% %! Separate Ways : https:// thiết kế hướng miền - practitioners.com/separate - ways - - >

Trang chủTrang chủBảng chú giảiBối cảnh giới hạn Mối quan hệ bối cảnh giới hạn Đường riêng

Đường riêng

Loại mối quan hệ này đề cập đến tình huống trong đó hai hoặc nhiều bối cảnh giới hạn quá khác nhau về mục đích, ngôn ngữ hoặc lĩnh vực đến mức chúng không thể hoặc không nên được tích hợp hoặc liên kết với nhau. Nói cách khác, họ cần đi theo con đường riêng và độc lập với nhau. Quyết định này thường được đưa ra khi thấy rõ rằng lợi ích của việc cố gắng điều chỉnh hoặc tích hợp các bối cảnh giới hạn sẽ lớn hơn chi phí hoặc rủi ro liên quan. Trong những trường hợp như vậy, các nhóm chịu trách nhiệm về từng bối cảnh giới hạn nên tập trung vào miền, ngôn ngữ và mô hình của riêng họ, đồng thời tránh sự phụ thuộc hoặc giả định về các bối cảnh giới hạn khác.

Hàm ý

Những ưu và nhược điểm của việc đi theo những con đường riêng biệt bao gồm:

Ưu điểm

Cải thiện khả năng bảo trì: Các thành phần nhỏ hơn, tập trung hơn sẽ dễ bảo trì, sửa đổi và mở rộng quy mô hơn.

Ranh giới rõ ràng: Mỗi bối cảnh được giới hạn đều có ranh giới được xác định rõ ràng, giúp ngăn ngừa sự nhầm lẫn và xung đột giữa các yêu cầu.

Triển khai độc lập: Các bối cảnh giới hạn có thể được triển khai và thử nghiệm độc lập, điều này có thể tăng tốc chu kỳ phát triển và triển khai.

Khuyến khích thiết kế theo miền: Quá trình tạo bối cảnh giới hạn có thể giúp các nhóm tập trung vào các khái niệm miền cốt lõi và đảm bảo rằng mỗi thành phần đều phù hợp với mục tiêu kinh doanh tổng thể.

Nhược điểm

Độ phức tạp ngày càng tăng: Việc chia hệ thống thành nhiều bối cảnh có giới hạn có thể tạo thêm độ phức tạp vì các nhà phát triển phải quản lý sự tương tác và giao tiếp giữa các thành phần.

Khả năng xảy ra những nỗ lực trùng lặp: Các nhóm khác nhau có thể cần phát triển chức năng tương tự trong bối cảnh giới hạn tương ứng của họ, điều này có thể dẫn đến những nỗ lực trùng lặp và tăng chi phí bảo trì.

Thách thức về tích hợp: Khi các bối cảnh giới hạn phát triển và thay đổi theo thời gian, việc tích hợp chúng vào hệ thống lớn hơn có thể gặp khó khăn, đặc biệt nếu các tương tác giữa các thành phần phức tạp.

Tăng cường phối hợp: Sự phối hợp và giao tiếp hiệu quả giữa các nhóm là rất quan trọng để đảm bảo rằng mỗi bối cảnh giới hạn đều phù hợp với mục tiêu kinh doanh tổng thể và hoạt động liền mạch với các thành phần khác.

Đi theo những cách riêng biệt có thể là một cách tiếp cận hiệu quả để quản lý các hệ thống phức tạp, nhưng nó đòi hỏi phải lập kế hoạch và phối hợp cẩn thận để đảm bảo rằng mỗi bối cảnh giới hạn hoạt động tốt với những bối cảnh khác và phù hợp với các mục tiêu kinh doanh tổng thể.

% %! Separate Ways : https:// thiết kế hướng miền - practitioners.com/separate - ways - - >

% %! Separate Ways : https:// thiết kế hướng miền - practitioners.com/separate - ways - - >