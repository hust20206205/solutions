Trong quá trình hoạt động kinh doanh, không phải mọi doanh nghiệp đều giữ nguyên mô hình kinh doanh được đưa ra ban đầu của mình. Khi quy mô thị trường thay đổi, việc chuyển đổi mô hình kinh doanh là điều cần thiết. Chuyển đổi kinh doanh như một công cụ linh hoạt giúp các doanh nghiệp có thể phát triển và tồn tại giữa các đối thủ của mình.

Ví dụ:

\begin{itemize}

\item Google bắt đầu như một công cụ tìm kiếm trực tuyến, nhưng sau đó đã mở rộng và thay đổi mô hình kinh doanh qua nhiều dịch vụ và sản phẩm khác nhau như: Dịch vụ đám mây Google Cloud Platform, thư điện tử Gmail, bản đồ Google Maps, lưu trữ tập tin Google Drive, \dots

\item Amazon từ hiệu sách trực tuyến đã trở thành thị trường cho nhà cung cấp khác như: Thương mại điện tử, Dịch vụ đám mây Amazon Web Services (AWS), \dots

\begin{figure}[H]

\centering

\includegraphics[width = 0.5\textwidth]{pictures/KienTrucViDichVuAmazon/main.png}

\caption{Kiến trúc vi dịch vụ của Amazon}

\end{figure}

\end{itemize}

Đối với những doanh nghiệp không chuyển đổi kinh doanh sẽ không thể tồn tại.

Ví dụ: Gần đây, dịch vụ giao đồ ăn Baemin đã rời khỏi thị trường Việt Nam cũng do sức ép từ các đối thủ khác khiến Baemin khó cạnh tranh trong mảng kinh doanh cốt lõi là giao đồ ăn. Các đối thủ này không chỉ cung cấp dịch vụ giao đồ ăn mà còn có đặt xe, giao hàng,...

\begin{figure}[H]

\centering

\includegraphics[width = 0.5\textwidth]{pictures/Baemin/main.png}

\caption{Baemin đã rời khỏi thị trường Việt Nam}

\end{figure}

%%%%%%%%%%%%%%%%%%%%%%%%%%%%%%%%%%%%%

Hiện nay, các tổ chức doanh nghiệp có nhu cầu chuyển đổi kinh doanh để có thể tồn tại và phát triển khi thị trường thay đổi. Từ đó, đáp ứng nhu cầu của khách hàng, mang lại ưu thế cạnh tranh so với các đối thủ. Do đó,  các     doanh nghiệp cần hệ thống chuyển đổi nhanh chóng đáp ứng nhu cầu của     mô hình kinh doanh       và mong đợi của khách hàng.

$\Rightarrow$ Kiến trúc vi dịch vụ giải quyết những thách thức và hỗ trợ doanh nghiệp chuyển đổi dễ dàng.        Tuy nhiên, để xây dựng được kiến trúc vi dịch vụ tốt, cần phải tạo ra các dịch vụ nhỏ phù hợp và duy trì tính độc lập.      Ở đồ án này, em sử dụng thiết kế hướng miền để phân tích và xây dựng kiến trúc vi dịch vụ.  

Thiết kế hướng miền xác định và tổ chức các dịch vụ dựa trên việc hiểu rõ về lĩnh vực kinh doanh, giúp dự án phản ánh đúng các quy trình và quy tắc kinh doanh.

%%%%%%%%%%%%%%%%%%%%%%%%%%%%%%%%%%%%%

% Thiết kế hướng miền (thiết kế hướng miền) có thể hỗ trợ kiến trúc và thiết kế kiến trúc vi dịch vụ theo nhiều cách:

% Bối cảnh giới hạn : thiết kế hướng miền nhấn mạnh việc xác định các bối cảnh giới hạn, là các khu vực riêng biệt của miền có ranh giới được xác định rõ ràng. Điều này có thể giúp xác định ranh giới của kiến trúc vi dịch vụ và đảm bảo rằng mỗi kiến trúc vi dịch vụ đều có trách nhiệm rõ ràng và tập trung.

% ngôn ngữ chung : thiết kế hướng miền khuyến khích sử dụng một ngôn ngữ chung được chia sẻ bởi cả chuyên gia ngành và nhân viên kỹ thuật. Điều này có thể giúp đảm bảo rằng các vi dịch vụ giao tiếp hiệu quả với nhau và phù hợp với yêu cầu kinh doanh.

% Ánh xạ bối cảnh: thiết kế hướng miền cung cấp các kỹ thuật để ánh xạ các mối quan hệ giữa các bối cảnh giới hạn. Điều này có thể giúp đảm bảo rằng các vi dịch vụ được thiết kế với sự hiểu biết rõ ràng về sự phụ thuộc của chúng vào các dịch vụ khác.

% Tập hợp: thiết kế hướng miền định nghĩa tập hợp là các cụm đối tượng liên quan cần được coi là một đơn vị nhất quán duy nhất. Điều này có thể giúp đảm bảo rằng các vi dịch vụ được thiết kế với sự hiểu biết rõ ràng về các yêu cầu về tính nhất quán của dữ liệu.

% Tương quan với bối cảnh giới hạn

% Mặc dù người ta thường khuyên nên căn chỉnh ranh giới của kiến trúc vi dịch vụ với ranh giới của bối cảnh giới hạn, nhưng điều đó không phải lúc nào cũng cần thiết hoặc khả thi. Một vi dịch vụ có thể gói gọn nhiều ngữ cảnh giới hạn hoặc một ngữ cảnh giới hạn có thể được phân chia thành nhiều vi dịch vụ, tùy thuộc vào nhu cầu cụ thể của hệ thống và sự cân bằng liên quan. Cuối cùng, mục tiêu là tạo ra một hệ thống mô - đun và có thể bảo trì, đáp ứng các yêu cầu kinh doanh và mối quan hệ giữa bối cảnh giới hạn và các vi dịch vụ phải được thiết kế phù hợp.

