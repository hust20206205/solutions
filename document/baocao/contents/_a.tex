
<!--@Tích hợp Liên tục (Continuous Integration)-->

Tích hợp Liên tục (Continuous Integration): là việc các thành viên trong nhóm phát triển tích hợp mã nguồn vào một hệ thống chung thường xuyên. Khi có mã nguồn mới việc tích hợp liên tục sẽ tự động kiểm thử và xây dựng giảm xung đột giữa các phiên bản mã nguồn khác nhau, giúp phát hiện và sửa lỗi sớm hơn.
% CD là gì
% CD là gì
% CD là gì
% CD là gì
% CD là gì
% CD là gì
% CD là gì
% CD là gì
% CD là gì
CI/CD là viết tắt của hai khái niệm quan trọng trong quá trình phát triển phần mềm: Continuous Integration (CI) và Continuous Delivery (CD). Đây là một phương pháp giúp tự động hóa quy trình phát triển, kiểm thử, và triển khai ứng dụng, giúp tăng cường chất lượng phần mềm và giảm thời gian cũng như rủi ro trong quá trình phát triển.

1. **Continuous Integration (CI - Tích hợp liên tục):**
   - **Mục tiêu:** Đảm bảo rằng mã nguồn mới được tích hợp vào mã nguồn chính (main codebase) một cách tự động và thường xuyên, giảm thời gian giữa việc viết mã và phát hành.
   - **Quy trình:** Mỗi khi một nhà phát triển hoàn thành một tính năng hoặc sửa lỗi, họ tích hợp mã của mình vào mã nguồn chính. Hệ thống CI sẽ tự động kiểm tra mã này bằng cách chạy các bài kiểm thử tự động để đảm bảo rằng nó không làm hỏng hệ thống.

2. **Continuous Delivery (CD - Phân phối liên tục):**
   - **Mục tiêu:** Tự động hóa việc triển khai ứng dụng để có thể phân phối bản vá, tính năng hoặc cập nhật một cách nhanh chóng và đáng tin cậy.
   - **Quy trình:** Nếu quá trình CI thành công, mã nguồn sẽ được triển khai tự động lên môi trường thử nghiệm (staging environment). Nếu mọi thứ ổn, nó có thể được triển khai tự động lên môi trường sản phẩm (production environment).

3. **Continuous Deployment (CD - Triển khai liên tục):**
   - **Khác biệt với Continuous Delivery:** Trong Continuous Deployment, nếu mọi thứ qua bài kiểm thử được tự động và thành công, mã nguồn sẽ tự động triển khai lên môi trường sản phẩm mà không cần sự can thiệp thủ công.
   - **Mục tiêu:** Tối ưu hóa quá trình triển khai, giảm thiểu sự chờ đợi và đảm bảo tính ổn định của hệ thống.

4. **Các công cụ thường được sử dụng:**
   - **Jenkins, Travis CI, CircleCI:** Đối với CI.
   - **Docker, Kubernetes:** Đối với CD, đặc biệt là việc triển khai và quản lý containerized applications.
   - **Ansible, Puppet, Chef:** Công cụ tự động hóa cấu hình và triển khai.

Tổng cộng, CI/CD giúp tăng cường khả năng linh hoạt, đảm bảo chất lượng mã nguồn, giảm rủi ro, và giảm thời gian giữa việc phát triển và triển khai sản phẩm.

% CD là gì
% CD là gì
% CD là gì
% CD là gì
% CD là gì
% CD là gì
% CD là gì
% CD là gì
% CD là gì
% CD là gì


Khi một bối cảnh giới hạn đã được xác định, chúng ta cần đảm bảo rằng nó luôn ở trạng thái mới và hoạt động tốt như kỳ vọng. Đáp ứng nhu cầu doanh nghiệp phát triển thay đổi liên tục và nhanh chóng.
Khi cùng vận hành và phát triển xung đột có thể xảy ra ở cùng hoặc khác bối cảnh giới hạn.
=> Vì vậy, cần sử dụng việc tích hợp liên tục tạo ra một quy trình tự động và liên tục từ việc tích hợp mã nguồn, kiểm thử tự động giúp tăng cường chất lượng phần mềm, giảm thời gian và rủi ro trong quá trình phát triển phần mềm.

<!--$VD: jenkins-->
<!--unit test-->
<!--test tích hợp-->





% vở
% thời gian khác nhau

% http://localhost nên không có CD