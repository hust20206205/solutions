%!<! - - Thiết kế hướng miền cung cấp 2 loại mẫu: - - >
Các mô hình chiến lược (Strategic Patterns): chia một vấn đề kinh doanh lớn và phức tạp thành các phần nhỏ hơn với ranh giới được xác định rõ ràng.
Các mẫu kỹ thuật (Tactical Patterns): chuyển các mô hình khái niệm sang các thiết kế dịch vụ và ứng dụng phần mềm.

% Hai vấn đề cốt lõi được giải quyết bằng thiết kế hướng miền:
% 1. Thiết kế, phân chia cơ cấu doanh nghiệp như thế nào cho hợp lý?
% 2. Kiến trúc kỹ thuật có phù hợp với kiến trúc kinh doanh không?

% hình vẽ ở đâu đó mà có 2 cái đấy? tùn quốc???


<!--@Các khuôn mẫu trong thiết kế hướng miền-->
<!--Thiết kế hướng miền cung cấp 2 loại mẫu:-->

Các mô hình chiến lược (Strategic Patterns): chia một vấn đề kinh doanh lớn và phức tạp thành các phần nhỏ hơn với ranh giới được xác định rõ ràng.
Các mẫu kỹ thuật (Tactical Patterns): chuyển các mô hình khái niệm sang các thiết kế dịch vụ và ứng dụng phần mềm.

<!--@Các mô hình chiến lược (Strategic Patterns)-->



<!--@Các mô hình chiến lược (Strategic Patterns)-->

![](pictures/CacMoHinhChienLuoc/0_CacMoHinhChienLuoc.png)

<!--Sơ đồ về các mô hình chiến lược-->
<!--$ Vẽ lại sau:-->
<!--Bối cảnh giới hạn (Bounded Context)-->
<!--[Giữ cho mô hình thống nhất] Tích hợp Liên tục (Continuous Integration)-->
<!--[Tính nhất quán trong trao đổi] Ngôn ngữ chung (Ubiquitous Language)-->

<!--[Tổng quan mối quan hệ] Bản đồ bối cảnh (Context Maps)-->

<!--Symmetric Relationship: Separate ways, Shared Kernel-->
<!--Asymmetric Relationship: Customer-Supplier, Conformist, Anti Corruption Layer-->
<!---->

<!--One-to-Many Relationship: Open Host Service, Published Language-->

<!--dịch và cách ly đơn phương với-->
<!--[ lớp ] lớp (Context Maps)-->
<!--"Bản đồ bối cảnh dịch chuyển và cách ly một cách đơn phương để tạo thành cấu trúc lớp."-->
<!--Tách biệt-->

<!--$ Vẽ lại sau:-->