Miền phụ cốt lõi là điểm khác biệt quan trọng cho doanh nghiệp.

Miền phụ cốt lõi   tập trung vào mục tiêu của người dùng, giúp phân biệt các doanh nghiệp và làm cho các doanh nghiệp   có giá trị.  

Thành công của một doanh nghiệp nằm ở miền phụ cốt lõi, mỗi doanh nghiệp trong một ngành cụ thể hoạt động khác nhau trong các miền phụ cốt lõi để đạt được một số lợi thế so với đối thủ cạnh tranh.

$\Rightarrow$ Doanh nghiệp luôn tìm cách thực hiện những điều khác biệt trong các miền phụ cốt lõi này để có được một số lợi thế cạnh tranh.

Ví dụ: Trong miền thẻ tín dụng , miền phụ cốt lõi có thể  là  \textit{"phát hành thẻ"}    chịu trách nhiệm về quá trình phát hành thẻ tín dụng   cho khách hàng.  Miền phụ cốt lõi này bao gồm các nhiệm vụ như:  thu thập thông tin khách hàng, thực hiện kiểm tra tín dụng,     kích hoạt thẻ , \dots

% là một phần của hệ thống phần mềm chứa logic và quy trình kinh doanh chính, thể hiện trung tâm chức năng của ứng dụng. 

% Đây là phần quan trọng và có giá trị nhất của hệ thống và việc triển khai nó có ý nghĩa quyết định đối với sự thành công của phần mềm. 

% Miền cốt lõi được xác định thông qua phân tích cẩn thận về miền vấn đề cũng như các quy tắc và quy trình kinh doanh tương ứng của nó và nó phải được xác định rõ ràng, theo mô - đun và có thể bảo trì được.

% Trong thiết kế hướng miền, miền lõi thường được gói gọn trong một tập hợp các đối tượng gắn kết và liên kết lỏng lẻo, được gọi là mô hình miền, mô hình hóa các khái niệm, quy tắc và quy trình kinh doanh của miền vấn đề.



% @Cần lưu ý rằng ý tưởng về miền phụ cốt lõi, hỗ trợ và chung có thể khác nhau ngay cả đối với các doanh nghiệp hoạt động trong cùng một miền. Điều này là do các miền phụ và vai trò của chúng được xác định theo nhu cầu kinh doanh và bối cảnh cụ thể của mỗi tổ chức. Ví dụ:
  

% Cần lưu ý rằng ý tưởng về miền phụ cốt lõi, hỗ trợ và chung có thể khác nhau ngay cả đối với các doanh nghiệp hoạt động trong cùng một miền. Điều này là do các miền phụ và vai trò của chúng được xác định theo nhu cầu kinh doanh và bối cảnh cụ thể của mỗi tổ chức. Ví dụ: trong miền thẻ tín dụng, một công ty tập trung vào các chương trình phần thưởng thẻ tín dụng có thể coi miền phụ của chương trình phần thưởng là cốt lõi, trong khi một công ty khác có trọng tâm khác có thể coi nó là hỗ trợ hoặc chung chung. Tương tự, một công ty tập trung mạnh vào việc ngăn chặn gian lận có thể coi miền phụ phát hiện gian lận là cốt lõi, trong khi một công ty khác có thể coi nó là miền hỗ trợ hoặc chung chung. Do đó, điều quan trọng là mỗi tổ chức phải xác định và ưu tiên các miền phụ cụ thể dựa trên nhu cầu và mục tiêu kinh doanh riêng của họ.

% % %!<! - - Core Domain https:// thiết kế hướng miền - practitioners.com/home/glossary/domain/core - domain/ - - >
 

% Cốt lõi nổi bật

% Trong ngữ cảnh của thiết kế hướng miền, phần cốt lõi được đánh dấu đề cập đến phần quan trọng và phức tạp nhất của hệ thống phần mềm, thể hiện logic miền cốt lõi và mang lại giá trị cao nhất cho doanh nghiệp. Phần lõi này phải được tách biệt khỏi các miền phụ hỗ trợ và chung, đồng thời phải được phát triển và duy trì bởi các chuyên gia và nhà phát triển miền có hiểu biết sâu sắc về doanh nghiệp cũng như các yêu cầu của nó. Lõi được đánh dấu phải được bảo vệ và cách ly khỏi những thay đổi và sửa đổi không liên quan trực tiếp đến miền lõi, điều này đạt được thông qua việc sử dụng các bối cảnh giới hạn, các sự kiện miền và các khái niệm thiết kế hướng miền khác. Bằng cách giữ phần lõi được đánh dấu tách biệt khỏi các miền phụ hỗ trợ và chung, hệ thống có thể duy trì mức độ gắn kết và mô đun hóa cao, giúp dễ hiểu, duy trì và phát triển hơn theo thời gian.

% % %!<! - - Highlighted Core : https:// thiết kế hướng miền - practitioners.com/highlighted - core - - > 

% Lõi tách biệt

% Lõi tách biệt là một mẫu thiết kế hướng miền bao gồm việc tách miền lõi thành các phần hoặc mô - đun nhỏ hơn, độc lập, mỗi phần có bối cảnh giới hạn riêng. Điều này cho phép tính linh hoạt và khả năng mở rộng cao hơn trong việc thiết kế các hệ thống phức tạp, cũng như cải thiện tính mô - đun và khả năng bảo trì. Ý tưởng là để tránh việc có một miền cốt lõi nguyên khối, được liên kết chặt chẽ, có thể trở nên khó thay đổi hoặc duy trì theo thời gian. Bằng cách tách phần cốt lõi thành các phần nhỏ hơn, tập trung, mỗi phần có bối cảnh và trách nhiệm riêng, các nhóm có thể làm việc hiệu quả hơn và thực hiện các thay đổi dễ dàng hơn. Mẫu lõi tách biệt thường được sử dụng kết hợp với các mẫu thiết kế hướng miền khác, chẳng hạn như bối cảnh giới hạn và bản đồ bối cảnh, để tạo ra kiến trúc có cấu trúc tốt và linh hoạt cho các hệ thống quy mô lớn.
 