Trong quá trình hoạt động kinh doanh, không phải mọi doanh nghiệp đều giữ nguyên mô hình kinh doanh được đưa ra ban đầu của mình. Khi quy mô thị trường thay đổi, việc chuyển đổi mô hình kinh doanh là điều cần thiết. Chuyển đổi kinh doanh như một công cụ linh hoạt giúp các doanh nghiệp có thể phát triển và tồn tại giữa các đối thủ của mình.

Ví dụ:

\begin{itemize}

\item Google bắt đầu như một công cụ tìm kiếm trực tuyến, nhưng sau đó đã mở rộng và thay đổi mô hình kinh doanh qua nhiều dịch vụ và sản phẩm khác nhau như: Dịch vụ đám mây Google Cloud Platform, thư điện tử Gmail, bản đồ Google Maps, lưu trữ tập tin Google Drive, \dots

\item Amazon từ hiệu sách trực tuyến đã trở thành thị trường cho nhà cung cấp khác như: Thương mại điện tử, Dịch vụ đám mây Amazon Web Services (AWS), \dots

\begin{figure}[H]

\centering

\includegraphics[width = 0.5\textwidth]{pictures/KienTrucViDichVuAmazon/main.png}

\caption{Kiến trúc vi dịch vụ của Amazon}

\end{figure}

\end{itemize}

Đối với những doanh nghiệp không chuyển đổi kinh doanh sẽ không thể tồn tại.

Ví dụ: Gần đây, dịch vụ giao đồ ăn Baemin đã rời khỏi thị trường Việt Nam cũng do sức ép từ các đối thủ khác khiến Baemin khó cạnh tranh trong mảng kinh doanh cốt lõi là giao đồ ăn. Các đối thủ này không chỉ cung cấp dịch vụ giao đồ ăn mà còn có đặt xe, giao hàng,...

\begin{figure}[H]

\centering

\includegraphics[width = 0.5\textwidth]{pictures/Baemin/main.png}

\caption{Baemin đã rời khỏi thị trường Việt Nam}

\end{figure} 

Hiện nay, các tổ chức doanh nghiệp có nhu cầu chuyển đổi kinh doanh để có thể tồn tại và phát triển khi thị trường thay đổi. Từ đó, đáp ứng nhu cầu của khách hàng, mang lại ưu thế cạnh tranh so với các đối thủ. Do đó, các doanh nghiệp cần hệ thống chuyển đổi nhanh chóng đáp ứng nhu cầu của mô hình kinh doanh và mong đợi của khách hàng.

$\Rightarrow$ Kiến trúc vi dịch vụ giải quyết những thách thức và hỗ trợ doanh nghiệp chuyển đổi dễ dàng. Tuy nhiên, để xây dựng được kiến trúc vi dịch vụ tốt, cần phải tạo ra các dịch vụ nhỏ phù hợp và duy trì tính độc lập. Ở đồ án này, em sử dụng thiết kế hướng miền để phân tích và xây dựng kiến trúc vi dịch vụ.    Thiết kế hướng miền xác định và tổ chức các dịch vụ dựa trên việc hiểu rõ về lĩnh vực kinh doanh, giúp dự án phản ánh đúng các quy trình và quy tắc kinh doanh. 
